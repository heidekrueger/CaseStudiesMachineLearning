%!TEX program = xelatex
\documentclass[10pt]{beamer}

\usetheme[progressbar=frametitle, noframetitleoffset, block=fill]{m}
%\definecolor{TUMblue}{RGB}{55,55,255}
%\setbeamercolor{alerted text}{fg=TUMblue}

\usepackage{booktabs}
\usepackage[scale=2]{ccicons}

\usepackage{pgfplots}
\usepackage{tikz}
\usepgfplotslibrary{dateplot}
\usepackage{caption}

\newlength\figureheight
\newlength\figurewidth
\DeclareMathOperator{\prox}{prox}
\DeclareMathOperator{\argmin}{argmin}

\title{Stochastic Optimization in Machine Learning}
\subtitle{Case Studies in Nonlinear Optimization}
\date{\today}
\author{F. Bauer \and S. Chambon \and R. Halbig \and S. Heidekrüger \and J. Heuke}
\institute{Technische Universität München}
%\titlegraphic{\hfill\includegraphics[height=1.5cm]{logo.eps}}

\begin{document}

\maketitle

\plain{
  \begin{quote}
    We're not running out of data anytime soon. It's maybe the only resource that grows exponentially.
    \\
    \flushright{\alert{Andreas Weigend}}
  \end{quote}
  }


\begin{frame}
  \frametitle{Outline}
  \setbeamertemplate{section in toc}[sections numbered]
  \tableofcontents[hideallsubsections]
\end{frame}

\section{Introduction}

  \begin{frame}[t]\frametitle{What is Machine Learning?}
	  	Implementation of autonomously learning software for:
        \begin{itemize}
        	\item Discovery of patterns and relationships in data
        	\item Prediction of future events
        \end{itemize}
        \alert{Examples:}
        \begin{columns}[T]
        	\begin{column}{.25\textwidth}
        		Higgs-Boson\\
        		\includegraphics[width = \linewidth]{CMS_Higgs-event.jpg}
        	\end{column}\hfill
        	\begin{column}{.25\textwidth}
        		Compressed Sensing\\
        		\includegraphics[width = \linewidth]{plot_tomography_l1_reconstruction_001.png}
        	\end{column}\hfill
        	\begin{column}{.25\textwidth}
        		EEG\\
        		\includegraphics[width = \linewidth]{eeg_pic.jpg}
        	\end{column}\hfill
        	\begin{column}{.25\textwidth}
        		Image Reconstruction\\
        		\includegraphics[width = \linewidth]{lena_pic.jpg}
        	\end{column}
        \end{columns}
  \end{frame}

  \begin{frame}
    \frametitle{Challenges in Machine Learning}
      \begin{itemize}
        \item Massive amounts of training data 
        \item Construction of very large models
        \item Handling high memory/computational demands
      \end{itemize}
      \vspace{36pt}
    \centering\alert{Stochastic Methods}
  \end{frame}
  
  \begin{frame}{Stochastic Framework}
  	
  \end{frame}

\section{A Stochastic Quasi-Newton Method}
 \plain{Classification\\
 	\vspace{10pt}
 	\alert{Did we just detect a Higgs-Boson?}
 	\vspace{15pt}\\
 	\includegraphics[width = 0.5\textwidth]{CMS_Higgs-event.jpg}}
  \begin{frame}
    \frametitle{Stochastic Quasi Newton}
      What is it?
      Why?
      Main ideas, high-level pseudo code overview?
      short bfgs repitition?
      Extreme Cases (L-BFGS, SGD)
  \end{frame}

  \begin{frame}
    \frametitle{HIGGS-Dataset}
    Explain the Dataset quickly.
    Why is this good for SQN testing?
    Why is it challenging? (file size etc)
  \end{frame}

  \begin{frame}
    \frametitle{Behavior}
      Pretty picures about the behaviour of SQN on HIGGS
      and comparison with traditional SGD
  \end{frame}

 \section{Proximal Method}
 \plain{Image Reconstruction\\
 	\vspace{10pt}
 	\alert{What did the original image look like?}
 	\vspace{15pt}\\
 	\includegraphics[width = 0.8\textwidth]{plot_tomography_l1_reconstruction_001.png}}
 
 
   \begin{frame}\frametitle{Proximal Method}
       \begin{flalign*}
       	\text{\alert{Problem}}&&
       	\min_x &\;F(x) := \underbrace{f(x)}_{smooth} \quad + \underbrace{h(x)}_{non-smooth}&
       \end{flalign*}
       \pause
       \begin{flalign*}
       	\text{\alert{Proximity Operator}}&&\prox_f(v) = &\underset{x}{\argmin} \; \bigl( f(x) + \frac{1}{2} \lVert x - v \rVert^2_2 \bigr)&
       \end{flalign*}
			\centering\includegraphics[width = 0.5\textwidth]{prox_boyd.jpg}
   \end{frame}
   
   \begin{frame}{Proximal Method}
   	\alert{Traditional Proximal Gradient Step:}
   	\begin{equation*}
   	x_{k+1} = \prox_{\lambda_kh}(x_k - \lambda_k\nabla f(x_k))
   	\end{equation*}
   	\alert{Quasi-Newton Proximal Step:}
   	\begin{equation*}
   	x_{k+1} = \prox_h^{B_k}(x_k - B_k^{-1}\nabla f(x_k)),
   	\end{equation*}
   	with $B_k = \underbrace{D_k}_{diag} + \underbrace{u_k}_{\in\mathbb{R}^n}u_k^T$.
   \end{frame}
   
   \begin{frame}{Proximal Method}
   	\begin{columns}[T]
   		\begin{column}{.5\textwidth}
   			$F(x) = \lVert Ax - b \rVert + \lambda \lVert x \rVert_1$\\
   			$A \in \mathbb{R}^{1500 \times 3000},\:b \in \mathbb{R}^{1500}$\\
   			$A_{ij},\:b_i\:$ \textasciitilde $\:\mathcal{N}(0,1)$, $\:\lambda = 0.1$\\
   			\vspace{10pt}
   			\resizebox{\linewidth}{!}{% This file was created by matplotlib v0.1.0.
% Copyright (c) 2010--2014, Nico Schl�mer <nico.schloemer@gmail.com>
% All rights reserved.
% 
% The lastest updates can be retrieved from
% 
% https://github.com/nschloe/matplotlib2tikz
% 
% where you can also submit bug reports and leavecomments.
% 
\begin{tikzpicture}

\begin{axis}[
xlabel={Number of Iterations},
ylabel={Function Value},
xmin=0, xmax=60,
ymin=3.73758288736179, ymax=100000,
ymode=log,
axis on top,
legend entries={{0SR1},{ProxGrad},{L-BFGS-B}}
]
\addplot [thick, red]
coordinates {
(0,23930.000884189)
(1,7180.45388604586)
(2,1602.10980923053)
(3,580.81114454538)
(4,519.996842220967)
(5,203.222944097077)
(6,121.922299073436)
(7,51.6523543005163)
(8,59.2695972484554)
(9,30.819611673989)
(10,25.614751730395)
(11,21.7592732486774)
(12,22.654003528062)
(13,20.15049001864)
(14,19.850460675318)
(15,18.3912111072782)
(16,23.73822362313)
(17,17.8866027335036)
(18,17.8196671916423)
(19,17.7775123499805)
(20,18.0615078838142)
(21,17.7506027746239)
(22,17.7424949039984)
(23,17.7297358775166)
(24,18.0476551294605)
(25,17.7168284338285)
(26,17.7118227102102)
(27,17.897336767244)
(28,17.7264425474283)
(29,17.6583909070735)
(30,17.6466963489978)
(31,17.5726603204587)
(32,17.9785494963624)
(33,17.5007993751903)
(34,17.4774762719696)
(35,17.4693572907426)
(36,18.0844886660129)
(37,17.4544077868014)
(38,17.4488948638463)
(39,17.3764214969115)
(40,17.5899917220372)
(41,17.1689206218387)
(42,17.0926131263914)
(43,17.193691075103)
(44,19.0707895757655)
(45,16.9630825309593)
(46,16.9036371416858)
(47,16.8413981063643)
(48,16.8743377801165)
(49,16.6801578404075)
(50,16.5909092149981)
(51,15.9390711318578)
(52,15.2212036355576)
(53,13.1777255991126)
(54,11.3509943341348)
(55,10.0944946484959)
(56,9.71465222794144)
(57,9.40843618327981)
(58,9.31008072896225)
(59,9.12293542675185)
(60,8.97211792181375)
(61,9.96955580951941)
(62,8.8971778312592)
(63,8.89032770510936)
(64,8.846532027156)
(65,9.14275424383174)
(66,8.76428614407871)
(67,8.74549967678879)
(68,8.81264423855369)
(69,8.92224051152909)
(70,8.71372433712952)
(71,8.69686120584995)
(72,8.86066032777056)
(73,8.64994348063148)
(74,8.64177834237841)
(75,8.61074429525064)
(76,8.97413158797716)
(77,8.57239195780947)
(78,8.56084322589053)
(79,8.56535473569669)
(80,8.70378144008176)
(81,8.55111754917089)
(82,8.54693544975204)
(83,8.66605947747419)
(84,8.49089417601578)
(85,8.48541554539908)
(86,8.44561206967175)
(87,8.58014556574557)
(88,8.40335231266141)
(89,8.3639219974089)
(90,8.30542294510122)
(91,8.53249537616582)
(92,8.29202808913906)
(93,8.25359007491097)
(94,8.22495187877562)
(95,8.2493220091081)
(96,8.31589617161871)
(97,8.19976279420752)
(98,8.18967260723272)
(99,8.37069393810889)
(100,8.23037470713979)
(101,8.17568861437345)
(102,8.16381875081452)
(103,8.13194423644109)
(104,8.40270307214338)
(105,8.10922711655525)
(106,8.1020911060756)
(107,8.27112654264333)
(108,8.11788920739675)
(109,8.04379136654635)
(110,8.01474013833664)
(111,8.22847077290569)
(112,7.99161827419456)
(113,7.98213107776018)
(114,7.9705887091828)
(115,7.9971843509347)
(116,7.97282647175694)
(117,7.93351580269397)
(118,7.90255155411873)
(119,7.99776642630242)
(120,7.91007457288172)
(121,7.88080319635567)
(122,7.87313798696741)
(123,7.84498867344205)
(124,7.79976321415661)
(125,7.98938146895335)
(126,7.60326661377574)
(127,7.52462148642087)
(128,7.34995983380932)
(129,8.10221113586057)
(130,6.9022590632367)
(131,6.80991180342276)
(132,6.88125458344041)
(133,6.8507836380378)
(134,6.71508640280255)
(135,6.70000126952806)
(136,6.85683699907552)
(137,6.55612655712418)
(138,6.52963963426299)
(139,6.97829713337909)
(140,6.89977526632402)
(141,6.35048115764401)
(142,6.26110580435994)
(143,7.07330692372099)
(144,6.31929255870998)
(145,6.18205067053115)
(146,6.16701844730448)
(147,6.11867830940032)
(148,6.13745017226778)
(149,6.09937758198822)
(150,6.09582597813946)
(151,6.06411366979733)
(152,6.14055772714741)
(153,5.94401023689714)
(154,5.89132811574154)
(155,5.68275758796475)
(156,6.12050945318889)
(157,5.51639726280943)
(158,5.49510016102023)
(159,5.44993976275907)
(160,5.58077524396)
(161,5.4347375308189)
(162,5.43206832958341)
(163,5.43955939273127)
(164,5.42309581140177)
(165,5.40496331872906)
(166,5.39776088705187)
(167,5.38504134703711)
(168,5.34783309143708)
(169,5.34382142546905)
(170,5.22681123881669)
(171,5.18440702402646)
(172,5.24431581083607)
(173,5.19540210255316)
(174,5.15062386574747)
(175,5.14416764148083)
(176,5.13341463369933)
(177,5.18173927919764)
(178,5.12141723919665)
(179,5.1184042877569)
(180,5.13109621710162)
(181,5.13363728292441)
(182,5.10312125837393)
(183,5.09785912201646)
(184,5.10735000469187)
(185,5.0926845635739)
(186,5.09034966393374)
(187,5.08821841949242)
(188,5.07488780550707)
(189,5.08849371417169)
(190,5.02222005421612)
(191,4.99207726277193)
(192,4.87914769868332)
(193,4.82925094001405)
(194,4.75376544240897)
(195,4.74473141754993)
(196,4.73009186342567)
(197,4.7490165043051)
(198,4.71971253985978)
(199,4.71714855523501)
(200,4.71978228094637)
(201,4.71455233520521)
(202,4.70787717391382)
(203,4.70435691995885)
(204,4.71018713940015)
(205,4.70020880519862)
(206,4.69615991183553)
(207,4.68871236789081)
(208,4.70198318498291)
(209,4.67512750689444)
(210,4.66603041063657)
(211,4.6415330874473)
(212,4.66782835952312)
(213,4.59443050387119)
(214,4.58660881032307)
(215,4.57713270754925)
(216,4.57506147736507)
(217,4.55372467737904)
(218,4.54519000896493)
(219,4.51435255105705)
(220,4.49560297960765)
(221,4.4774261009363)
(222,4.45191612991977)
(223,4.44720601916045)
(224,4.43599541033599)
(225,4.43276372123928)
(226,4.42934987534438)
(227,4.42128447486717)
(228,4.41809900599509)
(229,4.41547968748632)
(230,4.40155511455402)
(231,4.41165288105235)
(232,4.38775480844803)
(233,4.38489415754439)
(234,4.37688411667431)
(235,4.41441040680117)
(236,4.36767328848501)
(237,4.36149394650408)
(238,4.35441829353438)
(239,4.36849669898586)
(240,4.3509171724856)
(241,4.34845410776296)
(242,4.34440488118282)
(243,4.35165330633997)
(244,4.33990992103764)
(245,4.33719479734351)
(246,4.3270833745045)
(247,4.35093123292709)
(248,4.31867865163969)
(249,4.31601638116326)
(250,4.31314367797616)
(251,4.31774077386986)
(252,4.31128560370525)
(253,4.31075605785998)
(254,4.31211330242688)
(255,4.30829158320082)
(256,4.30473431044141)
(257,4.30077080190382)
(258,4.3062450713516)
(259,4.29844901192489)
(260,4.29534059217617)
(261,4.29341934045865)
(262,4.2889643717035)
(263,4.29711605951817)
(264,4.28401032948353)
(265,4.28275621895348)
(266,4.27964775582809)
(267,4.27891081839423)
(268,4.27600099347314)
(269,4.27332371487828)
(270,4.26542370206892)
(271,4.26547168294958)
(272,4.2518153012882)
(273,4.24504962575816)
(274,4.22596279759789)
(275,4.26679021875544)
(276,4.21859182459061)
(277,4.21680022623184)
(278,4.21345503474627)
(279,4.24742625597384)
(280,4.20294897654986)
(281,4.20046642199474)
(282,4.20866691822924)
(283,4.20151947744105)
(284,4.19322180171457)
(285,4.19143384888523)
(286,4.19590540956122)
(287,4.18858814000993)
(288,4.18689422656362)
(289,4.18527033142004)
(290,4.18359868520006)
(291,4.18386283880689)
(292,4.17901119799218)
(293,4.17660648867894)
(294,4.17388866246303)
(295,4.17437414476721)
(296,4.17052307165345)
(297,4.16896740987079)
(298,4.1657396188267)
(299,4.16207434351199)
(300,4.15562934480001)
(301,4.15221122706091)
(302,4.1468309024372)
(303,4.142731071332)
(304,4.13958253670866)
(305,4.13807034824092)
(306,4.13595295531159)
(307,4.1342975548661)
(308,4.13276508494234)
(309,4.12950517703989)
(310,4.12583870077403)
(311,4.12269500826494)
(312,4.12042283333458)
(313,4.1184613334651)
(314,4.11216151831005)
(315,4.10991196734199)
(316,4.10834813347944)
(317,4.10466833254876)
(318,4.10169796783637)
(319,4.09599684349197)
(320,4.09271550063422)
(321,4.08970990189555)
(322,4.0878053507012)
(323,4.08446831877605)
(324,4.08159081574492)
(325,4.08008004118005)
(326,4.07805189915379)
(327,4.07481683437214)
(328,4.07230668289582)
(329,4.07117704315002)
(330,4.06914148280182)
(331,4.06724141168374)
(332,4.06429414211129)
(333,4.06245476835401)
(334,4.05796516965501)
(335,4.05586999855709)
(336,4.05358214753478)
(337,4.05167372792326)
(338,4.04971536590856)
(339,4.04772865257353)
(340,4.0462682628228)
(341,4.0443337920353)
(342,4.04298815070817)
(343,4.04159669736941)
(344,4.03948635470198)
(345,4.03756551652368)
(346,4.03633060540019)
(347,4.03480070691241)
(348,4.03179842576502)
(349,4.02814764575733)
(350,4.02653745322011)
(351,4.02529318739869)
(352,4.02355851516367)
(353,4.02110622069766)
(354,4.01939315595809)
(355,4.01796252772672)
(356,4.01641766788388)
(357,4.01457265423333)
(358,4.01338887769868)
(359,4.01194794091037)
(360,4.0093231166358)
(361,4.00514939947506)
(362,4.00351546246402)
(363,4.00200907649201)
(364,3.99992281618718)
(365,3.99797430431843)
(366,3.99334918832357)
(367,3.99173253462353)
(368,3.98976171378918)
(369,3.9856036689915)
(370,3.98250512795621)
(371,3.97977832742336)
(372,3.97841061366499)
(373,3.97729992567629)
(374,3.97548532209751)
(375,3.97365457362757)
(376,3.9717840292761)
(377,3.97053456718478)
(378,3.96712437614579)
(379,3.96575290280055)
(380,3.96416468887183)
(381,3.96227331516023)
(382,3.96015123118512)
(383,3.95697348034623)
(384,3.95558559053402)
(385,3.95413029288639)
(386,3.95193285532975)
(387,3.94995248168693)
(388,3.94852118630789)
(389,3.94737903792242)
(390,3.94574874973164)
(391,3.94415680391129)
(392,3.94289188577825)
(393,3.94197297899114)
(394,3.93952283120682)
(395,3.93825160225002)
(396,3.93568543190763)
(397,3.93261556716918)
(398,3.930516268874)
(399,3.92949481126826)
(400,3.92769455749655)
(401,3.92597962780121)
(402,3.92405249564003)
(403,3.92298717514001)
(404,3.92024570123432)
(405,3.91798517635568)
(406,3.91605996843274)
(407,3.91484999273038)
(408,3.91348201245275)
(409,3.91205196729859)
(410,3.9101782087323)
(411,3.90907587383236)
(412,3.90769249170707)
(413,3.90573876369278)
(414,3.90393689253081)
(415,3.90248294606086)
(416,3.90162096816003)
(417,3.8997429477221)
(418,3.89848876411662)
(419,3.89595772299443)
(420,3.89340260315083)
(421,3.89169481233147)
(422,3.8905143792272)
(423,3.8893144775389)
(424,3.88810193082712)
(425,3.88648956528796)
(426,3.88543899169958)
(427,3.88434285366883)
(428,3.88275274532349)
(429,3.88074689719771)
(430,3.87955433051444)
(431,3.87854502418123)
(432,3.87741869136827)
(433,3.87615021891571)
(434,3.8748887963834)
(435,3.87401131226095)
(436,3.87269076174271)
(437,3.87130443426161)
(438,3.86889963678931)
(439,3.86766339312703)
(440,3.86591362701932)
(441,3.86435580437818)
(442,3.86263943159158)
(443,3.86160647111943)
(444,3.86069532203059)
(445,3.85921672643717)
(446,3.85780872044254)
(447,3.85689592512765)
(448,3.85587967890622)
(449,3.85403970348962)
(450,3.85177334918022)
(451,3.85069256255907)
(452,3.84914024782396)
(453,3.84711073784714)
(454,3.84586764734895)
(455,3.84488331013994)
(456,3.84374459931887)
(457,3.84258662457118)
(458,3.84159880585769)
(459,3.84071959478942)
(460,3.83956519914057)
(461,3.83839955250542)
(462,3.83752516217005)
(463,3.8364681708824)
(464,3.83475783495541)
(465,3.83331836750638)
(466,3.83238945048464)
(467,3.83153118390011)
(468,3.83030890070573)
(469,3.82888686851538)
(470,3.82797660550542)
(471,3.82700377061506)
(472,3.82597201441229)
(473,3.82484933367717)
(474,3.82411221757026)
(475,3.82314129973166)
(476,3.8215730346757)
(477,3.81989992430987)
(478,3.81901623535658)
(479,3.81795378904028)
(480,3.81632915808355)
(481,3.81533693137303)
(482,3.81423086285225)
(483,3.81319647591909)
(484,3.81191778283785)
(485,3.81107001334777)
(486,3.8103815294915)
(487,3.80953390229388)
(488,3.80827012532122)
(489,3.80702843221915)
(490,3.80635125361567)
(491,3.80547452706138)
(492,3.80395857022581)
(493,3.80304892442227)
(494,3.80228519530425)
(495,3.8015706361817)
(496,3.80077879827292)
(497,3.8001325670707)
(498,3.79948703152183)
(499,3.79881854489286)
(500,3.79775780431932)
(501,3.79703870764334)
(502,3.79633371289658)
(503,3.79528553833168)
(504,3.79428055004768)
(505,3.79323977232523)
(506,3.79254988970646)
(507,3.79153608118026)
(508,3.79082798087948)
(509,3.78994539411923)
(510,3.78894967876444)
(511,3.78788961864922)
(512,3.78719691851525)
(513,3.78666179387695)
(514,3.78596156857987)
(515,3.78520540040946)
(516,3.78448016559012)
(517,3.78388751805597)
(518,3.78314387190005)
(519,3.7824676279489)
(520,3.78195107164388)
(521,3.78125311905374)
(522,3.78046746083349)
(523,3.77957536512099)
(524,3.77892640810143)
(525,3.77824252607635)
(526,3.77767065902712)
(527,3.77707657851077)
(528,3.77649208597268)
(529,3.7759300063137)
(530,3.77538358825486)
(531,3.77485161860709)
(532,3.77430624311249)
(533,3.77375223887874)
(534,3.77321104279342)
(535,3.77273242062598)
(536,3.77219168528069)
(537,3.77170600779641)
(538,3.77128528941916)
(539,3.77089013070214)
(540,3.77024662166515)
(541,3.76977715929525)
(542,3.76931146679233)
(543,3.76880302275762)
(544,3.76809956172276)
(545,3.76750536192979)
(546,3.76710916414852)
(547,3.76659986048121)
(548,3.76617371502299)
(549,3.76577241485064)
(550,3.76535064864266)
(551,3.76481990493601)
(552,3.76441175083281)
(553,3.76410339463009)
(554,3.76368061822461)
(555,3.76321380149803)
(556,3.76288022592054)
(557,3.76256601896596)
(558,3.76216309488332)
(559,3.761795693057)
(560,3.76150354223073)
(561,3.76112864438217)
(562,3.76064017573548)
(563,3.76022980586341)
(564,3.75993451694517)
(565,3.75968143668119)
(566,3.7594045224648)
(567,3.75908943171438)
(568,3.75878164547724)
(569,3.75854689793333)
(570,3.7582424906445)
(571,3.75793903165706)
(572,3.75767756236639)
(573,3.75744597079137)
(574,3.75714153646432)
(575,3.75688711206787)
(576,3.75667015566394)
(577,3.75645647906458)
(578,3.75610494539715)
(579,3.75582014579273)
(580,3.75562943295601)
(581,3.75534553609542)
(582,3.75502417381085)
(583,3.75464369013743)
(584,3.75439444983922)
(585,3.75416681857923)
(586,3.75396285174997)
(587,3.75368322103135)
(588,3.7534706786618)
(589,3.75330504142941)
(590,3.75311566038413)
(591,3.75290362054593)
(592,3.75271969185744)
(593,3.75257129818678)
(594,3.75241264942064)
(595,3.7522280583499)
(596,3.75205469988612)
(597,3.75190819575032)
(598,3.7517626590235)
(599,3.7516058006531)
(600,3.75146560260072)
(601,3.75133494272921)
(602,3.75119503534922)
(603,3.75102037723042)
(604,3.75082668597587)
(605,3.75068676481354)
(606,3.75056747051451)
(607,3.75041781838715)
(608,3.75027463797489)
(609,3.75008712732238)
(610,3.74995732851354)
(611,3.74982178476451)
(612,3.74968371694349)
(613,3.7495285936222)
(614,3.74941296006289)
(615,3.74929911603172)
(616,3.74913726150537)
(617,3.74897805103517)
(618,3.74885705384141)
(619,3.74874963354451)
(620,3.74859161775803)
(621,3.74844889100133)
(622,3.74834278992655)
(623,3.74823883915335)
(624,3.74811365418672)
(625,3.74797828225327)
(626,3.74788246654743)
(627,3.74779959410751)
(628,3.74765022344617)
(629,3.74752767027)
(630,3.74741652128474)
(631,3.74732886333778)
(632,3.74721212443547)
(633,3.74711469319822)
(634,3.74702560643637)
(635,3.74692050358538)
(636,3.74683039793826)
(637,3.7467338058367)
(638,3.74663542230099)
(639,3.74651793374796)
(640,3.74642168315138)
(641,3.74636371576277)
(642,3.74624797803266)
(643,3.74614931110104)
(644,3.74606526901899)
(645,3.7460019867554)
(646,3.74593152978952)
(647,3.74584316102125)
(648,3.74574466453072)
(649,3.74566148777959)
(650,3.7456016893494)
(651,3.74553649377154)
(652,3.74546561773083)
(653,3.745381447244)
(654,3.74532102007605)
(655,3.74526159188137)
(656,3.7451950142971)
(657,3.74513328326609)
(658,3.74506771887444)
(659,3.74499347811469)
(660,3.74490380725585)
(661,3.74481710489874)
(662,3.74475316373512)
(663,3.74468106911017)
(664,3.74459336059384)
(665,3.74449975548279)
(666,3.74443718946213)
(667,3.7443756302593)
(668,3.74430307640286)
(669,3.74423234203767)
(670,3.74416515297006)
(671,3.74411518134775)
(672,3.74406317959127)
(673,3.74400852160073)
(674,3.74395010468026)
(675,3.74389852176432)
(676,3.74384505517438)
(677,3.74378165510622)
(678,3.74373536998093)
(679,3.74366747761197)
(680,3.74357277562586)
(681,3.74351617988416)
(682,3.74345821667858)
(683,3.74340965962514)
(684,3.7433665935811)
(685,3.74333013259509)
(686,3.74328028761075)
(687,3.74322247988481)
(688,3.74315135586653)
(689,3.74310666272858)
(690,3.7430634235613)
(691,3.74302093301781)
(692,3.74294178782519)
(693,3.74288642875492)
(694,3.74282451781943)
(695,3.74275586416584)
(696,3.74270062504022)
(697,3.74265698063228)
(698,3.74261078493935)
(699,3.74254825631828)
(700,3.74250476766571)
(701,3.74246859452954)
(702,3.74242291855706)
(703,3.74236374666149)
(704,3.74231110855055)
(705,3.74227756384092)
(706,3.74224207815006)
(707,3.74218943767502)
(708,3.74214330509948)
(709,3.74210817845599)
(710,3.7420717315306)
(711,3.74203152047682)
(712,3.7419903458656)
(713,3.74196060755244)
(714,3.74193384932532)
(715,3.74190839598638)
(716,3.74187825903688)
(717,3.74185410022818)
(718,3.74183194920258)
(719,3.74181160304062)
(720,3.74177874540215)
(721,3.74175473319888)
(722,3.74173262946911)
(723,3.74171284497681)
(724,3.74167804949778)
(725,3.7416411984487)
(726,3.74161090075658)
(727,3.7415812340218)
(728,3.74155611539142)
(729,3.74153487805987)
(730,3.74151339091035)
(731,3.74148753669067)
(732,3.74145328680621)
(733,3.74143276481216)
(734,3.74141322799996)
(735,3.74138671229007)
(736,3.74136439376675)
(737,3.74133941838526)
(738,3.74132234969917)
(739,3.74129485349071)
(740,3.74125184747336)
(741,3.74122666865854)
(742,3.74120262520594)
(743,3.74118042382181)
(744,3.74115458881404)
(745,3.7411327782176)
(746,3.74111854324206)
(747,3.74109251346467)
(748,3.74105890005783)
(749,3.7410259918781)
(750,3.74100336907399)
(751,3.74097584133875)
(752,3.74094155610841)
(753,3.74090984534707)
(754,3.74087833307684)
(755,3.74085294782907)
(756,3.74082857196165)
(757,3.74080397463984)
(758,3.74078198829117)
(759,3.74076020954738)
(760,3.74073467546796)
(761,3.74070650788942)
(762,3.74068586073433)
(763,3.74066855894915)
(764,3.74063928975958)
(765,3.74061364899738)
(766,3.74058760245367)
(767,3.7405671769452)
(768,3.74054176190312)
(769,3.74050975800182)
(770,3.74048711998747)
(771,3.74046532384755)
(772,3.74044546596015)
(773,3.74042513698655)
(774,3.7404080251903)
(775,3.74038633082624)
(776,3.74035675540909)
(777,3.7403331601501)
(778,3.74031439754291)
(779,3.74030036227755)
(780,3.74028473889477)
(781,3.74026089427677)
(782,3.74023698615883)
(783,3.74021840023201)
(784,3.74020271461056)
(785,3.7401793778023)
(786,3.74016176553833)
(787,3.74014325405571)
(788,3.74012839094366)
(789,3.74009756766588)
(790,3.74007122276407)
(791,3.7400540621925)
(792,3.74002753195669)
(793,3.73999969915007)
(794,3.73997945695549)
(795,3.73994979401525)
(796,3.73992703627021)
(797,3.73989866261752)
(798,3.73987908658785)
(799,3.73985613061492)
(800,3.73982824881314)
(801,3.73980248095223)
(802,3.73978208868868)
(803,3.73976522792638)
(804,3.73974663348216)
(805,3.73971735955802)
(806,3.73969965222043)
(807,3.73968223291466)
(808,3.73966165319271)
(809,3.73963675374451)
(810,3.73960934498171)
(811,3.73958863223683)
(812,3.73957074319721)
(813,3.73954618206766)
(814,3.73952497267199)
(815,3.73950592080853)
(816,3.73949214114282)
(817,3.73946824188981)
(818,3.73945144303768)
(819,3.73943386255338)
(820,3.73941490479722)
(821,3.73939545820391)
(822,3.73938053256158)
(823,3.73937161119739)
(824,3.73935227957639)
(825,3.73933362689217)
(826,3.73931087370792)
(827,3.73929602599376)
(828,3.73928521370935)
(829,3.73927561194478)
(830,3.73925375610562)
(831,3.73923790244632)
(832,3.73921915649662)
(833,3.73920713111434)
(834,3.73919271970968)
(835,3.73917258664557)
(836,3.73915761705733)
(837,3.73914421165319)
(838,3.73912728617475)
(839,3.7391117099442)
(840,3.73909365433055)
(841,3.73908051899921)
(842,3.73906963263531)
(843,3.73905969367928)
(844,3.73904879814693)
(845,3.73903931356062)
(846,3.73902992794148)
(847,3.73902245626219)
(848,3.73900672817484)
(849,3.73899821816021)
(850,3.73899063842558)
(851,3.73897714666595)
(852,3.73896617338543)
(853,3.73895361795179)
(854,3.73894554107593)
(855,3.73893898995414)
(856,3.73893158144997)
(857,3.73892201198589)
(858,3.73891386475841)
(859,3.73890782718117)
(860,3.73889991708818)
(861,3.73888815730623)
(862,3.73887837420678)
(863,3.73887134562543)
(864,3.73886014950315)
(865,3.73884846564701)
(866,3.73883480288308)
(867,3.73882386677594)
(868,3.73881596466138)
(869,3.73880813280328)
(870,3.73879913030641)
(871,3.73878879097343)
(872,3.73878166151021)
(873,3.73877555725501)
(874,3.73876650793524)
(875,3.73875862161415)
(876,3.73874917447512)
(877,3.73874158357103)
(878,3.73873058063019)
(879,3.73871983855582)
(880,3.73871251425391)
(881,3.73870614096508)
(882,3.73869867190344)
(883,3.73868898991226)
(884,3.73867975716229)
(885,3.73867133850102)
(886,3.73866461213373)
(887,3.73865815833177)
(888,3.7386487678763)
(889,3.7386390407896)
(890,3.73863069719786)
(891,3.73862275702807)
(892,3.73861689515872)
(893,3.73860972166786)
(894,3.73860349212175)
(895,3.73859651278253)
(896,3.73858781156957)
(897,3.73857603134044)
(898,3.73856751685302)
(899,3.73856146825392)
(900,3.73855062289585)
(901,3.73853946713225)
(902,3.73852827268226)
(903,3.73852114922868)
(904,3.7385106950112)
(905,3.7385015530988)
(906,3.73849008031748)
(907,3.73848368620631)
(908,3.73847563582143)
(909,3.73846275305727)
(910,3.73845561172462)
(911,3.73845029539306)
(912,3.73844382979155)
(913,3.73843533101044)
(914,3.73842835063973)
(915,3.73842176275498)
(916,3.73841421536803)
(917,3.73840913013374)
(918,3.73840364767521)
(919,3.73839773318718)
(920,3.73839138701912)
(921,3.73838250108545)
(922,3.73837709539469)
(923,3.73837336189215)
(924,3.7383671965923)
(925,3.7383594792087)
(926,3.73835321547978)
(927,3.73834895139464)
(928,3.73834481829251)
(929,3.73833467633875)
(930,3.73832856209702)
(931,3.738321173295)
(932,3.73831405727421)
(933,3.73830857762767)
(934,3.73830193022184)
(935,3.73829507228262)
(936,3.73828676375071)
(937,3.73827921075028)
(938,3.73827418307051)
(939,3.73826934651238)
(940,3.73826370417339)
(941,3.73825629708589)
(942,3.73825098259168)
(943,3.73824596705064)
(944,3.7382381604372)
(945,3.738230842765)
(946,3.73822118935351)
(947,3.73821669701143)
(948,3.73820926073113)
(949,3.73819967417343)
(950,3.73819419092195)
(951,3.73818869475105)
(952,3.73818245241404)
(953,3.73817512212339)
(954,3.7381709160093)
(955,3.73816662397702)
(956,3.73816128898538)
(957,3.73815424012405)
(958,3.73814809474663)
(959,3.73814281970791)
(960,3.73813752691553)
(961,3.73813251426639)
(962,3.73812799115518)
(963,3.73812396132077)
(964,3.7381201167569)
(965,3.73811627091309)
(966,3.73810898454529)
(967,3.73810351331011)
(968,3.73809833230564)
(969,3.73809518794971)
(970,3.73809085258716)
(971,3.73808603520853)
(972,3.73807896274891)
(973,3.73807504867394)
(974,3.73807214123749)
(975,3.73806886910099)
(976,3.73806343804419)
(977,3.73805503208483)
(978,3.7380516460052)
(979,3.73804782534891)
(980,3.7380396603955)
(981,3.73803615699193)
(982,3.73803226926532)
(983,3.73802582947157)
(984,3.73802131253667)
(985,3.73801669982391)
(986,3.73801380400576)
(987,3.73800835239561)
(988,3.73800480480318)
(989,3.73799885961324)
(990,3.7379934996031)
(991,3.73798884356009)
(992,3.7379844933136)
(993,3.73798174378352)
(994,3.73797867157368)
(995,3.73797552739654)
(996,3.73797222056717)
(997,3.73796928502145)
(998,3.73796712769717)
(999,3.73796244917576)
(1000,3.73795886872431)
(1001,3.73795621363599)
(1002,3.737954034773)
(1003,3.73795013823017)
(1004,3.73794612477552)
(1005,3.73794354535786)
(1006,3.73794083239536)
(1007,3.73793852511006)
(1008,3.737935448868)
(1009,3.73793350712922)
(1010,3.7379314343794)
(1011,3.73792833376877)
(1012,3.73792414724719)
(1013,3.7379213960211)
(1014,3.73791932485491)
(1015,3.73791595937145)
(1016,3.73791214134045)
(1017,3.73790895411229)
(1018,3.73790649336976)
(1019,3.73790355003503)
(1020,3.73790049644154)
(1021,3.73789698262809)
(1022,3.73789381835944)
(1023,3.7378914192543)
(1024,3.73788844934678)
(1025,3.73788528313874)
(1026,3.7378817625565)
(1027,3.73787951527683)
(1028,3.73787800821526)
(1029,3.7378755364729)
(1030,3.73787244123614)
(1031,3.73786726149042)
(1032,3.7378651395781)
(1033,3.73786396806278)
(1034,3.73785806935631)
(1035,3.73785479501547)
(1036,3.73785281238868)
(1037,3.73785143768922)
(1038,3.73784955728909)
(1039,3.73784702871634)
(1040,3.73784335822393)
(1041,3.73784121905668)
(1042,3.73783952987051)
(1043,3.73783795506459)
(1044,3.73783512317316)
(1045,3.73783285948799)
(1046,3.73783073466828)
(1047,3.73782942492894)
(1048,3.73782770275979)
(1049,3.73782488619734)
(1050,3.73782207031728)
(1051,3.73781978453762)
(1052,3.73781798687509)
(1053,3.73781642388851)
(1054,3.73781465127283)
(1055,3.73781275397751)
(1056,3.73781130792716)
(1057,3.73781005115785)
(1058,3.73780858859627)
(1059,3.73780712898731)
(1060,3.73780609251047)
(1061,3.73780483566953)
(1062,3.73780258184138)
(1063,3.73779905211149)
(1064,3.73779685887446)
(1065,3.73779520339098)
(1066,3.73779391386575)
(1067,3.73779113036419)
(1068,3.73778933045907)
(1069,3.73778745178796)
(1070,3.73778623957811)
(1071,3.73778398546632)
(1072,3.73778246691832)
(1073,3.73778116182345)
(1074,3.73777995239303)
(1075,3.73777847499299)
(1076,3.73777638132182)
(1077,3.7377749088876)
(1078,3.73777283638095)
(1079,3.73777168771525)
(1080,3.73777081150565)
(1081,3.73776954731008)
(1082,3.73776799836559)
(1083,3.73776647139632)
(1084,3.7377655680971)
(1085,3.73776479082134)
(1086,3.73776385868514)
(1087,3.73776256608389)
(1088,3.73776118522476)
(1089,3.73776047083588)
(1090,3.73775926305187)
(1091,3.73775756564635)
(1092,3.73775631655305)
(1093,3.73775557665796)
(1094,3.73775400191915)
(1095,3.73775253304987)
(1096,3.73775160343876)
(1097,3.73775012549399)
(1098,3.73774867466238)
(1099,3.73774759500421)
(1100,3.73774686411834)
(1101,3.73774557086208)
(1102,3.73774381309094)
(1103,3.73774264808728)
(1104,3.73774133697788)
(1105,3.73773984901162)
(1106,3.73773847562961)
(1107,3.73773681852295)
(1108,3.73773518315082)
(1109,3.73773366248561)
(1110,3.7377323002413)
(1111,3.73773133752454)
(1112,3.73773003308583)
(1113,3.73772899012056)
(1114,3.73772806028997)
(1115,3.73772714509731)
(1116,3.73772582516982)
(1117,3.7377239785843)
(1118,3.7377232073355)
(1119,3.73772260689154)
(1120,3.73771861559686)
(1121,3.73771756612217)
(1122,3.73771637638089)
(1123,3.73771459724721)
(1124,3.73771363254166)
(1125,3.7377127192486)
(1126,3.73771211989452)
(1127,3.73771166747312)
(1128,3.73771100165631)
(1129,3.73771029181356)
(1130,3.73770884534222)
(1131,3.73770729940469)
(1132,3.73770644766985)
(1133,3.73770575668162)
(1134,3.73770494172878)
(1135,3.7377042359973)
(1136,3.73770360993247)
(1137,3.73770295363523)
(1138,3.73770204710784)
(1139,3.73770135422735)
(1140,3.73770082556236)
(1141,3.7376998836284)
(1142,3.7376990462984)
(1143,3.73769736814266)
(1144,3.73769591967591)
(1145,3.73769495544059)
(1146,3.73769417038829)
(1147,3.73769313369961)
(1148,3.73769148771761)
(1149,3.73769058250868)
(1150,3.73768965190977)
(1151,3.73768872425276)
(1152,3.73768785213861)
(1153,3.73768725095523)
(1154,3.73768670628886)
(1155,3.73768577714168)
(1156,3.73768489425626)
(1157,3.73768425057952)
(1158,3.73768362555022)
(1159,3.73768317353421)
(1160,3.7376826910267)
(1161,3.73768235307943)
(1162,3.73768199255039)
(1163,3.73768155347612)
(1164,3.73768110438571)
(1165,3.73768057549603)
(1166,3.73768006870966)
(1167,3.73767955743052)
(1168,3.73767814601958)
(1169,3.73767778222665)
(1170,3.73767734418728)
(1171,3.73767656310651)
(1172,3.73767603210726)
(1173,3.73767549831176)
(1174,3.73767514912425)
(1175,3.73767474285055)
(1176,3.73767377017291)
(1177,3.73767327796882)
(1178,3.73767276312744)
(1179,3.73767234141855)
(1180,3.73767193233348)
(1181,3.73767161338212)
(1182,3.73767121949531)
(1183,3.7376706641179)
(1184,3.73767013673895)
(1185,3.73766981887848)
(1186,3.7376695427727)
(1187,3.73766922781491)
(1188,3.73766881454572)
(1189,3.73766854428889)
(1190,3.73766833448331)
(1191,3.73766804377534)
(1192,3.73766762207462)
(1193,3.73766679005467)
(1194,3.73766649799751)
(1195,3.7376663006698)
(1196,3.73766579409718)
(1197,3.73766476509563)
(1198,3.73766374819113)
(1199,3.73766342414792)
(1200,3.73766319860106)
(1201,3.73766267797265)
(1202,3.73766233471793)
(1203,3.73766202117655)
(1204,3.73766165819181)
(1205,3.73766126832734)
(1206,3.73766076484854)
(1207,3.73766053697472)
(1208,3.73766031068214)
(1209,3.7376596692715)
(1210,3.73765924429688)
(1211,3.73765891053244)
(1212,3.73765867411844)
(1213,3.73765829702792)
(1214,3.73765759472138)
(1215,3.73765730627193)
(1216,3.73765686452479)
(1217,3.73765604469132)
(1218,3.73765572450469)
(1219,3.73765544041223)
(1220,3.73765463450631)
(1221,3.73765429708738)
(1222,3.73765407432424)
(1223,3.73765346398549)
(1224,3.73765308612734)
(1225,3.73765233568442)
(1226,3.737651763711)
(1227,3.73765138628321)
(1228,3.7376511133877)
(1229,3.7376507819095)
(1230,3.73765036837937)
(1231,3.73764977120418)
(1232,3.73764949080883)
(1233,3.73764926638424)
(1234,3.73764882012534)
(1235,3.73764845540375)
(1236,3.73764819680104)
(1237,3.73764786190632)
(1238,3.73764741058131)
(1239,3.73764695812387)
(1240,3.73764668611141)
(1241,3.73764644131247)
(1242,3.73764572222343)
(1243,3.73764534939324)
(1244,3.73764480703855)
(1245,3.73764440211104)
(1246,3.7376437866241)
(1247,3.73764348988926)
(1248,3.73764320165607)
(1249,3.73764289850436)
(1250,3.737642472943)
(1251,3.73764180007472)
(1252,3.7376412096376)
(1253,3.73764075599067)
(1254,3.73764041359348)
(1255,3.73764018816783)
(1256,3.73763966251974)
(1257,3.73763908824867)
(1258,3.73763877938582)
(1259,3.73763837794714)
(1260,3.73763789855471)
(1261,3.73763754906325)
(1262,3.73763730365613)
(1263,3.73763685658245)
(1264,3.73763643312884)
(1265,3.73763614413284)
(1266,3.7376359232286)
(1267,3.7376353717435)
(1268,3.73763477923102)
(1269,3.73763443376275)
(1270,3.73763382788124)
(1271,3.73763337114625)
(1272,3.73763285506992)
(1273,3.73763250102621)
(1274,3.73763209665208)
(1275,3.7376318347477)
(1276,3.73763155821631)
(1277,3.73763119144495)
(1278,3.7376308504812)
(1279,3.73763047535611)
(1280,3.7376302488955)
(1281,3.73763005418233)
(1282,3.73762973302334)
(1283,3.73762942843885)
(1284,3.73762893171723)
(1285,3.73762855153197)
(1286,3.73762808575924)
(1287,3.73762780320721)
(1288,3.73762744753946)
(1289,3.73762710459974)
(1290,3.73762683573962)
(1291,3.73762657828504)
(1292,3.73762642643795)
(1293,3.73762617329268)
(1294,3.73762601050756)
(1295,3.73762588957408)
(1296,3.73762562122054)
(1297,3.73762523073697)
(1298,3.73762508665577)
(1299,3.73762489891204)
(1300,3.7376230974625)
(1301,3.73762267012478)
(1302,3.73762224143321)
(1303,3.73762149729731)
(1304,3.73762114889393)
(1305,3.7376206416011)
(1306,3.73762044655844)
(1307,3.73762023830673)
(1308,3.73761970486658)
(1309,3.7376194455041)
(1310,3.73761897529248)
(1311,3.73761864373451)
(1312,3.73761842198267)
(1313,3.73761827908011)
(1314,3.73761812939962)
(1315,3.73761796319856)
(1316,3.73761775045959)
(1317,3.73761747039154)
(1318,3.73761731116085)
(1319,3.73761701036397)
(1320,3.73761684238523)
(1321,3.73761671439021)
(1322,3.73761654255161)
(1323,3.73761635447267)
(1324,3.73761590614413)
(1325,3.73761566735629)
(1326,3.7376154400938)
(1327,3.73761529731819)
(1328,3.73761492989069)
(1329,3.73761468866737)
(1330,3.73761440300676)
(1331,3.73761424104455)
(1332,3.73761408614492)
(1333,3.73761390943591)
(1334,3.73761362922652)
(1335,3.73761342484814)
(1336,3.73761324022176)
(1337,3.73761312459037)
(1338,3.73761289900038)
(1339,3.73761271531347)
(1340,3.7376123032293)
(1341,3.73761209866479)
(1342,3.73761192263487)
(1343,3.73761170424469)
(1344,3.73761150600832)
(1345,3.73761131762441)
(1346,3.73761119345184)
(1347,3.73761108882423)
(1348,3.73761097487601)
(1349,3.73761084588636)
(1350,3.73761072363959)
(1351,3.73761064215836)
(1352,3.73761053209004)
(1353,3.73761044130857)
(1354,3.73761029183933)
(1355,3.73761015247928)
(1356,3.73760996685499)
(1357,3.73760988775832)
(1358,3.73760978738946)
(1359,3.7376090020316)
(1360,3.7376089035519)
(1361,3.73760881206537)
(1362,3.73760861602219)
(1363,3.73760842656771)
(1364,3.73760832177562)
(1365,3.73760818150878)
(1366,3.73760809890135)
(1367,3.73760797151546)
(1368,3.73760787161063)
(1369,3.73760756961736)
(1370,3.73760747116711)
(1371,3.73760736011604)
(1372,3.73760723202149)
(1373,3.73760712947204)
(1374,3.73760701600486)
(1375,3.73760687731659)
(1376,3.73760679008259)
(1377,3.73760670106301)
(1378,3.73760663864601)
(1379,3.7376065350028)
(1380,3.73760642026817)
(1381,3.73760632415419)
(1382,3.73760623833773)
(1383,3.73760618753237)
(1384,3.73760610578057)
(1385,3.73760604648956)
(1386,3.73760597689513)
(1387,3.73760586593274)
(1388,3.73760575484324)
(1389,3.73760546334045)
(1390,3.73760510095436)
(1391,3.737604956683)
(1392,3.73760490445019)
(1393,3.73760468410573)
(1394,3.73760444259203)
(1395,3.73760430410201)
(1396,3.73760409273969)
(1397,3.73760397099405)
(1398,3.73760382561482)
(1399,3.73760373220187)
(1400,3.73760364059813)
(1401,3.73760355782486)
(1402,3.73760348082942)
(1403,3.73760340073912)
(1404,3.73760332667934)
(1405,3.73760324347233)
(1406,3.73760318147964)
(1407,3.73760307996749)
(1408,3.7376029898135)
(1409,3.73760278335389)
(1410,3.73760270823199)
(1411,3.73760264710096)
(1412,3.73760242922555)
(1413,3.73760193988499)
(1414,3.73760143838954)
(1415,3.73760109213916)
(1416,3.73760098132394)
(1417,3.7376005070204)
(1418,3.73760038565941)
(1419,3.73760025857145)
(1420,3.73760010073035)
(1421,3.73759999377192)
(1422,3.73759984564174)
(1423,3.73759977410477)
(1424,3.73759969744466)
(1425,3.7375995128137)
(1426,3.73759942087224)
(1427,3.73759926648255)
(1428,3.73759900859604)
(1429,3.73759894122858)
(1430,3.73759889091598)
(1431,3.73759878592945)
(1432,3.73759870726108)
(1433,3.73759860482734)
(1434,3.7375985560469)
(1435,3.73759848994877)
(1436,3.73759836673548)
(1437,3.73759830340948)
(1438,3.73759826165397)
(1439,3.73759821807697)
(1440,3.73759812914216)
(1441,3.73759806283282)
(1442,3.73759802330871)
(1443,3.73759799204713)
(1444,3.73759790880162)
(1445,3.73759785022702)
(1446,3.7375975859094)
(1447,3.73759756369907)
(1448,3.73759751055027)
(1449,3.73759749673014)
(1450,3.73759742949714)
(1451,3.73759731236853)
(1452,3.73759727864989)
(1453,3.73759721879489)
(1454,3.73759715662844)
(1455,3.7375971194844)
(1456,3.73759709491096)
(1457,3.73759706867463)
(1458,3.7375970320241)
(1459,3.73759698421843)
(1460,3.73759696493256)
(1461,3.73759694817078)
(1462,3.73759692153828)
(1463,3.73759688118051)
(1464,3.73759686435211)
(1465,3.73759682020134)
(1466,3.73759674792578)
(1467,3.73759671372138)
(1468,3.73759670557411)
(1469,3.73759666530939)
(1470,3.73759658906249)
(1471,3.73759653305572)
(1472,3.73759648627582)
(1473,3.73759645098238)
(1474,3.73759641837893)
(1475,3.73759639873333)
(1476,3.73759636800384)
(1477,3.73759634768158)
(1478,3.73759632900134)
(1479,3.73759631346046)
(1480,3.73759629297124)
(1481,3.73759627519922)
(1482,3.73759626390486)
(1483,3.73759624653445)
(1484,3.73759623070539)
(1485,3.73759620949157)
(1486,3.73759618572849)
(1487,3.73759615275141)
(1488,3.7375961137877)
(1489,3.73759608989624)
(1490,3.7375960715564)
(1491,3.73759600599166)
(1492,3.73759596763902)
(1493,3.7375959020056)
(1494,3.73759587258152)
(1495,3.73759581548826)
(1496,3.73759578491382)
(1497,3.73759575465653)
(1498,3.73759573273887)
(1499,3.73759571318224)
(1500,3.73759569690808)
(1501,3.73759567680252)
(1502,3.73759564875242)
(1503,3.73759559550243)
(1504,3.737595575628)
(1505,3.73759554620513)
(1506,3.73759548423987)
(1507,3.73759545080701)
(1508,3.73759543956058)
(1509,3.73759535200989)
(1510,3.73759531904693)
(1511,3.73759516601144)
(1512,3.73759501692505)
(1513,3.73759497910142)
(1514,3.73759494664961)
(1515,3.73759488374651)
(1516,3.73759485856696)
(1517,3.73759483267793)
(1518,3.73759478656688)
(1519,3.73759474630067)
(1520,3.73759471942792)
(1521,3.73759469760238)
(1522,3.73759467574851)
(1523,3.73759465371966)
(1524,3.73759463090968)
(1525,3.73759461416489)
(1526,3.73759458981272)
(1527,3.73759457100742)
(1528,3.73759455188972)
(1529,3.73759452211803)
(1530,3.73759448909325)
(1531,3.73759440265633)
(1532,3.73759435980884)
(1533,3.73759431673363)
(1534,3.73759429823854)
(1535,3.73759424922084)
(1536,3.73759421141819)
(1537,3.73759414264468)
(1538,3.73759411267539)
(1539,3.73759407811473)
(1540,3.73759405336316)
(1541,3.73759401424284)
(1542,3.73759398973353)
(1543,3.73759396835957)
(1544,3.73759391325268)
(1545,3.73759388295736)
(1546,3.73759385693513)
(1547,3.73759383687114)
(1548,3.73759381003253)
(1549,3.73759378730555)
(1550,3.7375937564466)
(1551,3.73759370104203)
(1552,3.73759367512416)
(1553,3.73759365622307)
(1554,3.73759362562735)
(1555,3.73759358626582)
(1556,3.73759347653489)
(1557,3.73759345024392)
(1558,3.73759342511212)
(1559,3.73759341132892)
(1560,3.73759335898225)
(1561,3.73759313473795)
(1562,3.73759305764427)
(1563,3.73759293198748)
(1564,3.73759289553699)
(1565,3.7375928579892)
(1566,3.73759282784336)
(1567,3.73759280045312)
(1568,3.73759276790708)
(1569,3.73759274988997)
(1570,3.73759273290781)
(1571,3.73759268669778)
(1572,3.73759266738146)
(1573,3.73759263290324)
(1574,3.73759255886586)
(1575,3.73759251383096)
(1576,3.7375925070418)
(1577,3.73759242893098)
(1578,3.7375923742678)
(1579,3.73759233939731)
(1580,3.73759229802056)
(1581,3.73759223710942)
(1582,3.73759218883031)
(1583,3.73759217046377)
(1584,3.73759215824898)
(1585,3.73759214050205)
(1586,3.73759212422926)
(1587,3.73759210838957)
(1588,3.7375920977849)
(1589,3.7375920806783)
(1590,3.73759206509513)
(1591,3.7375920512522)
(1592,3.7375920427595)
(1593,3.73759203566467)
(1594,3.73759201164258)
(1595,3.73759200326938)
(1596,3.73759199855686)
(1597,3.73759198496145)
(1598,3.73759193719825)
(1599,3.73759191104112)
(1600,3.73759188761455)
(1601,3.73759187757655)
(1602,3.73759184500687)
(1603,3.73759182561239)
(1604,3.73759181019494)
(1605,3.73759180430032)
(1606,3.73759177453126)
(1607,3.73759175400277)
(1608,3.73759174229279)
(1609,3.73759173315729)
(1610,3.73759172253214)
(1611,3.73759171082768)
(1612,3.73759169446135)
(1613,3.73759168454076)
(1614,3.73759165435977)
(1615,3.73759164035126)
(1616,3.73759159928345)
(1617,3.73759159291005)
(1618,3.73759158465185)
(1619,3.73759158192753)
(1620,3.73759156969715)
(1621,3.73759151600569)
(1622,3.73759148988752)
(1623,3.7375914706473)
(1624,3.73759145927664)
(1625,3.73759144218356)
(1626,3.73759143116457)
(1627,3.73759142435886)
(1628,3.73759141310619)
(1629,3.73759140664004)
(1630,3.73759140123351)
(1631,3.73759139867231)
(1632,3.73759139030815)
(1633,3.73759138701396)
(1634,3.73759138347821)
(1635,3.73759133868459)
(1636,3.73759133254748)
(1637,3.73759132614577)
(1638,3.73759132348959)
(1639,3.73759131674587)
(1640,3.73759131416952)
(1641,3.73759131154659)
(1642,3.73759130977906)
(1643,3.73759130692713)
(1644,3.73759130446745)
(1645,3.73759130232601)
(1646,3.73759130043594)
(1647,3.73759129731258)
(1648,3.73759129422323)
(1649,3.73759129093343)
(1650,3.73759128750128)
(1651,3.73759128103571)
(1652,3.73759127148391)
(1653,3.73759126692418)
(1654,3.73759126367369)
(1655,3.73759123957256)
(1656,3.73759121715229)
(1657,3.73759121362837)
(1658,3.73759120500385)
(1659,3.73759118534987)
(1660,3.73759117905565)
(1661,3.7375911715676)
(1662,3.73759114205985)
(1663,3.7375911296927)
(1664,3.73759112242247)
(1665,3.73759109324951)
(1666,3.7375910839578)
(1667,3.73759106826165)
(1668,3.73759104985113)
(1669,3.73759103751238)
(1670,3.73759102897253)
(1671,3.73759101654705)
(1672,3.73759100585387)
(1673,3.73759098455513)
(1674,3.73759097121983)
(1675,3.73759093213954)
(1676,3.73759091525101)
(1677,3.73759089829387)
(1678,3.73759086096342)
(1679,3.73759082893052)
(1680,3.737590796788)
(1681,3.73759077442082)
(1682,3.73759075586395)
(1683,3.73759072643293)
(1684,3.73759060919327)
(1685,3.73759053295097)
(1686,3.73759049330429)
(1687,3.73759045862152)
(1688,3.73759041816716)
(1689,3.73759038411041)
(1690,3.73759031493546)
(1691,3.73759026531969)
(1692,3.73759023236924)
(1693,3.73759021148846)
(1694,3.73759018624055)
(1695,3.73759016704711)
(1696,3.73759015163669)
(1697,3.73759013078604)
(1698,3.73759010222169)
(1699,3.73759004772387)
(1700,3.73759002866821)
(1701,3.73759001495885)
(1702,3.73758998552235)
(1703,3.73758990908409)
(1704,3.73758981090818)
(1705,3.7375897391046)
(1706,3.73758971982188)
(1707,3.73758958869862)
(1708,3.73758956483426)
(1709,3.73758952139986)
(1710,3.73758949144691)
(1711,3.7375894175407)
(1712,3.73758937572804)
(1713,3.73758930651014)
(1714,3.73758927962953)
(1715,3.73758924224832)
(1716,3.73758917367011)
(1717,3.73758914582746)
(1718,3.73758913643835)
(1719,3.73758907126005)
(1720,3.73758905054521)
(1721,3.73758900816774)
(1722,3.73758898061566)
(1723,3.73758896051808)
(1724,3.73758894322232)
(1725,3.73758891995705)
(1726,3.7375888949516)
(1727,3.7375888645591)
(1728,3.73758884634554)
(1729,3.73758881829787)
(1730,3.73758880539018)
(1731,3.73758879586456)
(1732,3.73758872655574)
(1733,3.73758870953495)
(1734,3.73758869697307)
(1735,3.73758869116605)
(1736,3.73758863969246)
(1737,3.73758860853091)
(1738,3.73758858457416)
(1739,3.73758856546749)
(1740,3.737588538409)
(1741,3.73758850959037)
(1742,3.7375884882394)
(1743,3.73758846185977)
(1744,3.73758843680878)
(1745,3.73758840779896)
(1746,3.73758839137689)
(1747,3.73758838196593)
(1748,3.73758833144478)
(1749,3.73758831282487)
(1750,3.7375882950344)
(1751,3.73758828347222)
(1752,3.73758826764018)
(1753,3.73758825533004)
(1754,3.73758823443736)
(1755,3.73758822491147)
(1756,3.73758821606131)
(1757,3.73758819846738)
(1758,3.73758818643535)
(1759,3.73758816046739)
(1760,3.73758815304434)
(1761,3.73758814697514)
(1762,3.73758813482113)
(1763,3.73758810359343)
(1764,3.73758806042036)
(1765,3.73758803428063)
(1766,3.73758802804734)
(1767,3.73758799420251)
(1768,3.73758798110165)
(1769,3.73758795410743)
(1770,3.73758794263395)
(1771,3.73758793133417)
(1772,3.73758791458983)
(1773,3.73758790058685)
(1774,3.73758788234625)
(1775,3.73758787314109)
(1776,3.73758786330207)
(1777,3.73758785015541)
(1778,3.73758784004478)
(1779,3.73758783140052)
(1780,3.73758782532189)
(1781,3.7375878214215)
(1782,3.73758781596135)
(1783,3.73758780832482)
(1784,3.73758780068191)
(1785,3.73758779655553)
(1786,3.73758778946524)
(1787,3.73758777600372)
(1788,3.73758776880541)
(1789,3.73758776533992)
(1790,3.73758773800892)
(1791,3.73758772826122)
(1792,3.73758770965911)
(1793,3.73758769304365)
(1794,3.73758768226737)
(1795,3.73758767278667)
(1796,3.73758766509388)
(1797,3.73758765588906)
(1798,3.73758764419383)
(1799,3.73758762851168)
(1800,3.7375876186642)
(1801,3.73758761245765)
(1802,3.73758758441205)
(1803,3.73758756867851)
(1804,3.73758751219017)
(1805,3.73758750232962)
(1806,3.73758748829042)
(1807,3.73758747973202)
(1808,3.7375874585515)
(1809,3.7375873153233)
(1810,3.73758726747758)
(1811,3.73758723671357)
(1812,3.73758721933354)
(1813,3.73758719110762)
(1814,3.73758716037321)
(1815,3.73758714541441)
(1816,3.73758713521314)
(1817,3.73758711254284)
(1818,3.73758709908092)
(1819,3.73758705395136)
(1820,3.73758703909125)
(1821,3.73758702435067)

};
\addplot [thick, blue]
coordinates {
(0,23930.000884189)
(1,7180.45388604586)
(2,3617.86527996726)
(3,2206.80793392946)
(4,1491.00767628939)
(5,1073.99586354884)
(6,808.278854547473)
(7,628.014922400163)
(8,499.960644278589)
(9,405.752764676303)
(10,334.52634655712)
(11,279.477813847677)
(12,236.164491398145)
(13,201.577684383878)
(14,173.611856832551)
(15,150.754899712087)
(16,131.896367104543)
(17,116.20926011348)
(18,103.06516983272)
(19,91.9789779656275)
(20,82.5721972749523)
(21,74.5480919899652)
(22,67.6703468931341)
(23,61.7492516151206)
(24,56.6308080266971)
(25,52.1895417181108)
(26,48.3230738518772)
(27,44.9473811889776)
(28,41.9908958545994)
(29,39.3943307887263)
(30,37.1076648133911)
(31,35.0895582332656)
(32,33.3047098499916)
(33,31.7230458561848)
(34,30.3185369243055)
(35,29.0691392036425)
(36,27.9557168948839)
(37,26.9619473758925)
(38,26.0735819006833)
(39,25.2785051309949)
(40,24.5658953302441)
(41,23.9261996677162)
(42,23.3512128923731)
(43,22.8338555363931)
(44,22.3679803523897)
(45,21.9478508869475)
(46,21.5686369463874)
(47,21.2260325999356)
(48,20.9161983824383)
(49,20.6357394037178)
(50,20.3816052485759)
(51,20.1511846245288)
(52,19.942073913322)
(53,19.7522963849272)
(54,19.5798157329886)
(55,19.4229051369124)
(56,19.2800427060627)
(57,19.1499227239291)
(58,19.0313113044641)
(59,18.9230788312623)
(60,18.8242439624631)
(61,18.7339264319528)
(62,18.6513238709167)
(63,18.5757271356125)
(64,18.5064950547755)
(65,18.4430397931058)
(66,18.3848534199294)
(67,18.3314621000904)
(68,18.282410130483)
(69,18.2373078847479)
(70,18.195837809721)
(71,18.1576476003302)
(72,18.1224457540708)
(73,18.0899698163812)
(74,18.0599809133384)
(75,18.032269842075)
(76,18.0066439340179)
(77,17.9829130684296)
(78,17.9609160188493)
(79,17.9405020453336)
(80,17.9215358365279)
(81,17.9038985280102)
(82,17.8874801575507)
(83,17.8721768211419)
(84,17.8578894936326)
(85,17.8445328788038)
(86,17.83202928813)
(87,17.8203078087826)
(88,17.8093094635342)
(89,17.7989701384791)
(90,17.7892345518416)
(91,17.7800530702916)
(92,17.7713803468455)
(93,17.7631748927227)
(94,17.755398724103)
(95,17.7480170503084)
(96,17.7409979934438)
(97,17.7343123346806)
(98,17.7279332839925)
(99,17.7218362708545)
(100,17.7160071042362)

};
\addplot [thick, green!50.0!black]
coordinates {
(0,23930.000884189)
(1,12083.1529253535)
(2,1962.70076992746)
(3,950.485602345173)
(4,331.959633328347)
(5,243.914762566925)
(6,134.329835058351)
(7,101.908974098636)
(8,71.430490269532)
(9,53.8008057292489)
(10,49.2070909809805)
(11,47.2377989642264)
(12,45.8394992598531)
(13,44.9246042975355)
(14,44.7440629754329)
(15,44.39329787842)
(16,44.3236229723654)
(17,44.2184239820041)
(18,44.1931882049634)
(19,44.1267629597911)
(20,44.1051433432534)
(21,44.0736620641302)
(22,44.0227863049802)
(23,43.9604053846835)
(24,43.8383240409788)
(25,43.7012466950384)
(26,43.4900906898139)
(27,43.1528664627698)
(28,42.5004315629993)
(29,41.8013543791512)
(30,40.8296742339329)
(31,39.900023272227)
(32,38.5823534699433)
(33,37.4156430458995)
(34,36.2719668390337)
(35,35.0776601950381)
(36,34.1837687644875)
(37,33.2698645775237)
(38,32.1962015032741)
(39,31.3082549503913)
(40,30.4262826422634)
(41,29.5345560514281)
(42,28.6995811800974)
(43,27.8507440638526)
(44,26.8903832876716)
(45,26.0249848856949)
(46,25.0716863775733)
(47,24.1555764755733)
(48,23.4070450974266)
(49,22.7049953977644)
(50,22.0172886765464)
(51,21.3371046021488)
(52,20.6002987471974)
(53,19.9394587462096)
(54,19.3137752887323)
(55,18.6682287336718)
(56,18.1127667913341)
(57,17.5853401003318)
(58,16.9831567654981)
(59,16.4442553456098)
(60,15.9565899295854)
(61,15.4524799469826)
(62,15.0154266084085)
(63,14.6158710219892)
(64,14.1749219112577)
(65,13.8168371772112)
(66,13.4635649950696)
(67,13.0778586597113)
(68,12.766579104139)
(69,12.4211022836258)
(70,12.0819160127479)
(71,11.7910516256896)
(72,11.5003137939156)
(73,11.234474233497)
(74,10.9852905269443)
(75,10.7240520078585)
(76,10.4627083449499)
(77,10.2286926962082)
(78,9.95946000368242)
(79,9.74002365928325)
(80,9.51656034513117)
(81,9.29145988899816)
(82,9.07267351528514)
(83,8.88697268772216)
(84,8.66699360507882)
(85,8.48608469643221)
(86,8.31017432250414)
(87,8.12849250541886)
(88,7.95268498774693)
(89,7.80156034591783)
(90,7.63432584968156)
(91,7.4884815251341)
(92,7.34172281544927)
(93,7.2094978287168)
(94,7.08523071948241)
(95,6.97027976094049)
(96,6.85746063617941)
(97,6.74882343897467)
(98,6.6395369792259)
(99,6.54752507330449)
(100,6.451547775383)
(101,6.36162519097851)
(102,6.27636659871515)
(103,6.19576519855579)
(104,6.1105722950634)
(105,6.02718102167213)
(106,5.95082535982036)
(107,5.87578582284964)
(108,5.81233595151331)
(109,5.73816283625261)
(110,5.67001782292722)
(111,5.59877510876061)
(112,5.53334659609807)
(113,5.47110793085513)
(114,5.40991942506864)
(115,5.36059759085225)
(116,5.30003871138622)
(117,5.25644195605928)
(118,5.2079330088986)
(119,5.16682679077942)
(120,5.12543393447383)
(121,5.08741167787886)
(122,5.04874894827296)
(123,5.01259404523217)
(124,4.97309259872131)
(125,4.93509187871559)
(126,4.89996412221353)
(127,4.86620424754931)
(128,4.83259517430058)
(129,4.79999147089031)
(130,4.76493049905084)
(131,4.73325694446595)
(132,4.70416695570773)
(133,4.67888476619529)
(134,4.65262739639857)
(135,4.62762957717384)
(136,4.60369921903072)
(137,4.57990557443312)
(138,4.55729925905472)
(139,4.53696012173993)
(140,4.51517453524093)
(141,4.49495399856832)
(142,4.47542802021928)
(143,4.45406482650698)
(144,4.43656339391869)
(145,4.4160779300751)
(146,4.40306779014329)
(147,4.38606923868873)
(148,4.37190599538975)
(149,4.3577969376911)
(150,4.34318842794173)
(151,4.32919007359116)
(152,4.31158257764717)
(153,4.29677898650613)
(154,4.28305080552483)
(155,4.267086204204)
(156,4.2556660992037)
(157,4.23959397046955)
(158,4.22830194655125)
(159,4.21514694220386)
(160,4.205095004479)
(161,4.19570757464146)
(162,4.18438526562959)
(163,4.17710246770172)
(164,4.16764634202139)
(165,4.15979207046328)
(166,4.15082356546804)
(167,4.1413433964435)
(168,4.12834874500854)
(169,4.11980319580987)
(170,4.10692696875007)
(171,4.10011473718714)
(172,4.09123078620307)
(173,4.08648205789493)
(174,4.0787785079874)
(175,4.07174538454882)
(176,4.06528955168316)
(177,4.05784389039987)
(178,4.05216401723596)
(179,4.0450031924621)
(180,4.03969141173713)
(181,4.03158734139422)
(182,4.02736646948498)
(183,4.02067225174152)
(184,4.01607415543408)
(185,4.01021462980189)
(186,4.00613493575736)
(187,3.99983563711056)
(188,3.99648862233057)
(189,3.99125870742269)
(190,3.98846829193574)
(191,3.9831902487313)
(192,3.98007499293121)
(193,3.97524163934253)
(194,3.97215460976433)
(195,3.96632082396578)
(196,3.96254319211348)
(197,3.95708910777948)
(198,3.95383850180867)
(199,3.94852121439348)
(200,3.94568940290271)
(201,3.94072141291591)
(202,3.93782105410923)
(203,3.93305155506064)
(204,3.93019398139779)
(205,3.92532870600185)
(206,3.92209170875403)
(207,3.91810749726888)
(208,3.91534363714)
(209,3.91186221718249)
(210,3.90928785970875)
(211,3.90606044995817)
(212,3.90349266209369)
(213,3.89958691920946)
(214,3.897287214864)
(215,3.8936821450916)
(216,3.89139327046655)
(217,3.88793774047786)
(218,3.88557234615864)
(219,3.88228682452595)
(220,3.88051617709015)
(221,3.87729796257439)
(222,3.87513932736856)
(223,3.87216538408708)
(224,3.87028102692261)
(225,3.86773271296983)
(226,3.8662517928345)
(227,3.86408512950891)
(228,3.86259043297563)
(229,3.86035480014273)
(230,3.85884153265201)
(231,3.85705879546716)
(232,3.85539309545359)
(233,3.85410561380124)
(234,3.85242128398972)
(235,3.85102128013812)
(236,3.84945327587287)
(237,3.84781293402378)
(238,3.84604008183371)
(239,3.84417200140538)
(240,3.84274745653806)
(241,3.84098541886405)
(242,3.83971776306798)
(243,3.83798034973253)
(244,3.83683190988865)
(245,3.83548293350208)
(246,3.83424116842435)
(247,3.83283540727062)
(248,3.8313033218924)
(249,3.82981663297735)
(250,3.82848929462528)
(251,3.8267681263994)
(252,3.82556631007534)
(253,3.8237236519563)
(254,3.82266083539909)
(255,3.82121486467379)
(256,3.82027029191084)
(257,3.81914306693968)
(258,3.81788632471194)
(259,3.81681665579276)
(260,3.815442034354)
(261,3.81415800671505)
(262,3.81297447001371)
(263,3.81152340702821)
(264,3.81045569222702)
(265,3.80941642619398)
(266,3.80846482708664)
(267,3.80743060710025)
(268,3.80619376358734)
(269,3.80522889452713)
(270,3.80401851256108)
(271,3.80299889538528)
(272,3.80205030908258)
(273,3.80106889898641)
(274,3.79988412022831)
(275,3.7991917822155)
(276,3.79807124376999)
(277,3.797482715769)
(278,3.7965630211434)
(279,3.79595838279024)
(280,3.79510609114931)
(281,3.79446580203799)
(282,3.79370736410598)
(283,3.79306436028809)
(284,3.79237663383407)
(285,3.79176445214526)
(286,3.79109035607477)
(287,3.79035981049862)
(288,3.7898544928744)
(289,3.78909664395785)
(290,3.78852891958106)
(291,3.78780672972926)
(292,3.78705331214125)
(293,3.78632831068458)
(294,3.78561088257541)
(295,3.78494126319544)
(296,3.78424645159915)
(297,3.78362091545339)
(298,3.78279288500698)
(299,3.78225339809526)
(300,3.78155230857642)
(301,3.78100235288025)
(302,3.78031075503927)
(303,3.77981799763149)
(304,3.77917037382963)
(305,3.77870749903801)
(306,3.77819085472531)
(307,3.77761639759165)
(308,3.77715341323193)
(309,3.77664655414066)
(310,3.77623097367384)
(311,3.77576984061671)
(312,3.77536295173602)
(313,3.77503195579037)
(314,3.77465982150755)
(315,3.77434166336993)
(316,3.77396276713725)
(317,3.77363004159907)
(318,3.77316841264676)
(319,3.77276421897502)
(320,3.77225566919488)
(321,3.77189031199103)
(322,3.7714721403252)
(323,3.77109490607283)
(324,3.77077672848363)
(325,3.77040763452058)
(326,3.77014261566575)
(327,3.76981112057641)
(328,3.76950132945398)
(329,3.76917320470531)
(330,3.76883523362868)
(331,3.76848952063809)
(332,3.76815760320699)
(333,3.76782985213799)
(334,3.76754482814153)
(335,3.76729175462336)
(336,3.76702112772215)
(337,3.76676071816487)
(338,3.76650710166178)
(339,3.76624529610397)
(340,3.7660012369687)
(341,3.76576871156933)
(342,3.76548216567509)
(343,3.76525100190972)
(344,3.76494754504821)
(345,3.76469876108863)
(346,3.76442697240637)
(347,3.76416546542455)
(348,3.76393475400801)
(349,3.76363455908212)
(350,3.76337640316017)
(351,3.76311301911484)
(352,3.76284771453187)
(353,3.76253703911119)
(354,3.76232167142751)
(355,3.76202573480877)
(356,3.76185425270663)
(357,3.76156246706336)
(358,3.76138344332866)
(359,3.76117289703059)
(360,3.76100270837172)
(361,3.76082446550187)
(362,3.7606289573713)
(363,3.76044311543043)
(364,3.76024629428892)
(365,3.76006954475379)
(366,3.75985640443088)
(367,3.75969235902624)
(368,3.75951702274354)
(369,3.75931657626994)
(370,3.75910121730723)
(371,3.75891229381617)
(372,3.75867411566804)
(373,3.75849673663124)
(374,3.75820699197885)
(375,3.75804020440156)
(376,3.75779450624678)
(377,3.75761621409044)
(378,3.75742597013973)
(379,3.75727316308266)
(380,3.75707887403196)
(381,3.75691804333044)
(382,3.75671995029196)
(383,3.75657361317903)
(384,3.75636936444537)
(385,3.75623420136118)
(386,3.75605211770352)
(387,3.75591023229299)
(388,3.75572308588805)
(389,3.75559401401372)
(390,3.75539650343974)
(391,3.75527251214585)
(392,3.75510164488203)
(393,3.75496109680896)
(394,3.75481262853061)
(395,3.75469609001998)
(396,3.75454271565955)
(397,3.75440771834154)
(398,3.75426445318479)
(399,3.75411262043161)
(400,3.75399171515408)
(401,3.75384700299386)
(402,3.75373601510596)
(403,3.75358952992677)
(404,3.75346851370497)
(405,3.75331368838498)
(406,3.75321270157323)
(407,3.75304360830039)
(408,3.75293439432026)
(409,3.75276541695022)
(410,3.75262054753382)
(411,3.75247200037655)
(412,3.7523475081682)
(413,3.75219140568697)
(414,3.75206671474024)
(415,3.75189307979836)
(416,3.75173763675237)
(417,3.75156268024528)
(418,3.75142599690311)
(419,3.75125107517422)
(420,3.75111450674077)
(421,3.7509454269434)
(422,3.75081900849697)
(423,3.75064999403041)
(424,3.75052161893989)
(425,3.75036528789738)
(426,3.75026399910255)
(427,3.75010917698355)
(428,3.75001596172723)
(429,3.7498732375621)
(430,3.74977431046982)
(431,3.74964562735367)
(432,3.74954942303358)
(433,3.74944652403868)
(434,3.7493681752848)
(435,3.74927872296471)
(436,3.74919218837585)
(437,3.74911693308352)
(438,3.74904535603257)
(439,3.74898662304563)
(440,3.74891812214901)
(441,3.74886006547367)
(442,3.74878864997678)
(443,3.74873493000449)
(444,3.74866742227346)
(445,3.74860335566448)
(446,3.74853886909196)
(447,3.7484689239028)
(448,3.74839482936899)
(449,3.7483291956542)
(450,3.74825233183658)
(451,3.74818854703671)
(452,3.74811401122678)
(453,3.7480493651035)
(454,3.74797064112921)
(455,3.74790148235455)
(456,3.74782462089375)
(457,3.74773651887886)
(458,3.74766042018675)
(459,3.74757585211129)
(460,3.74750236376181)
(461,3.74743283260543)
(462,3.74736352734666)
(463,3.74729553947663)
(464,3.74722657349309)
(465,3.74717382868019)
(466,3.74710339329863)
(467,3.74704646640288)
(468,3.74697943582619)
(469,3.74691128760395)
(470,3.74685245681109)
(471,3.7467801417168)
(472,3.74673345968546)
(473,3.74665729852526)
(474,3.74659552995826)
(475,3.74652509912179)
(476,3.74645390176811)
(477,3.74637731978863)
(478,3.74631515250338)
(479,3.74622957037861)
(480,3.74616387803336)
(481,3.74608521368325)
(482,3.74603074191055)
(483,3.74595306429884)
(484,3.74589855954142)
(485,3.74583462599904)
(486,3.74577077734902)
(487,3.74571831393103)
(488,3.74565593313321)
(489,3.74560545139983)
(490,3.74554468118862)
(491,3.74548637616047)
(492,3.74542800526426)
(493,3.74536301874227)
(494,3.74530919324732)
(495,3.74524474401653)
(496,3.74519043103749)
(497,3.74513261719584)
(498,3.7450748078865)
(499,3.74501474774243)
(500,3.74495562664613)
(501,3.74489141848624)
(502,3.74482564836354)
(503,3.74476995007867)
(504,3.74470217498241)
(505,3.74466013798536)
(506,3.74459241735894)
(507,3.74454516156133)
(508,3.74449130315165)
(509,3.74443834806183)
(510,3.74438116607466)
(511,3.74432569129946)
(512,3.74427277852877)
(513,3.74421468938001)
(514,3.74416898588479)
(515,3.74411319810954)
(516,3.74407042018594)
(517,3.74401511214301)
(518,3.74396160607964)
(519,3.74390840544191)
(520,3.74384421006221)
(521,3.74378485738189)
(522,3.74371527275934)
(523,3.74364856496327)
(524,3.74359798408825)
(525,3.74352553396694)
(526,3.74347413806783)
(527,3.7434139073384)
(528,3.74336809436373)
(529,3.74331388671594)
(530,3.74326481803457)
(531,3.74321706082349)
(532,3.7431667639514)
(533,3.74311644970281)
(534,3.74306550277262)
(535,3.74301612338473)
(536,3.74296511114281)
(537,3.74292679709068)
(538,3.7428840961028)
(539,3.74284083603382)
(540,3.74279934013523)
(541,3.74275079549761)
(542,3.74271288375513)
(543,3.74265985488221)
(544,3.74262619436029)
(545,3.74257588637796)
(546,3.74253814252726)
(547,3.74249694296591)
(548,3.74246291698524)
(549,3.74242574050466)
(550,3.74239141277331)
(551,3.74235198117102)
(552,3.74231594388783)
(553,3.74228000186351)
(554,3.74225188910381)
(555,3.74222155872168)
(556,3.74219576824704)
(557,3.74216113099798)
(558,3.7421291097801)
(559,3.74209025127021)
(560,3.74205645012881)
(561,3.742021328813)
(562,3.74198671723298)
(563,3.74195160010508)
(564,3.74192422502758)
(565,3.74189243587585)
(566,3.74186892772694)
(567,3.74184104773447)
(568,3.74181717989176)
(569,3.74178758912634)
(570,3.74176170676605)
(571,3.74172994771925)
(572,3.7417043910203)
(573,3.74167522898495)
(574,3.74164707029444)
(575,3.74161954493233)
(576,3.74159400714088)
(577,3.74156332337963)
(578,3.74153752882663)
(579,3.74151144021312)
(580,3.74148733439032)
(581,3.74146244079137)
(582,3.74143818563236)
(583,3.74141818184761)
(584,3.74139000959428)
(585,3.74137234442168)
(586,3.7413478078677)
(587,3.74132927231714)
(588,3.74130673362227)
(589,3.74128907232733)
(590,3.74127362276867)
(591,3.74124914971593)
(592,3.74123446674069)
(593,3.74121632466596)
(594,3.74120328924757)
(595,3.74118909198467)
(596,3.74117680023017)
(597,3.74116386712557)
(598,3.74115020641502)
(599,3.74113609115005)
(600,3.74112102775107)
(601,3.74110749144362)
(602,3.74109314215094)
(603,3.74107917636756)
(604,3.74106530064851)
(605,3.74105188044314)
(606,3.74104058745549)
(607,3.74102722452002)
(608,3.74101564643921)
(609,3.7410008112848)
(610,3.74098842846916)
(611,3.74097421955018)
(612,3.74096139595054)
(613,3.74094836322696)
(614,3.74093659798061)
(615,3.74092290311171)
(616,3.74091219227729)
(617,3.74089814241077)
(618,3.7408861800869)
(619,3.74087294730084)
(620,3.7408597322999)
(621,3.74084563998964)
(622,3.74082985004121)
(623,3.74081335151383)
(624,3.74079551969156)
(625,3.74077958068779)
(626,3.74076175305565)
(627,3.74074628545173)
(628,3.74072490354971)
(629,3.74070646601223)
(630,3.74068413792834)
(631,3.74066539847874)
(632,3.74064288016811)
(633,3.74062455595132)
(634,3.74060225253168)
(635,3.74058485230222)
(636,3.74056720432599)
(637,3.74055080307954)
(638,3.7405355904744)
(639,3.74051831025509)
(640,3.74049967410211)
(641,3.7404805820288)
(642,3.7404614181446)
(643,3.7404429937889)
(644,3.74042496445342)
(645,3.74040890097807)
(646,3.74038926776979)
(647,3.74037152353002)
(648,3.74035252625746)
(649,3.74033603939086)
(650,3.74031681998737)
(651,3.74029990664518)
(652,3.74028027147941)
(653,3.74026014164315)
(654,3.74024156506649)
(655,3.74022117047007)
(656,3.74020356747955)
(657,3.74018450850117)
(658,3.74016633904289)
(659,3.74014958282022)
(660,3.7401301214953)
(661,3.74011029350001)
(662,3.74009019470237)
(663,3.74006904952426)
(664,3.74004884025743)
(665,3.74002541632284)
(666,3.7400060571524)
(667,3.7399824110428)
(668,3.7399631255681)
(669,3.73994258380541)
(670,3.73992532750292)
(671,3.73990245314094)
(672,3.73988269064243)
(673,3.73986097450602)
(674,3.73983834320463)
(675,3.73982346149618)
(676,3.739808806616)
(677,3.7397894434327)
(678,3.73977530314519)
(679,3.73975817222208)
(680,3.73974305035264)
(681,3.73972789694524)
(682,3.73971465170234)
(683,3.73969961708132)
(684,3.73968447763494)
(685,3.73966606560046)
(686,3.73964990507722)
(687,3.73963166453637)
(688,3.73961687748722)
(689,3.73960278539316)
(690,3.73958896406016)
(691,3.73957411917519)
(692,3.73955883151207)
(693,3.73954533116955)
(694,3.73953066789554)
(695,3.73951893758622)
(696,3.73950519910128)
(697,3.73949317816142)
(698,3.73948031930539)
(699,3.7394688100309)
(700,3.73945584122258)
(701,3.73944436754757)
(702,3.73943138711405)
(703,3.73941893881355)
(704,3.73940744492966)
(705,3.73939534553567)
(706,3.73938459506139)
(707,3.73937227444512)
(708,3.73936214675933)
(709,3.73935006882915)
(710,3.73934039984275)
(711,3.7393292378273)
(712,3.73931901661485)
(713,3.73930878956023)
(714,3.73929906521196)
(715,3.7392875530797)
(716,3.739277916638)
(717,3.73926730332421)
(718,3.73925774458292)
(719,3.7392479932643)
(720,3.73923874038437)
(721,3.73922981791569)
(722,3.73922154686658)
(723,3.73921167046033)
(724,3.73920331711842)
(725,3.73919456397858)
(726,3.73918631566387)
(727,3.73917834474922)
(728,3.73917042270363)
(729,3.73916237844795)
(730,3.73915299143573)
(731,3.73914427576656)
(732,3.73913543069423)
(733,3.7391267066131)
(734,3.73911726655116)
(735,3.73910819104071)
(736,3.73909971441933)
(737,3.73909149254162)
(738,3.73908401550042)
(739,3.7390763009519)
(740,3.73906843692166)
(741,3.73905968627323)
(742,3.73905065076069)
(743,3.73904192751378)
(744,3.73903357636314)
(745,3.73902604377139)
(746,3.73901843709885)
(747,3.73900889415562)
(748,3.73900160309667)
(749,3.73899319368035)
(750,3.73898650568641)
(751,3.73897822879846)
(752,3.73897089324449)
(753,3.73896301795268)
(754,3.73895517076489)
(755,3.73894765985481)
(756,3.73894012512977)
(757,3.73893298043012)
(758,3.73892562206563)
(759,3.73891939171553)
(760,3.73891149068061)
(761,3.73890577199222)
(762,3.73889911737097)
(763,3.73889281153022)
(764,3.73888634911453)
(765,3.73887975607411)
(766,3.73887342211177)
(767,3.73886617331568)
(768,3.73885876920499)
(769,3.73885000520325)
(770,3.73884207976913)
(771,3.7388343769273)
(772,3.7388274048653)
(773,3.73882060786759)
(774,3.73881355154971)
(775,3.73880633393491)
(776,3.73879796483344)
(777,3.73879031328148)
(778,3.73878204196246)
(779,3.73877368954621)
(780,3.7387640563381)
(781,3.73875435569102)
(782,3.73874441370142)
(783,3.73873577632889)
(784,3.73872607226107)
(785,3.73871813795245)
(786,3.73870868666259)
(787,3.73870018513551)
(788,3.73869178497775)
(789,3.73868341674856)
(790,3.73867597355206)
(791,3.73866757594054)
(792,3.73866073760676)
(793,3.73865220868223)
(794,3.73864498636212)
(795,3.73863867871789)
(796,3.73863129113675)
(797,3.7386256295367)
(798,3.73861884723437)
(799,3.73861248452653)
(800,3.73860612432279)
(801,3.73860026767612)
(802,3.73859455441694)
(803,3.73858820576177)
(804,3.73858127593893)
(805,3.73857346277667)
(806,3.73856626769453)
(807,3.73855891495757)
(808,3.7385528572699)
(809,3.73854581476239)
(810,3.73853905091437)
(811,3.73853219868304)
(812,3.73852646389334)
(813,3.73852015640883)
(814,3.738514909154)
(815,3.73850853000471)
(816,3.73850284708773)
(817,3.73849721927389)
(818,3.73849239098663)
(819,3.73848737215917)
(820,3.73848264838812)
(821,3.7384774604575)
(822,3.73847211337382)
(823,3.73846617751523)
(824,3.73846124619406)
(825,3.73845554427388)
(826,3.73845028752177)
(827,3.73844497493247)
(828,3.73843973399836)
(829,3.73843455112398)
(830,3.73842932788865)
(831,3.7384248373687)
(832,3.73841926490839)
(833,3.73841382887902)
(834,3.73840825490896)
(835,3.73840316697943)
(836,3.73839725571907)
(837,3.73839259373779)
(838,3.73838684888716)
(839,3.73838157655308)
(840,3.73837646790498)
(841,3.73837097186236)
(842,3.73836663183237)
(843,3.73836108335151)
(844,3.73835602080752)
(845,3.73835055704804)
(846,3.7383453271969)
(847,3.73834024667299)
(848,3.73833463207854)
(849,3.73832918703276)
(850,3.73832291413042)
(851,3.73831687036546)
(852,3.73831125938319)
(853,3.73830514651685)
(854,3.73829991680505)
(855,3.73829440892003)
(856,3.73828878673197)
(857,3.73828492855531)
(858,3.73827867671623)
(859,3.73827541310651)
(860,3.73827013837977)
(861,3.73826546873338)
(862,3.73826032775041)
(863,3.73825597165023)
(864,3.7382512648817)
(865,3.73824696765095)
(866,3.73824254523716)
(867,3.7382379994685)
(868,3.73823371601288)
(869,3.73822972549451)
(870,3.73822632560494)
(871,3.73822244872733)
(872,3.73821918459051)
(873,3.73821545528507)
(874,3.73821302782372)
(875,3.73820938709438)
(876,3.73820629217906)
(877,3.73820275150971)
(878,3.73819902044947)
(879,3.73819619970524)
(880,3.73819282857863)
(881,3.73818966762606)
(882,3.7381860574362)
(883,3.73818265774078)
(884,3.73817922485455)
(885,3.73817584588696)
(886,3.73817224097197)
(887,3.73816854419949)
(888,3.73816513570857)
(889,3.73816169701156)
(890,3.73815822907931)
(891,3.738154984763)
(892,3.7381514622296)
(893,3.73814881262211)
(894,3.73814528322617)
(895,3.73814222112149)
(896,3.73813896525792)
(897,3.73813595598848)
(898,3.73813340634269)
(899,3.73813059803052)
(900,3.7381277791384)
(901,3.73812488939121)
(902,3.73812223485436)
(903,3.738119523562)
(904,3.73811680377472)
(905,3.73811364019401)
(906,3.73811058778631)
(907,3.73810768352435)
(908,3.73810482442504)
(909,3.73810208533219)
(910,3.73809903221664)
(911,3.73809600569151)
(912,3.73809255900566)
(913,3.73808923457356)
(914,3.73808606315053)
(915,3.73808249898209)
(916,3.73807919745919)
(917,3.7380758203985)
(918,3.7380730837772)
(919,3.73806991960299)
(920,3.73806706663024)
(921,3.73806439392417)
(922,3.73806144849824)
(923,3.73805888068136)
(924,3.73805574565061)
(925,3.73805283912954)
(926,3.73804986567239)
(927,3.73804727935088)
(928,3.73804461017825)
(929,3.738042270261)
(930,3.73803924710539)
(931,3.73803617191435)
(932,3.73803334611889)
(933,3.73803077743158)
(934,3.73802840684875)
(935,3.7380254392557)
(936,3.73802278937475)
(937,3.73801986469609)
(938,3.73801735068798)
(939,3.7380152051394)
(940,3.73801277545141)
(941,3.73801038977061)
(942,3.73800794877972)
(943,3.738005806859)
(944,3.73800336383142)
(945,3.73800155049602)
(946,3.73799943022637)
(947,3.7379975465971)
(948,3.73799545398977)
(949,3.73799329822184)
(950,3.7379913470985)
(951,3.73798894849061)
(952,3.73798661737822)
(953,3.73798395675553)
(954,3.7379811405716)
(955,3.73797845319369)
(956,3.73797576076701)
(957,3.73797395783296)
(958,3.73797136749773)
(959,3.73796935482986)
(960,3.7379668332172)
(961,3.73796445905763)
(962,3.73796222369774)
(963,3.73795983591007)
(964,3.73795762023251)
(965,3.73795502080226)
(966,3.73795254930971)
(967,3.73794994355625)
(968,3.73794740372277)
(969,3.73794484234453)
(970,3.73794235330776)
(971,3.73793976003286)
(972,3.73793738086568)
(973,3.73793478184286)
(974,3.73793241889013)
(975,3.73792973722227)
(976,3.73792759920222)
(977,3.73792499382056)
(978,3.73792301105486)
(979,3.73792085709439)
(980,3.73791863842735)
(981,3.73791644598183)
(982,3.73791424258246)
(983,3.73791210217146)
(984,3.73790993142622)
(985,3.7379079095607)
(986,3.73790596917292)
(987,3.73790406917185)
(988,3.7379021328438)
(989,3.73790010390277)
(990,3.73789798257262)
(991,3.73789573849813)
(992,3.73789332214728)
(993,3.73789099321558)
(994,3.73788872315638)
(995,3.73788659227978)
(996,3.7378844678229)
(997,3.73788238443127)
(998,3.73788001244578)
(999,3.73787777361877)
(1000,3.73787559857741)
(1001,3.73787363657249)
(1002,3.73787134590187)
(1003,3.73786962462704)
(1004,3.73786732096394)
(1005,3.73786543193935)
(1006,3.73786341171314)
(1007,3.73786143982061)
(1008,3.73785933256784)
(1009,3.73785727496065)
(1010,3.7378549937527)
(1011,3.73785327666587)
(1012,3.73785120321333)
(1013,3.73784936251772)
(1014,3.73784744529523)
(1015,3.73784545629565)
(1016,3.73784367740762)
(1017,3.73784174955561)
(1018,3.73784013250464)
(1019,3.73783820960295)
(1020,3.73783656398327)
(1021,3.73783475691151)
(1022,3.73783290794361)
(1023,3.73783092634172)
(1024,3.73782895564209)
(1025,3.73782706026381)
(1026,3.7378251225944)
(1027,3.73782290273813)
(1028,3.73782103014618)
(1029,3.73781944793876)
(1030,3.73781743459305)
(1031,3.73781606714392)
(1032,3.73781439517502)
(1033,3.73781260601281)
(1034,3.73781077490974)
(1035,3.73780907424129)
(1036,3.73780732914821)
(1037,3.73780586743132)
(1038,3.73780396584181)
(1039,3.73780249381773)
(1040,3.73780080961943)
(1041,3.7377995018793)
(1042,3.73779803826332)
(1043,3.73779657575801)
(1044,3.73779491621912)
(1045,3.7377931979854)
(1046,3.73779161775134)
(1047,3.73779033791641)
(1048,3.73778887316342)
(1049,3.7377877897758)
(1050,3.73778620500413)
(1051,3.73778506843068)
(1052,3.73778416471776)
(1053,3.73778260549099)
(1054,3.73778174262242)
(1055,3.73778069006054)
(1056,3.73777969017323)
(1057,3.73777880875506)
(1058,3.73777783454191)
(1059,3.73777695419631)
(1060,3.73777600634838)
(1061,3.7377750327335)
(1062,3.73777398920056)
(1063,3.73777306393962)
(1064,3.73777215564349)
(1065,3.73777142353804)
(1066,3.73777041282186)
(1067,3.73776957896351)
(1068,3.73776859524338)
(1069,3.73776779469995)
(1070,3.7377669318555)
(1071,3.73776614028247)
(1072,3.73776528438792)
(1073,3.73776444469955)
(1074,3.73776366949448)
(1075,3.73776295787662)
(1076,3.73776229564913)
(1077,3.73776153431477)
(1078,3.73776080537183)
(1079,3.73776001308687)
(1080,3.73775925332851)
(1081,3.73775848189563)
(1082,3.73775770819448)
(1083,3.73775692860469)
(1084,3.73775606777868)
(1085,3.7377551993551)
(1086,3.73775426707449)
(1087,3.7377533883616)
(1088,3.73775249517319)
(1089,3.73775159515056)
(1090,3.73775062651259)
(1091,3.73774964547174)
(1092,3.7377487280397)
(1093,3.73774783450545)
(1094,3.73774696245769)
(1095,3.7377460913254)
(1096,3.73774521027453)
(1097,3.73774437109701)
(1098,3.73774352008182)
(1099,3.73774265814851)
(1100,3.73774176048393)
(1101,3.73774084119726)
(1102,3.73773990751915)
(1103,3.73773900572545)
(1104,3.73773806794322)
(1105,3.73773719278047)
(1106,3.73773624508499)
(1107,3.73773521718683)
(1108,3.73773412527184)
(1109,3.73773313687008)
(1110,3.73773214220548)
(1111,3.73773144429205)
(1112,3.73773055924199)
(1113,3.73772974500404)
(1114,3.73772887768844)
(1115,3.73772805425816)
(1116,3.73772733127552)
(1117,3.73772646498446)
(1118,3.73772584015222)
(1119,3.73772502866064)
(1120,3.73772442444391)
(1121,3.73772376832508)
(1122,3.73772310878609)
(1123,3.73772237340603)
(1124,3.73772156767873)
(1125,3.73772078112301)
(1126,3.73771991074325)
(1127,3.73771924954994)
(1128,3.73771856522111)
(1129,3.73771795111994)
(1130,3.73771733073921)
(1131,3.73771679049915)
(1132,3.7377162780014)
(1133,3.73771582706726)
(1134,3.73771535828913)
(1135,3.73771490951183)
(1136,3.73771446921926)
(1137,3.73771403111815)
(1138,3.73771358812464)
(1139,3.73771316301315)
(1140,3.73771279031696)
(1141,3.73771237189736)
(1142,3.73771203750067)
(1143,3.7377116599452)
(1144,3.73771127172099)
(1145,3.73771085822908)
(1146,3.73771044543171)
(1147,3.73771003003628)
(1148,3.73770961395682)
(1149,3.73770923764058)
(1150,3.73770889922817)
(1151,3.73770859452372)
(1152,3.73770827919211)
(1153,3.73770797626651)
(1154,3.73770767117597)
(1155,3.73770736857502)
(1156,3.73770706574281)
(1157,3.7377067357417)
(1158,3.73770644452416)
(1159,3.7377061624599)
(1160,3.73770591618359)
(1161,3.73770565186327)
(1162,3.73770540349742)
(1163,3.7377051498843)
(1164,3.73770489828832)
(1165,3.73770464728056)
(1166,3.73770439070047)
(1167,3.73770415575694)
(1168,3.73770391578511)
(1169,3.73770368437559)
(1170,3.73770345536176)
(1171,3.73770322558527)
(1172,3.73770298686895)
(1173,3.73770273028132)
(1174,3.73770245482769)
(1175,3.73770217413684)
(1176,3.7377019331503)
(1177,3.73770169907311)
(1178,3.7377015139343)
(1179,3.73770127989218)
(1180,3.73770105128693)
(1181,3.73770079954456)
(1182,3.73770054955769)
(1183,3.73770029304605)
(1184,3.73770004125366)
(1185,3.73769977530764)
(1186,3.73769951295856)
(1187,3.73769926266991)
(1188,3.73769900336662)
(1189,3.73769871651389)
(1190,3.73769839480367)
(1191,3.737698074719)
(1192,3.73769774179859)
(1193,3.73769748221526)
(1194,3.73769719518852)
(1195,3.73769690637777)
(1196,3.73769663167631)
(1197,3.73769633574138)
(1198,3.73769607853776)
(1199,3.7376958144059)
(1200,3.73769554493178)
(1201,3.73769526084129)
(1202,3.73769497588662)
(1203,3.7376947008467)
(1204,3.73769439126651)
(1205,3.73769413306921)
(1206,3.73769380868222)
(1207,3.73769350853485)
(1208,3.73769317393108)
(1209,3.73769283009798)
(1210,3.73769247135905)
(1211,3.73769209858761)
(1212,3.73769171699872)
(1213,3.73769132551348)
(1214,3.73769094369066)
(1215,3.73769054557034)
(1216,3.73769011067508)
(1217,3.73768962369373)
(1218,3.73768910674606)
(1219,3.73768860218847)
(1220,3.73768810351532)
(1221,3.73768761414745)
(1222,3.73768711915892)
(1223,3.73768659307928)
(1224,3.73768607392712)
(1225,3.73768554996432)
(1226,3.73768501421687)
(1227,3.73768446010595)
(1228,3.73768388223272)
(1229,3.73768327250755)
(1230,3.73768265894461)
(1231,3.73768206471346)
(1232,3.73768146312711)
(1233,3.737680829661)
(1234,3.73768015617143)
(1235,3.73767948454796)
(1236,3.73767880574565)
(1237,3.7376781884305)
(1238,3.73767757991299)
(1239,3.73767702860267)
(1240,3.73767640534893)
(1241,3.73767589011394)
(1242,3.73767534393083)
(1243,3.73767481378436)
(1244,3.73767425344704)
(1245,3.73767369977887)
(1246,3.73767312260475)
(1247,3.73767250402789)
(1248,3.73767185901266)
(1249,3.73767124867968)
(1250,3.7376707212574)
(1251,3.73767020360522)
(1252,3.73766979089013)
(1253,3.73766931914386)
(1254,3.73766888418013)
(1255,3.7376684384465)
(1256,3.73766802886091)
(1257,3.73766763370181)
(1258,3.73766722439677)
(1259,3.73766678772803)
(1260,3.73766633837701)
(1261,3.73766593285758)
(1262,3.73766557431165)
(1263,3.73766522909212)
(1264,3.73766486655208)
(1265,3.73766447741273)
(1266,3.73766407889038)
(1267,3.7376636872703)
(1268,3.73766330797546)
(1269,3.73766293878962)
(1270,3.73766256917968)
(1271,3.73766217565688)
(1272,3.73766178751651)
(1273,3.73766138093199)
(1274,3.73766097219518)
(1275,3.73766054868311)
(1276,3.73766012064062)
(1277,3.73765968773187)
(1278,3.73765929004936)
(1279,3.7376588983647)
(1280,3.73765857729948)
(1281,3.73765824298788)
(1282,3.73765791391428)
(1283,3.73765754589012)
(1284,3.73765716449733)
(1285,3.73765675708174)
(1286,3.73765636136766)
(1287,3.73765592835829)
(1288,3.73765549899108)
(1289,3.73765507510088)
(1290,3.73765465342127)
(1291,3.73765421736184)
(1292,3.73765375636746)
(1293,3.73765328605144)
(1294,3.7376528031705)
(1295,3.73765230855558)
(1296,3.73765182613358)
(1297,3.73765138604914)
(1298,3.73765090601406)
(1299,3.73765049491484)
(1300,3.73765005768532)
(1301,3.7376496716272)
(1302,3.73764923433049)
(1303,3.73764886032749)
(1304,3.73764846126802)
(1305,3.73764806695725)
(1306,3.73764766494201)
(1307,3.73764733409389)
(1308,3.73764695345439)
(1309,3.73764667951635)
(1310,3.73764636466859)
(1311,3.73764606151295)
(1312,3.73764572119771)
(1313,3.73764539506465)
(1314,3.7376450736006)
(1315,3.73764475219649)
(1316,3.73764441786011)
(1317,3.73764409772302)
(1318,3.73764384546962)
(1319,3.7376435493154)
(1320,3.7376433645728)
(1321,3.73764314343803)
(1322,3.73764293393858)
(1323,3.73764270632371)
(1324,3.73764248831257)
(1325,3.73764226579396)
(1326,3.73764204223729)
(1327,3.73764181162365)
(1328,3.73764158827036)
(1329,3.73764138767165)
(1330,3.73764120492894)
(1331,3.7376410260218)
(1332,3.73764083475087)
(1333,3.73764061870902)
(1334,3.73764037767109)
(1335,3.73764012360505)
(1336,3.73763987771454)
(1337,3.7376396517877)
(1338,3.73763943309605)
(1339,3.73763920315945)
(1340,3.73763896945233)
(1341,3.73763873555047)
(1342,3.73763851357945)
(1343,3.7376382983679)
(1344,3.73763807808236)
(1345,3.73763784460468)
(1346,3.73763762497371)
(1347,3.73763741941811)
(1348,3.73763727192147)
(1349,3.73763710826708)
(1350,3.73763694811504)
(1351,3.73763675994024)
(1352,3.73763657289571)
(1353,3.73763639096094)
(1354,3.73763621887288)
(1355,3.73763606516677)
(1356,3.73763589480864)
(1357,3.73763575572018)
(1358,3.7376355877625)
(1359,3.73763540746619)
(1360,3.73763520898492)
(1361,3.73763501634436)
(1362,3.73763483216651)
(1363,3.73763465906271)
(1364,3.73763444415595)
(1365,3.73763426762508)
(1366,3.7376341249083)
(1367,3.73763399850648)
(1368,3.73763387003244)
(1369,3.73763372118102)
(1370,3.73763355394754)
(1371,3.73763337076347)
(1372,3.73763318226683)
(1373,3.73763300957557)
(1374,3.73763285627397)
(1375,3.73763272921596)
(1376,3.73763258036911)
(1377,3.73763242562811)
(1378,3.73763225679472)
(1379,3.73763210166237)
(1380,3.73763194127076)
(1381,3.73763177538933)
(1382,3.73763159616711)
(1383,3.73763142420793)
(1384,3.73763127614555)
(1385,3.73763114330174)
(1386,3.73763101100519)
(1387,3.73763086605537)
(1388,3.73763071601502)
(1389,3.73763056102673)
(1390,3.73763040678395)
(1391,3.73763025443388)
(1392,3.73763010651576)
(1393,3.73762996457433)
(1394,3.73762981090172)
(1395,3.73762967157435)
(1396,3.73762953207794)
(1397,3.73762940403861)
(1398,3.73762927126971)
(1399,3.73762913085929)
(1400,3.73762897165465)
(1401,3.73762881289319)
(1402,3.73762866940379)
(1403,3.73762851794946)
(1404,3.73762840128511)
(1405,3.73762827898554)
(1406,3.73762815249144)
(1407,3.73762801440821)
(1408,3.7376278791837)
(1409,3.737627741139)
(1410,3.73762760239068)
(1411,3.73762746070811)
(1412,3.73762731893616)
(1413,3.7376271795589)
(1414,3.73762704360246)
(1415,3.73762691339655)
(1416,3.73762678357674)
(1417,3.73762664896301)
(1418,3.73762650301095)
(1419,3.73762634551589)
(1420,3.73762619738923)
(1421,3.73762605917187)
(1422,3.73762592842149)
(1423,3.73762579325494)
(1424,3.73762565187908)
(1425,3.73762550779101)
(1426,3.73762536482536)
(1427,3.7376252194639)
(1428,3.73762506785865)
(1429,3.73762491092845)
(1430,3.73762474815163)
(1431,3.73762458447438)
(1432,3.73762442224002)
(1433,3.73762426148421)
(1434,3.73762409788079)
(1435,3.73762392111662)
(1436,3.73762372939314)
(1437,3.73762353033168)
(1438,3.7376233431922)
(1439,3.73762316337049)
(1440,3.73762300499781)
(1441,3.73762283337373)
(1442,3.73762265294986)
(1443,3.73762246609427)
(1444,3.73762227691635)
(1445,3.73762208489515)
(1446,3.73762188226807)
(1447,3.73762166692031)
(1448,3.73762145832925)
(1449,3.7376212527125)
(1450,3.73762109354055)
(1451,3.73762093275122)
(1452,3.73762076558291)
(1453,3.73762057658399)
(1454,3.73762037330604)
(1455,3.73762015793678)
(1456,3.73761992935639)
(1457,3.73761968848752)
(1458,3.73761944760438)
(1459,3.73761923936954)
(1460,3.73761904170001)
(1461,3.73761887684832)
(1462,3.73761867942299)
(1463,3.73761848064535)
(1464,3.73761825704742)
(1465,3.73761803861037)
(1466,3.73761780815527)
(1467,3.73761762408757)
(1468,3.73761741376535)
(1469,3.73761720951149)
(1470,3.73761699511336)
(1471,3.73761677640477)
(1472,3.73761654364145)
(1473,3.73761629476186)
(1474,3.73761603005415)
(1475,3.73761576831072)
(1476,3.73761553803933)
(1477,3.73761532837887)
(1478,3.73761509561706)
(1479,3.73761488263892)
(1480,3.73761466455759)
(1481,3.73761445499419)
(1482,3.73761423251479)
(1483,3.73761401873113)
(1484,3.73761379358154)
(1485,3.73761359131229)
(1486,3.73761336656554)
(1487,3.73761315347859)
(1488,3.73761294611287)
(1489,3.7376127445138)
(1490,3.73761254543819)
(1491,3.73761235722166)
(1492,3.73761215305585)
(1493,3.73761200841957)
(1494,3.73761185159645)
(1495,3.73761170991816)
(1496,3.73761154242304)
(1497,3.73761136162733)
(1498,3.73761117634399)
(1499,3.73761098890008)
(1500,3.73761082245617)
(1501,3.73761063768726)
(1502,3.73761047922827)
(1503,3.73761033182616)
(1504,3.73761020301296)
(1505,3.73761000742428)
(1506,3.73760986981618)
(1507,3.73760971615042)
(1508,3.73760957121213)
(1509,3.73760942139378)
(1510,3.73760926889972)
(1511,3.73760911978392)
(1512,3.73760895219256)
(1513,3.73760878556245)
(1514,3.73760862738326)
(1515,3.73760848420832)
(1516,3.73760835176741)
(1517,3.73760821591986)
(1518,3.73760807665891)
(1519,3.73760794281897)
(1520,3.73760780074793)
(1521,3.73760768653305)
(1522,3.7376075678191)
(1523,3.73760745190176)
(1524,3.73760732645725)
(1525,3.73760720331273)
(1526,3.7376070915162)
(1527,3.73760697349069)
(1528,3.73760685797723)
(1529,3.73760673826564)
(1530,3.7376066257528)
(1531,3.73760651605343)
(1532,3.73760640586584)
(1533,3.73760630379022)
(1534,3.73760620534581)
(1535,3.73760611051832)
(1536,3.73760601273951)
(1537,3.73760591134992)
(1538,3.7376058061787)
(1539,3.7376057042312)
(1540,3.73760560961895)
(1541,3.73760552315037)
(1542,3.73760544056886)
(1543,3.73760535909402)
(1544,3.73760527598597)
(1545,3.73760518856434)
(1546,3.73760509504293)
(1547,3.73760499745276)
(1548,3.73760489961755)
(1549,3.73760480296526)
(1550,3.73760470619765)
(1551,3.73760460864083)
(1552,3.7376045116938)
(1553,3.7376044157956)
(1554,3.7376043186507)
(1555,3.73760421799344)
(1556,3.73760411461452)
(1557,3.73760401132712)
(1558,3.73760390919485)
(1559,3.73760380724487)
(1560,3.73760370539528)
(1561,3.73760360498452)
(1562,3.73760350568369)
(1563,3.7376034047352)
(1564,3.73760330000792)
(1565,3.73760319386773)
(1566,3.73760309534674)
(1567,3.73760299899253)
(1568,3.73760291749075)
(1569,3.73760282680345)
(1570,3.73760273719772)
(1571,3.73760263997921)
(1572,3.73760253821961)
(1573,3.73760242776214)
(1574,3.73760230886317)
(1575,3.73760218526445)
(1576,3.73760205660743)
(1577,3.73760194672052)
(1578,3.73760184960159)
(1579,3.73760174999683)
(1580,3.73760164386156)
(1581,3.73760154318748)
(1582,3.73760143631602)
(1583,3.7376013341355)
(1584,3.73760122584062)
(1585,3.73760111936052)
(1586,3.7376010180417)
(1587,3.73760092056162)
(1588,3.73760082191887)
(1589,3.73760072204599)
(1590,3.73760062309931)
(1591,3.7376005230976)
(1592,3.73760041799236)
(1593,3.73760030903755)
(1594,3.73760020331269)
(1595,3.73760010263913)
(1596,3.73760000831251)
(1597,3.73759990605645)
(1598,3.73759981901611)
(1599,3.73759972244397)
(1600,3.73759963309246)
(1601,3.73759953859021)
(1602,3.73759944682532)
(1603,3.73759934051997)
(1604,3.73759925126324)
(1605,3.73759915527417)
(1606,3.73759907933958)
(1607,3.73759900008198)
(1608,3.7375989272918)
(1609,3.73759884266253)
(1610,3.73759875718111)
(1611,3.7375986624251)
(1612,3.73759859751473)
(1613,3.73759852887345)
(1614,3.73759846374757)
(1615,3.737598388458)
(1616,3.73759831139809)
(1617,3.73759823257879)
(1618,3.73759815282548)
(1619,3.73759806848271)
(1620,3.73759798172708)
(1621,3.73759789742966)
(1622,3.73759781075065)
(1623,3.73759771969863)
(1624,3.73759762845506)
(1625,3.73759754105537)
(1626,3.73759745671501)
(1627,3.73759737377796)
(1628,3.7375972743465)
(1629,3.73759719044345)
(1630,3.73759710801062)
(1631,3.73759703233459)
(1632,3.73759695208945)
(1633,3.7375968687333)
(1634,3.73759678093925)
(1635,3.73759668817345)
(1636,3.7375965903474)
(1637,3.73759649536603)
(1638,3.7375964087352)
(1639,3.73759632066769)
(1640,3.73759622205811)
(1641,3.73759611316816)
(1642,3.73759600556999)
(1643,3.73759590537958)
(1644,3.73759580750648)
(1645,3.73759570442166)
(1646,3.73759559725357)
(1647,3.73759549378158)
(1648,3.73759540099863)
(1649,3.7375953244327)
(1650,3.73759523858968)
(1651,3.73759517218731)
(1652,3.73759509783738)
(1653,3.73759502838762)
(1654,3.737594958195)
(1655,3.7375948944902)
(1656,3.73759483394906)
(1657,3.73759477310349)
(1658,3.73759468928392)
(1659,3.73759461627794)
(1660,3.73759455163258)
(1661,3.73759449817772)
(1662,3.73759444381947)
(1663,3.73759438472119)
(1664,3.73759432202903)
(1665,3.73759426185691)
(1666,3.73759420683496)
(1667,3.73759415627843)
(1668,3.73759410713664)
(1669,3.73759405527205)
(1670,3.7375939987328)
(1671,3.73759394046587)
(1672,3.73759388655925)
(1673,3.73759383848533)
(1674,3.73759379096115)
(1675,3.73759373764579)
(1676,3.73759367790876)
(1677,3.73759361847882)
(1678,3.73759356651393)
(1679,3.73759352238252)
(1680,3.73759348022548)
(1681,3.73759343555033)
(1682,3.73759338615417)
(1683,3.73759333710601)
(1684,3.73759328996952)
(1685,3.73759324855301)
(1686,3.73759320899794)
(1687,3.73759316732303)
(1688,3.73759312149506)
(1689,3.73759307467979)
(1690,3.73759303176461)
(1691,3.73759299401631)
(1692,3.73759295527858)
(1693,3.73759291455147)
(1694,3.737592868634)
(1695,3.73759282127566)
(1696,3.73759277628577)
(1697,3.73759273495557)
(1698,3.73759269409416)
(1699,3.73759265002587)
(1700,3.7375926011103)
(1701,3.7375925494732)
(1702,3.73759250017444)
(1703,3.73759245756809)
(1704,3.73759242040558)
(1705,3.73759238222926)
(1706,3.73759234399478)
(1707,3.73759230393978)
(1708,3.73759226777552)
(1709,3.73759224117847)
(1710,3.73759221274482)
(1711,3.73759218450323)
(1712,3.73759215258561)
(1713,3.73759212025795)
(1714,3.73759208880398)
(1715,3.73759205983218)
(1716,3.73759203015348)
(1717,3.73759199901293)
(1718,3.7375919689404)
(1719,3.7375919381267)
(1720,3.73759190717107)
(1721,3.73759187659196)
(1722,3.73759184221879)
(1723,3.73759180994768)
(1724,3.73759177551439)
(1725,3.73759174031505)
(1726,3.73759170489796)
(1727,3.73759167089929)
(1728,3.73759163656053)
(1729,3.7375916004347)
(1730,3.7375915630783)
(1731,3.7375915282947)
(1732,3.73759149809411)
(1733,3.73759146362884)
(1734,3.73759143386064)
(1735,3.73759140115549)
(1736,3.73759136793234)
(1737,3.73759133096433)
(1738,3.73759129161984)
(1739,3.73759124804235)
(1740,3.73759120083625)
(1741,3.73759114731531)
(1742,3.73759109352298)
(1743,3.7375910446454)
(1744,3.73759100027234)
(1745,3.73759095600196)
(1746,3.73759090707817)
(1747,3.73759085215696)
(1748,3.73759079503385)
(1749,3.73759074406772)
(1750,3.73759069253628)
(1751,3.73759065676692)
(1752,3.73759061428324)
(1753,3.73759057499449)
(1754,3.73759052645851)
(1755,3.73759048064913)
(1756,3.73759043455754)
(1757,3.73759038827876)
(1758,3.73759033712082)
(1759,3.73759028639018)
(1760,3.73759024157345)
(1761,3.73759019948979)
(1762,3.73759015618433)
(1763,3.73759010850923)
(1764,3.73759005680098)
(1765,3.7375900017183)
(1766,3.73758994411765)
(1767,3.73758988682915)
(1768,3.73758983203402)
(1769,3.73758978032802)
(1770,3.7375897260059)
(1771,3.7375896762351)
(1772,3.7375896203976)
(1773,3.73758957089212)
(1774,3.73758951796816)
(1775,3.73758946328458)
(1776,3.73758940384114)
(1777,3.73758934322023)
(1778,3.73758928272387)
(1779,3.7375892256255)
(1780,3.73758917537962)
(1781,3.73758912699768)
(1782,3.73758908072046)
(1783,3.73758903985921)
(1784,3.73758897858366)
(1785,3.73758894248497)
(1786,3.73758890463421)
(1787,3.73758887145974)
(1788,3.73758883391051)
(1789,3.73758879580625)
(1790,3.73758875521873)
(1791,3.73758871383408)
(1792,3.73758867142734)
(1793,3.73758862957653)
(1794,3.73758859023648)
(1795,3.73758855283849)
(1796,3.73758851594527)
(1797,3.7375884784074)
(1798,3.73758844092796)
(1799,3.73758840253326)
(1800,3.73758836633553)
(1801,3.73758833080204)
(1802,3.73758829566746)
(1803,3.73758825943895)
(1804,3.73758822261112)
(1805,3.73758818636827)
(1806,3.73758815188703)
(1807,3.73758811837005)
(1808,3.73758808434215)
(1809,3.73758804967374)
(1810,3.7375880141639)
(1811,3.73758797843636)
(1812,3.73758794257115)
(1813,3.73758790706634)
(1814,3.73758787145831)
(1815,3.73758783522635)
(1816,3.73758779735354)
(1817,3.73758775838679)
(1818,3.73758771995462)
(1819,3.73758768302255)
(1820,3.73758764647465)
(1821,3.73758760874016)
(1822,3.7375875698626)
(1823,3.73758753234608)
(1824,3.73758749891761)
(1825,3.73758746699951)
(1826,3.73758743997837)
(1827,3.73758741102558)
(1828,3.7375873807198)
(1829,3.73758734904967)
(1830,3.7375873159441)
(1831,3.73758728306383)
(1832,3.73758724941525)
(1833,3.73758721429308)
(1834,3.73758717894605)
(1835,3.73758714768612)
(1836,3.73758711820917)
(1837,3.7375870900657)
(1838,3.73758706294216)
(1839,3.73758703212777)
(1840,3.73758700582025)
(1841,3.73758698081453)
(1842,3.73758695723698)
(1843,3.73758693405397)
(1844,3.73758691019978)
(1845,3.73758688348379)
(1846,3.73758685442763)
(1847,3.73758682330132)
(1848,3.73758679452154)
(1849,3.73758676927051)
(1850,3.73758674517658)
(1851,3.73758671970837)
(1852,3.73758669209737)
(1853,3.73758666355251)
(1854,3.73758663585114)
(1855,3.73758661037551)
(1856,3.73758658579673)
(1857,3.73758656060622)
(1858,3.73758653405735)
(1859,3.73758650720927)
(1860,3.73758648167257)
(1861,3.73758645798092)
(1862,3.7375864346905)
(1863,3.73758640996677)
(1864,3.7375863836568)
(1865,3.73758635614852)
(1866,3.7375863316304)
(1867,3.73758630945259)
(1868,3.73758628857717)
(1869,3.73758626639471)
(1870,3.73758624174587)
(1871,3.73758621505475)
(1872,3.73758618847645)
(1873,3.73758616339974)
(1874,3.73758613902944)
(1875,3.73758611352485)
(1876,3.73758608456499)
(1877,3.73758605190861)
(1878,3.73758601717574)
(1879,3.73758598278636)
(1880,3.73758594961869)
(1881,3.7375859165012)
(1882,3.73758588189546)
(1883,3.73758584570932)
(1884,3.73758580970537)
(1885,3.73758577671192)
(1886,3.73758574529538)
(1887,3.73758571897931)
(1888,3.73758568752634)
(1889,3.73758566235755)
(1890,3.73758563498478)
(1891,3.73758561179947)
(1892,3.7375855885306)
(1893,3.73758556468993)
(1894,3.73758553707286)
(1895,3.73758550590131)
(1896,3.7375854748175)
(1897,3.73758544566073)
(1898,3.73758542108586)
(1899,3.73758539761021)
(1900,3.73758537253755)
(1901,3.73758534456351)
(1902,3.73758531510785)
(1903,3.73758528569317)
(1904,3.73758525670081)
(1905,3.73758522724986)
(1906,3.73758519624261)
(1907,3.73758516396205)
(1908,3.73758513103556)
(1909,3.73758509905426)
(1910,3.73758506783006)
(1911,3.73758503589382)
(1912,3.73758500093893)
(1913,3.73758496184296)
(1914,3.73758492024349)
(1915,3.73758488025846)
(1916,3.73758484464449)
(1917,3.73758481238568)
(1918,3.73758478076888)
(1919,3.73758474834576)
(1920,3.73758471583524)
(1921,3.73758468415501)
(1922,3.7375846530251)
(1923,3.737584621063)
(1924,3.73758458712408)
(1925,3.73758455127049)
(1926,3.73758451508774)
(1927,3.73758448074828)
(1928,3.73758444900891)
(1929,3.73758441827352)
(1930,3.73758438636364)
(1931,3.73758435315471)
(1932,3.73758431776736)
(1933,3.73758428677122)
(1934,3.73758425661823)
(1935,3.73758422748775)
(1936,3.73758419605623)
(1937,3.73758416357973)
(1938,3.73758413089271)
(1939,3.73758409918838)
(1940,3.73758406771552)
(1941,3.73758403658106)
(1942,3.73758400854234)
(1943,3.73758398078771)
(1944,3.7375839517959)
(1945,3.73758392029415)
(1946,3.73758388761671)
(1947,3.73758385512754)
(1948,3.73758382280765)
(1949,3.73758378950549)
(1950,3.73758375483989)
(1951,3.73758371990952)
(1952,3.73758368631191)
(1953,3.73758365389461)
(1954,3.73758362264007)
(1955,3.73758359257549)
(1956,3.73758356369335)
(1957,3.73758353541574)
(1958,3.73758350816288)
(1959,3.73758348413104)
(1960,3.73758346080752)
(1961,3.73758344343344)
(1962,3.73758342475346)
(1963,3.73758340574346)
(1964,3.73758338437881)
(1965,3.73758336269404)
(1966,3.73758334033704)
(1967,3.73758331619967)
(1968,3.73758328894128)
(1969,3.73758326141635)
(1970,3.7375832388656)
(1971,3.73758321866958)
(1972,3.73758319773051)
(1973,3.73758317340571)
(1974,3.73758314693893)
(1975,3.73758312067294)
(1976,3.73758309583166)
(1977,3.73758307259917)
(1978,3.73758305081917)
(1979,3.7375830300488)
(1980,3.73758300936925)
(1981,3.73758298828815)
(1982,3.73758296812426)
(1983,3.73758295151039)
(1984,3.73758293723906)
(1985,3.73758292621097)
(1986,3.73758291413958)
(1987,3.7375829041734)
(1988,3.73758289526207)
(1989,3.73758288736179)

};
\path [draw=black, fill opacity=0] (axis cs:13,1)--(axis cs:13,1);

\path [draw=black, fill opacity=0] (axis cs:13,0)--(axis cs:13,0);

\path [draw=black, fill opacity=0] (axis cs:1,13)--(axis cs:1,13);

\path [draw=black, fill opacity=0] (axis cs:0,13)--(axis cs:0,13);

\end{axis}

\end{tikzpicture}}
   		\end{column}\hfill
   		\begin{column}{.5\textwidth}
   			$F(x) = \lVert Ax - b \rVert + \lambda \lVert x \rVert_1$\\
   			$A \in \mathbb{R}^{1500 \times 3000},\:b \in \mathbb{R}^{1500}$\\
   			$A_{ij},\:b_i\:$ \textasciitilde $\:\mathcal{N}(0,1)$, $\:\lambda = 0.1$\\
   			\vspace{10pt}
   			\resizebox{\linewidth}{!}{% This file was created by matplotlib v0.1.0.
% Copyright (c) 2010--2014, Nico Schl�mer <nico.schloemer@gmail.com>
% All rights reserved.
% 
% The lastest updates can be retrieved from
% 
% https://github.com/nschloe/matplotlib2tikz
% 
% where you can also submit bug reports and leavecomments.
% 
\begin{tikzpicture}

\begin{axis}[
xlabel={Number of Iterations},
ylabel={Function Value},
xmin=0, xmax=60,
ymin=3.73758288736179, ymax=100000,
ymode=log,
axis on top,
legend entries={{0SR1},{ProxGrad},{L-BFGS-B}}
]
\addplot [thick, red]
coordinates {
(0,23930.000884189)
(1,7180.45388604586)
(2,1602.10980923053)
(3,580.81114454538)
(4,519.996842220967)
(5,203.222944097077)
(6,121.922299073436)
(7,51.6523543005163)
(8,59.2695972484554)
(9,30.819611673989)
(10,25.614751730395)
(11,21.7592732486774)
(12,22.654003528062)
(13,20.15049001864)
(14,19.850460675318)
(15,18.3912111072782)
(16,23.73822362313)
(17,17.8866027335036)
(18,17.8196671916423)
(19,17.7775123499805)
(20,18.0615078838142)
(21,17.7506027746239)
(22,17.7424949039984)
(23,17.7297358775166)
(24,18.0476551294605)
(25,17.7168284338285)
(26,17.7118227102102)
(27,17.897336767244)
(28,17.7264425474283)
(29,17.6583909070735)
(30,17.6466963489978)
(31,17.5726603204587)
(32,17.9785494963624)
(33,17.5007993751903)
(34,17.4774762719696)
(35,17.4693572907426)
(36,18.0844886660129)
(37,17.4544077868014)
(38,17.4488948638463)
(39,17.3764214969115)
(40,17.5899917220372)
(41,17.1689206218387)
(42,17.0926131263914)
(43,17.193691075103)
(44,19.0707895757655)
(45,16.9630825309593)
(46,16.9036371416858)
(47,16.8413981063643)
(48,16.8743377801165)
(49,16.6801578404075)
(50,16.5909092149981)
(51,15.9390711318578)
(52,15.2212036355576)
(53,13.1777255991126)
(54,11.3509943341348)
(55,10.0944946484959)
(56,9.71465222794144)
(57,9.40843618327981)
(58,9.31008072896225)
(59,9.12293542675185)
(60,8.97211792181375)
(61,9.96955580951941)
(62,8.8971778312592)
(63,8.89032770510936)
(64,8.846532027156)
(65,9.14275424383174)
(66,8.76428614407871)
(67,8.74549967678879)
(68,8.81264423855369)
(69,8.92224051152909)
(70,8.71372433712952)
(71,8.69686120584995)
(72,8.86066032777056)
(73,8.64994348063148)
(74,8.64177834237841)
(75,8.61074429525064)
(76,8.97413158797716)
(77,8.57239195780947)
(78,8.56084322589053)
(79,8.56535473569669)
(80,8.70378144008176)
(81,8.55111754917089)
(82,8.54693544975204)
(83,8.66605947747419)
(84,8.49089417601578)
(85,8.48541554539908)
(86,8.44561206967175)
(87,8.58014556574557)
(88,8.40335231266141)
(89,8.3639219974089)
(90,8.30542294510122)
(91,8.53249537616582)
(92,8.29202808913906)
(93,8.25359007491097)
(94,8.22495187877562)
(95,8.2493220091081)
(96,8.31589617161871)
(97,8.19976279420752)
(98,8.18967260723272)
(99,8.37069393810889)
(100,8.23037470713979)
(101,8.17568861437345)
(102,8.16381875081452)
(103,8.13194423644109)
(104,8.40270307214338)
(105,8.10922711655525)
(106,8.1020911060756)
(107,8.27112654264333)
(108,8.11788920739675)
(109,8.04379136654635)
(110,8.01474013833664)
(111,8.22847077290569)
(112,7.99161827419456)
(113,7.98213107776018)
(114,7.9705887091828)
(115,7.9971843509347)
(116,7.97282647175694)
(117,7.93351580269397)
(118,7.90255155411873)
(119,7.99776642630242)
(120,7.91007457288172)
(121,7.88080319635567)
(122,7.87313798696741)
(123,7.84498867344205)
(124,7.79976321415661)
(125,7.98938146895335)
(126,7.60326661377574)
(127,7.52462148642087)
(128,7.34995983380932)
(129,8.10221113586057)
(130,6.9022590632367)
(131,6.80991180342276)
(132,6.88125458344041)
(133,6.8507836380378)
(134,6.71508640280255)
(135,6.70000126952806)
(136,6.85683699907552)
(137,6.55612655712418)
(138,6.52963963426299)
(139,6.97829713337909)
(140,6.89977526632402)
(141,6.35048115764401)
(142,6.26110580435994)
(143,7.07330692372099)
(144,6.31929255870998)
(145,6.18205067053115)
(146,6.16701844730448)
(147,6.11867830940032)
(148,6.13745017226778)
(149,6.09937758198822)
(150,6.09582597813946)
(151,6.06411366979733)
(152,6.14055772714741)
(153,5.94401023689714)
(154,5.89132811574154)
(155,5.68275758796475)
(156,6.12050945318889)
(157,5.51639726280943)
(158,5.49510016102023)
(159,5.44993976275907)
(160,5.58077524396)
(161,5.4347375308189)
(162,5.43206832958341)
(163,5.43955939273127)
(164,5.42309581140177)
(165,5.40496331872906)
(166,5.39776088705187)
(167,5.38504134703711)
(168,5.34783309143708)
(169,5.34382142546905)
(170,5.22681123881669)
(171,5.18440702402646)
(172,5.24431581083607)
(173,5.19540210255316)
(174,5.15062386574747)
(175,5.14416764148083)
(176,5.13341463369933)
(177,5.18173927919764)
(178,5.12141723919665)
(179,5.1184042877569)
(180,5.13109621710162)
(181,5.13363728292441)
(182,5.10312125837393)
(183,5.09785912201646)
(184,5.10735000469187)
(185,5.0926845635739)
(186,5.09034966393374)
(187,5.08821841949242)
(188,5.07488780550707)
(189,5.08849371417169)
(190,5.02222005421612)
(191,4.99207726277193)
(192,4.87914769868332)
(193,4.82925094001405)
(194,4.75376544240897)
(195,4.74473141754993)
(196,4.73009186342567)
(197,4.7490165043051)
(198,4.71971253985978)
(199,4.71714855523501)
(200,4.71978228094637)
(201,4.71455233520521)
(202,4.70787717391382)
(203,4.70435691995885)
(204,4.71018713940015)
(205,4.70020880519862)
(206,4.69615991183553)
(207,4.68871236789081)
(208,4.70198318498291)
(209,4.67512750689444)
(210,4.66603041063657)
(211,4.6415330874473)
(212,4.66782835952312)
(213,4.59443050387119)
(214,4.58660881032307)
(215,4.57713270754925)
(216,4.57506147736507)
(217,4.55372467737904)
(218,4.54519000896493)
(219,4.51435255105705)
(220,4.49560297960765)
(221,4.4774261009363)
(222,4.45191612991977)
(223,4.44720601916045)
(224,4.43599541033599)
(225,4.43276372123928)
(226,4.42934987534438)
(227,4.42128447486717)
(228,4.41809900599509)
(229,4.41547968748632)
(230,4.40155511455402)
(231,4.41165288105235)
(232,4.38775480844803)
(233,4.38489415754439)
(234,4.37688411667431)
(235,4.41441040680117)
(236,4.36767328848501)
(237,4.36149394650408)
(238,4.35441829353438)
(239,4.36849669898586)
(240,4.3509171724856)
(241,4.34845410776296)
(242,4.34440488118282)
(243,4.35165330633997)
(244,4.33990992103764)
(245,4.33719479734351)
(246,4.3270833745045)
(247,4.35093123292709)
(248,4.31867865163969)
(249,4.31601638116326)
(250,4.31314367797616)
(251,4.31774077386986)
(252,4.31128560370525)
(253,4.31075605785998)
(254,4.31211330242688)
(255,4.30829158320082)
(256,4.30473431044141)
(257,4.30077080190382)
(258,4.3062450713516)
(259,4.29844901192489)
(260,4.29534059217617)
(261,4.29341934045865)
(262,4.2889643717035)
(263,4.29711605951817)
(264,4.28401032948353)
(265,4.28275621895348)
(266,4.27964775582809)
(267,4.27891081839423)
(268,4.27600099347314)
(269,4.27332371487828)
(270,4.26542370206892)
(271,4.26547168294958)
(272,4.2518153012882)
(273,4.24504962575816)
(274,4.22596279759789)
(275,4.26679021875544)
(276,4.21859182459061)
(277,4.21680022623184)
(278,4.21345503474627)
(279,4.24742625597384)
(280,4.20294897654986)
(281,4.20046642199474)
(282,4.20866691822924)
(283,4.20151947744105)
(284,4.19322180171457)
(285,4.19143384888523)
(286,4.19590540956122)
(287,4.18858814000993)
(288,4.18689422656362)
(289,4.18527033142004)
(290,4.18359868520006)
(291,4.18386283880689)
(292,4.17901119799218)
(293,4.17660648867894)
(294,4.17388866246303)
(295,4.17437414476721)
(296,4.17052307165345)
(297,4.16896740987079)
(298,4.1657396188267)
(299,4.16207434351199)
(300,4.15562934480001)
(301,4.15221122706091)
(302,4.1468309024372)
(303,4.142731071332)
(304,4.13958253670866)
(305,4.13807034824092)
(306,4.13595295531159)
(307,4.1342975548661)
(308,4.13276508494234)
(309,4.12950517703989)
(310,4.12583870077403)
(311,4.12269500826494)
(312,4.12042283333458)
(313,4.1184613334651)
(314,4.11216151831005)
(315,4.10991196734199)
(316,4.10834813347944)
(317,4.10466833254876)
(318,4.10169796783637)
(319,4.09599684349197)
(320,4.09271550063422)
(321,4.08970990189555)
(322,4.0878053507012)
(323,4.08446831877605)
(324,4.08159081574492)
(325,4.08008004118005)
(326,4.07805189915379)
(327,4.07481683437214)
(328,4.07230668289582)
(329,4.07117704315002)
(330,4.06914148280182)
(331,4.06724141168374)
(332,4.06429414211129)
(333,4.06245476835401)
(334,4.05796516965501)
(335,4.05586999855709)
(336,4.05358214753478)
(337,4.05167372792326)
(338,4.04971536590856)
(339,4.04772865257353)
(340,4.0462682628228)
(341,4.0443337920353)
(342,4.04298815070817)
(343,4.04159669736941)
(344,4.03948635470198)
(345,4.03756551652368)
(346,4.03633060540019)
(347,4.03480070691241)
(348,4.03179842576502)
(349,4.02814764575733)
(350,4.02653745322011)
(351,4.02529318739869)
(352,4.02355851516367)
(353,4.02110622069766)
(354,4.01939315595809)
(355,4.01796252772672)
(356,4.01641766788388)
(357,4.01457265423333)
(358,4.01338887769868)
(359,4.01194794091037)
(360,4.0093231166358)
(361,4.00514939947506)
(362,4.00351546246402)
(363,4.00200907649201)
(364,3.99992281618718)
(365,3.99797430431843)
(366,3.99334918832357)
(367,3.99173253462353)
(368,3.98976171378918)
(369,3.9856036689915)
(370,3.98250512795621)
(371,3.97977832742336)
(372,3.97841061366499)
(373,3.97729992567629)
(374,3.97548532209751)
(375,3.97365457362757)
(376,3.9717840292761)
(377,3.97053456718478)
(378,3.96712437614579)
(379,3.96575290280055)
(380,3.96416468887183)
(381,3.96227331516023)
(382,3.96015123118512)
(383,3.95697348034623)
(384,3.95558559053402)
(385,3.95413029288639)
(386,3.95193285532975)
(387,3.94995248168693)
(388,3.94852118630789)
(389,3.94737903792242)
(390,3.94574874973164)
(391,3.94415680391129)
(392,3.94289188577825)
(393,3.94197297899114)
(394,3.93952283120682)
(395,3.93825160225002)
(396,3.93568543190763)
(397,3.93261556716918)
(398,3.930516268874)
(399,3.92949481126826)
(400,3.92769455749655)
(401,3.92597962780121)
(402,3.92405249564003)
(403,3.92298717514001)
(404,3.92024570123432)
(405,3.91798517635568)
(406,3.91605996843274)
(407,3.91484999273038)
(408,3.91348201245275)
(409,3.91205196729859)
(410,3.9101782087323)
(411,3.90907587383236)
(412,3.90769249170707)
(413,3.90573876369278)
(414,3.90393689253081)
(415,3.90248294606086)
(416,3.90162096816003)
(417,3.8997429477221)
(418,3.89848876411662)
(419,3.89595772299443)
(420,3.89340260315083)
(421,3.89169481233147)
(422,3.8905143792272)
(423,3.8893144775389)
(424,3.88810193082712)
(425,3.88648956528796)
(426,3.88543899169958)
(427,3.88434285366883)
(428,3.88275274532349)
(429,3.88074689719771)
(430,3.87955433051444)
(431,3.87854502418123)
(432,3.87741869136827)
(433,3.87615021891571)
(434,3.8748887963834)
(435,3.87401131226095)
(436,3.87269076174271)
(437,3.87130443426161)
(438,3.86889963678931)
(439,3.86766339312703)
(440,3.86591362701932)
(441,3.86435580437818)
(442,3.86263943159158)
(443,3.86160647111943)
(444,3.86069532203059)
(445,3.85921672643717)
(446,3.85780872044254)
(447,3.85689592512765)
(448,3.85587967890622)
(449,3.85403970348962)
(450,3.85177334918022)
(451,3.85069256255907)
(452,3.84914024782396)
(453,3.84711073784714)
(454,3.84586764734895)
(455,3.84488331013994)
(456,3.84374459931887)
(457,3.84258662457118)
(458,3.84159880585769)
(459,3.84071959478942)
(460,3.83956519914057)
(461,3.83839955250542)
(462,3.83752516217005)
(463,3.8364681708824)
(464,3.83475783495541)
(465,3.83331836750638)
(466,3.83238945048464)
(467,3.83153118390011)
(468,3.83030890070573)
(469,3.82888686851538)
(470,3.82797660550542)
(471,3.82700377061506)
(472,3.82597201441229)
(473,3.82484933367717)
(474,3.82411221757026)
(475,3.82314129973166)
(476,3.8215730346757)
(477,3.81989992430987)
(478,3.81901623535658)
(479,3.81795378904028)
(480,3.81632915808355)
(481,3.81533693137303)
(482,3.81423086285225)
(483,3.81319647591909)
(484,3.81191778283785)
(485,3.81107001334777)
(486,3.8103815294915)
(487,3.80953390229388)
(488,3.80827012532122)
(489,3.80702843221915)
(490,3.80635125361567)
(491,3.80547452706138)
(492,3.80395857022581)
(493,3.80304892442227)
(494,3.80228519530425)
(495,3.8015706361817)
(496,3.80077879827292)
(497,3.8001325670707)
(498,3.79948703152183)
(499,3.79881854489286)
(500,3.79775780431932)
(501,3.79703870764334)
(502,3.79633371289658)
(503,3.79528553833168)
(504,3.79428055004768)
(505,3.79323977232523)
(506,3.79254988970646)
(507,3.79153608118026)
(508,3.79082798087948)
(509,3.78994539411923)
(510,3.78894967876444)
(511,3.78788961864922)
(512,3.78719691851525)
(513,3.78666179387695)
(514,3.78596156857987)
(515,3.78520540040946)
(516,3.78448016559012)
(517,3.78388751805597)
(518,3.78314387190005)
(519,3.7824676279489)
(520,3.78195107164388)
(521,3.78125311905374)
(522,3.78046746083349)
(523,3.77957536512099)
(524,3.77892640810143)
(525,3.77824252607635)
(526,3.77767065902712)
(527,3.77707657851077)
(528,3.77649208597268)
(529,3.7759300063137)
(530,3.77538358825486)
(531,3.77485161860709)
(532,3.77430624311249)
(533,3.77375223887874)
(534,3.77321104279342)
(535,3.77273242062598)
(536,3.77219168528069)
(537,3.77170600779641)
(538,3.77128528941916)
(539,3.77089013070214)
(540,3.77024662166515)
(541,3.76977715929525)
(542,3.76931146679233)
(543,3.76880302275762)
(544,3.76809956172276)
(545,3.76750536192979)
(546,3.76710916414852)
(547,3.76659986048121)
(548,3.76617371502299)
(549,3.76577241485064)
(550,3.76535064864266)
(551,3.76481990493601)
(552,3.76441175083281)
(553,3.76410339463009)
(554,3.76368061822461)
(555,3.76321380149803)
(556,3.76288022592054)
(557,3.76256601896596)
(558,3.76216309488332)
(559,3.761795693057)
(560,3.76150354223073)
(561,3.76112864438217)
(562,3.76064017573548)
(563,3.76022980586341)
(564,3.75993451694517)
(565,3.75968143668119)
(566,3.7594045224648)
(567,3.75908943171438)
(568,3.75878164547724)
(569,3.75854689793333)
(570,3.7582424906445)
(571,3.75793903165706)
(572,3.75767756236639)
(573,3.75744597079137)
(574,3.75714153646432)
(575,3.75688711206787)
(576,3.75667015566394)
(577,3.75645647906458)
(578,3.75610494539715)
(579,3.75582014579273)
(580,3.75562943295601)
(581,3.75534553609542)
(582,3.75502417381085)
(583,3.75464369013743)
(584,3.75439444983922)
(585,3.75416681857923)
(586,3.75396285174997)
(587,3.75368322103135)
(588,3.7534706786618)
(589,3.75330504142941)
(590,3.75311566038413)
(591,3.75290362054593)
(592,3.75271969185744)
(593,3.75257129818678)
(594,3.75241264942064)
(595,3.7522280583499)
(596,3.75205469988612)
(597,3.75190819575032)
(598,3.7517626590235)
(599,3.7516058006531)
(600,3.75146560260072)
(601,3.75133494272921)
(602,3.75119503534922)
(603,3.75102037723042)
(604,3.75082668597587)
(605,3.75068676481354)
(606,3.75056747051451)
(607,3.75041781838715)
(608,3.75027463797489)
(609,3.75008712732238)
(610,3.74995732851354)
(611,3.74982178476451)
(612,3.74968371694349)
(613,3.7495285936222)
(614,3.74941296006289)
(615,3.74929911603172)
(616,3.74913726150537)
(617,3.74897805103517)
(618,3.74885705384141)
(619,3.74874963354451)
(620,3.74859161775803)
(621,3.74844889100133)
(622,3.74834278992655)
(623,3.74823883915335)
(624,3.74811365418672)
(625,3.74797828225327)
(626,3.74788246654743)
(627,3.74779959410751)
(628,3.74765022344617)
(629,3.74752767027)
(630,3.74741652128474)
(631,3.74732886333778)
(632,3.74721212443547)
(633,3.74711469319822)
(634,3.74702560643637)
(635,3.74692050358538)
(636,3.74683039793826)
(637,3.7467338058367)
(638,3.74663542230099)
(639,3.74651793374796)
(640,3.74642168315138)
(641,3.74636371576277)
(642,3.74624797803266)
(643,3.74614931110104)
(644,3.74606526901899)
(645,3.7460019867554)
(646,3.74593152978952)
(647,3.74584316102125)
(648,3.74574466453072)
(649,3.74566148777959)
(650,3.7456016893494)
(651,3.74553649377154)
(652,3.74546561773083)
(653,3.745381447244)
(654,3.74532102007605)
(655,3.74526159188137)
(656,3.7451950142971)
(657,3.74513328326609)
(658,3.74506771887444)
(659,3.74499347811469)
(660,3.74490380725585)
(661,3.74481710489874)
(662,3.74475316373512)
(663,3.74468106911017)
(664,3.74459336059384)
(665,3.74449975548279)
(666,3.74443718946213)
(667,3.7443756302593)
(668,3.74430307640286)
(669,3.74423234203767)
(670,3.74416515297006)
(671,3.74411518134775)
(672,3.74406317959127)
(673,3.74400852160073)
(674,3.74395010468026)
(675,3.74389852176432)
(676,3.74384505517438)
(677,3.74378165510622)
(678,3.74373536998093)
(679,3.74366747761197)
(680,3.74357277562586)
(681,3.74351617988416)
(682,3.74345821667858)
(683,3.74340965962514)
(684,3.7433665935811)
(685,3.74333013259509)
(686,3.74328028761075)
(687,3.74322247988481)
(688,3.74315135586653)
(689,3.74310666272858)
(690,3.7430634235613)
(691,3.74302093301781)
(692,3.74294178782519)
(693,3.74288642875492)
(694,3.74282451781943)
(695,3.74275586416584)
(696,3.74270062504022)
(697,3.74265698063228)
(698,3.74261078493935)
(699,3.74254825631828)
(700,3.74250476766571)
(701,3.74246859452954)
(702,3.74242291855706)
(703,3.74236374666149)
(704,3.74231110855055)
(705,3.74227756384092)
(706,3.74224207815006)
(707,3.74218943767502)
(708,3.74214330509948)
(709,3.74210817845599)
(710,3.7420717315306)
(711,3.74203152047682)
(712,3.7419903458656)
(713,3.74196060755244)
(714,3.74193384932532)
(715,3.74190839598638)
(716,3.74187825903688)
(717,3.74185410022818)
(718,3.74183194920258)
(719,3.74181160304062)
(720,3.74177874540215)
(721,3.74175473319888)
(722,3.74173262946911)
(723,3.74171284497681)
(724,3.74167804949778)
(725,3.7416411984487)
(726,3.74161090075658)
(727,3.7415812340218)
(728,3.74155611539142)
(729,3.74153487805987)
(730,3.74151339091035)
(731,3.74148753669067)
(732,3.74145328680621)
(733,3.74143276481216)
(734,3.74141322799996)
(735,3.74138671229007)
(736,3.74136439376675)
(737,3.74133941838526)
(738,3.74132234969917)
(739,3.74129485349071)
(740,3.74125184747336)
(741,3.74122666865854)
(742,3.74120262520594)
(743,3.74118042382181)
(744,3.74115458881404)
(745,3.7411327782176)
(746,3.74111854324206)
(747,3.74109251346467)
(748,3.74105890005783)
(749,3.7410259918781)
(750,3.74100336907399)
(751,3.74097584133875)
(752,3.74094155610841)
(753,3.74090984534707)
(754,3.74087833307684)
(755,3.74085294782907)
(756,3.74082857196165)
(757,3.74080397463984)
(758,3.74078198829117)
(759,3.74076020954738)
(760,3.74073467546796)
(761,3.74070650788942)
(762,3.74068586073433)
(763,3.74066855894915)
(764,3.74063928975958)
(765,3.74061364899738)
(766,3.74058760245367)
(767,3.7405671769452)
(768,3.74054176190312)
(769,3.74050975800182)
(770,3.74048711998747)
(771,3.74046532384755)
(772,3.74044546596015)
(773,3.74042513698655)
(774,3.7404080251903)
(775,3.74038633082624)
(776,3.74035675540909)
(777,3.7403331601501)
(778,3.74031439754291)
(779,3.74030036227755)
(780,3.74028473889477)
(781,3.74026089427677)
(782,3.74023698615883)
(783,3.74021840023201)
(784,3.74020271461056)
(785,3.7401793778023)
(786,3.74016176553833)
(787,3.74014325405571)
(788,3.74012839094366)
(789,3.74009756766588)
(790,3.74007122276407)
(791,3.7400540621925)
(792,3.74002753195669)
(793,3.73999969915007)
(794,3.73997945695549)
(795,3.73994979401525)
(796,3.73992703627021)
(797,3.73989866261752)
(798,3.73987908658785)
(799,3.73985613061492)
(800,3.73982824881314)
(801,3.73980248095223)
(802,3.73978208868868)
(803,3.73976522792638)
(804,3.73974663348216)
(805,3.73971735955802)
(806,3.73969965222043)
(807,3.73968223291466)
(808,3.73966165319271)
(809,3.73963675374451)
(810,3.73960934498171)
(811,3.73958863223683)
(812,3.73957074319721)
(813,3.73954618206766)
(814,3.73952497267199)
(815,3.73950592080853)
(816,3.73949214114282)
(817,3.73946824188981)
(818,3.73945144303768)
(819,3.73943386255338)
(820,3.73941490479722)
(821,3.73939545820391)
(822,3.73938053256158)
(823,3.73937161119739)
(824,3.73935227957639)
(825,3.73933362689217)
(826,3.73931087370792)
(827,3.73929602599376)
(828,3.73928521370935)
(829,3.73927561194478)
(830,3.73925375610562)
(831,3.73923790244632)
(832,3.73921915649662)
(833,3.73920713111434)
(834,3.73919271970968)
(835,3.73917258664557)
(836,3.73915761705733)
(837,3.73914421165319)
(838,3.73912728617475)
(839,3.7391117099442)
(840,3.73909365433055)
(841,3.73908051899921)
(842,3.73906963263531)
(843,3.73905969367928)
(844,3.73904879814693)
(845,3.73903931356062)
(846,3.73902992794148)
(847,3.73902245626219)
(848,3.73900672817484)
(849,3.73899821816021)
(850,3.73899063842558)
(851,3.73897714666595)
(852,3.73896617338543)
(853,3.73895361795179)
(854,3.73894554107593)
(855,3.73893898995414)
(856,3.73893158144997)
(857,3.73892201198589)
(858,3.73891386475841)
(859,3.73890782718117)
(860,3.73889991708818)
(861,3.73888815730623)
(862,3.73887837420678)
(863,3.73887134562543)
(864,3.73886014950315)
(865,3.73884846564701)
(866,3.73883480288308)
(867,3.73882386677594)
(868,3.73881596466138)
(869,3.73880813280328)
(870,3.73879913030641)
(871,3.73878879097343)
(872,3.73878166151021)
(873,3.73877555725501)
(874,3.73876650793524)
(875,3.73875862161415)
(876,3.73874917447512)
(877,3.73874158357103)
(878,3.73873058063019)
(879,3.73871983855582)
(880,3.73871251425391)
(881,3.73870614096508)
(882,3.73869867190344)
(883,3.73868898991226)
(884,3.73867975716229)
(885,3.73867133850102)
(886,3.73866461213373)
(887,3.73865815833177)
(888,3.7386487678763)
(889,3.7386390407896)
(890,3.73863069719786)
(891,3.73862275702807)
(892,3.73861689515872)
(893,3.73860972166786)
(894,3.73860349212175)
(895,3.73859651278253)
(896,3.73858781156957)
(897,3.73857603134044)
(898,3.73856751685302)
(899,3.73856146825392)
(900,3.73855062289585)
(901,3.73853946713225)
(902,3.73852827268226)
(903,3.73852114922868)
(904,3.7385106950112)
(905,3.7385015530988)
(906,3.73849008031748)
(907,3.73848368620631)
(908,3.73847563582143)
(909,3.73846275305727)
(910,3.73845561172462)
(911,3.73845029539306)
(912,3.73844382979155)
(913,3.73843533101044)
(914,3.73842835063973)
(915,3.73842176275498)
(916,3.73841421536803)
(917,3.73840913013374)
(918,3.73840364767521)
(919,3.73839773318718)
(920,3.73839138701912)
(921,3.73838250108545)
(922,3.73837709539469)
(923,3.73837336189215)
(924,3.7383671965923)
(925,3.7383594792087)
(926,3.73835321547978)
(927,3.73834895139464)
(928,3.73834481829251)
(929,3.73833467633875)
(930,3.73832856209702)
(931,3.738321173295)
(932,3.73831405727421)
(933,3.73830857762767)
(934,3.73830193022184)
(935,3.73829507228262)
(936,3.73828676375071)
(937,3.73827921075028)
(938,3.73827418307051)
(939,3.73826934651238)
(940,3.73826370417339)
(941,3.73825629708589)
(942,3.73825098259168)
(943,3.73824596705064)
(944,3.7382381604372)
(945,3.738230842765)
(946,3.73822118935351)
(947,3.73821669701143)
(948,3.73820926073113)
(949,3.73819967417343)
(950,3.73819419092195)
(951,3.73818869475105)
(952,3.73818245241404)
(953,3.73817512212339)
(954,3.7381709160093)
(955,3.73816662397702)
(956,3.73816128898538)
(957,3.73815424012405)
(958,3.73814809474663)
(959,3.73814281970791)
(960,3.73813752691553)
(961,3.73813251426639)
(962,3.73812799115518)
(963,3.73812396132077)
(964,3.7381201167569)
(965,3.73811627091309)
(966,3.73810898454529)
(967,3.73810351331011)
(968,3.73809833230564)
(969,3.73809518794971)
(970,3.73809085258716)
(971,3.73808603520853)
(972,3.73807896274891)
(973,3.73807504867394)
(974,3.73807214123749)
(975,3.73806886910099)
(976,3.73806343804419)
(977,3.73805503208483)
(978,3.7380516460052)
(979,3.73804782534891)
(980,3.7380396603955)
(981,3.73803615699193)
(982,3.73803226926532)
(983,3.73802582947157)
(984,3.73802131253667)
(985,3.73801669982391)
(986,3.73801380400576)
(987,3.73800835239561)
(988,3.73800480480318)
(989,3.73799885961324)
(990,3.7379934996031)
(991,3.73798884356009)
(992,3.7379844933136)
(993,3.73798174378352)
(994,3.73797867157368)
(995,3.73797552739654)
(996,3.73797222056717)
(997,3.73796928502145)
(998,3.73796712769717)
(999,3.73796244917576)
(1000,3.73795886872431)
(1001,3.73795621363599)
(1002,3.737954034773)
(1003,3.73795013823017)
(1004,3.73794612477552)
(1005,3.73794354535786)
(1006,3.73794083239536)
(1007,3.73793852511006)
(1008,3.737935448868)
(1009,3.73793350712922)
(1010,3.7379314343794)
(1011,3.73792833376877)
(1012,3.73792414724719)
(1013,3.7379213960211)
(1014,3.73791932485491)
(1015,3.73791595937145)
(1016,3.73791214134045)
(1017,3.73790895411229)
(1018,3.73790649336976)
(1019,3.73790355003503)
(1020,3.73790049644154)
(1021,3.73789698262809)
(1022,3.73789381835944)
(1023,3.7378914192543)
(1024,3.73788844934678)
(1025,3.73788528313874)
(1026,3.7378817625565)
(1027,3.73787951527683)
(1028,3.73787800821526)
(1029,3.7378755364729)
(1030,3.73787244123614)
(1031,3.73786726149042)
(1032,3.7378651395781)
(1033,3.73786396806278)
(1034,3.73785806935631)
(1035,3.73785479501547)
(1036,3.73785281238868)
(1037,3.73785143768922)
(1038,3.73784955728909)
(1039,3.73784702871634)
(1040,3.73784335822393)
(1041,3.73784121905668)
(1042,3.73783952987051)
(1043,3.73783795506459)
(1044,3.73783512317316)
(1045,3.73783285948799)
(1046,3.73783073466828)
(1047,3.73782942492894)
(1048,3.73782770275979)
(1049,3.73782488619734)
(1050,3.73782207031728)
(1051,3.73781978453762)
(1052,3.73781798687509)
(1053,3.73781642388851)
(1054,3.73781465127283)
(1055,3.73781275397751)
(1056,3.73781130792716)
(1057,3.73781005115785)
(1058,3.73780858859627)
(1059,3.73780712898731)
(1060,3.73780609251047)
(1061,3.73780483566953)
(1062,3.73780258184138)
(1063,3.73779905211149)
(1064,3.73779685887446)
(1065,3.73779520339098)
(1066,3.73779391386575)
(1067,3.73779113036419)
(1068,3.73778933045907)
(1069,3.73778745178796)
(1070,3.73778623957811)
(1071,3.73778398546632)
(1072,3.73778246691832)
(1073,3.73778116182345)
(1074,3.73777995239303)
(1075,3.73777847499299)
(1076,3.73777638132182)
(1077,3.7377749088876)
(1078,3.73777283638095)
(1079,3.73777168771525)
(1080,3.73777081150565)
(1081,3.73776954731008)
(1082,3.73776799836559)
(1083,3.73776647139632)
(1084,3.7377655680971)
(1085,3.73776479082134)
(1086,3.73776385868514)
(1087,3.73776256608389)
(1088,3.73776118522476)
(1089,3.73776047083588)
(1090,3.73775926305187)
(1091,3.73775756564635)
(1092,3.73775631655305)
(1093,3.73775557665796)
(1094,3.73775400191915)
(1095,3.73775253304987)
(1096,3.73775160343876)
(1097,3.73775012549399)
(1098,3.73774867466238)
(1099,3.73774759500421)
(1100,3.73774686411834)
(1101,3.73774557086208)
(1102,3.73774381309094)
(1103,3.73774264808728)
(1104,3.73774133697788)
(1105,3.73773984901162)
(1106,3.73773847562961)
(1107,3.73773681852295)
(1108,3.73773518315082)
(1109,3.73773366248561)
(1110,3.7377323002413)
(1111,3.73773133752454)
(1112,3.73773003308583)
(1113,3.73772899012056)
(1114,3.73772806028997)
(1115,3.73772714509731)
(1116,3.73772582516982)
(1117,3.7377239785843)
(1118,3.7377232073355)
(1119,3.73772260689154)
(1120,3.73771861559686)
(1121,3.73771756612217)
(1122,3.73771637638089)
(1123,3.73771459724721)
(1124,3.73771363254166)
(1125,3.7377127192486)
(1126,3.73771211989452)
(1127,3.73771166747312)
(1128,3.73771100165631)
(1129,3.73771029181356)
(1130,3.73770884534222)
(1131,3.73770729940469)
(1132,3.73770644766985)
(1133,3.73770575668162)
(1134,3.73770494172878)
(1135,3.7377042359973)
(1136,3.73770360993247)
(1137,3.73770295363523)
(1138,3.73770204710784)
(1139,3.73770135422735)
(1140,3.73770082556236)
(1141,3.7376998836284)
(1142,3.7376990462984)
(1143,3.73769736814266)
(1144,3.73769591967591)
(1145,3.73769495544059)
(1146,3.73769417038829)
(1147,3.73769313369961)
(1148,3.73769148771761)
(1149,3.73769058250868)
(1150,3.73768965190977)
(1151,3.73768872425276)
(1152,3.73768785213861)
(1153,3.73768725095523)
(1154,3.73768670628886)
(1155,3.73768577714168)
(1156,3.73768489425626)
(1157,3.73768425057952)
(1158,3.73768362555022)
(1159,3.73768317353421)
(1160,3.7376826910267)
(1161,3.73768235307943)
(1162,3.73768199255039)
(1163,3.73768155347612)
(1164,3.73768110438571)
(1165,3.73768057549603)
(1166,3.73768006870966)
(1167,3.73767955743052)
(1168,3.73767814601958)
(1169,3.73767778222665)
(1170,3.73767734418728)
(1171,3.73767656310651)
(1172,3.73767603210726)
(1173,3.73767549831176)
(1174,3.73767514912425)
(1175,3.73767474285055)
(1176,3.73767377017291)
(1177,3.73767327796882)
(1178,3.73767276312744)
(1179,3.73767234141855)
(1180,3.73767193233348)
(1181,3.73767161338212)
(1182,3.73767121949531)
(1183,3.7376706641179)
(1184,3.73767013673895)
(1185,3.73766981887848)
(1186,3.7376695427727)
(1187,3.73766922781491)
(1188,3.73766881454572)
(1189,3.73766854428889)
(1190,3.73766833448331)
(1191,3.73766804377534)
(1192,3.73766762207462)
(1193,3.73766679005467)
(1194,3.73766649799751)
(1195,3.7376663006698)
(1196,3.73766579409718)
(1197,3.73766476509563)
(1198,3.73766374819113)
(1199,3.73766342414792)
(1200,3.73766319860106)
(1201,3.73766267797265)
(1202,3.73766233471793)
(1203,3.73766202117655)
(1204,3.73766165819181)
(1205,3.73766126832734)
(1206,3.73766076484854)
(1207,3.73766053697472)
(1208,3.73766031068214)
(1209,3.7376596692715)
(1210,3.73765924429688)
(1211,3.73765891053244)
(1212,3.73765867411844)
(1213,3.73765829702792)
(1214,3.73765759472138)
(1215,3.73765730627193)
(1216,3.73765686452479)
(1217,3.73765604469132)
(1218,3.73765572450469)
(1219,3.73765544041223)
(1220,3.73765463450631)
(1221,3.73765429708738)
(1222,3.73765407432424)
(1223,3.73765346398549)
(1224,3.73765308612734)
(1225,3.73765233568442)
(1226,3.737651763711)
(1227,3.73765138628321)
(1228,3.7376511133877)
(1229,3.7376507819095)
(1230,3.73765036837937)
(1231,3.73764977120418)
(1232,3.73764949080883)
(1233,3.73764926638424)
(1234,3.73764882012534)
(1235,3.73764845540375)
(1236,3.73764819680104)
(1237,3.73764786190632)
(1238,3.73764741058131)
(1239,3.73764695812387)
(1240,3.73764668611141)
(1241,3.73764644131247)
(1242,3.73764572222343)
(1243,3.73764534939324)
(1244,3.73764480703855)
(1245,3.73764440211104)
(1246,3.7376437866241)
(1247,3.73764348988926)
(1248,3.73764320165607)
(1249,3.73764289850436)
(1250,3.737642472943)
(1251,3.73764180007472)
(1252,3.7376412096376)
(1253,3.73764075599067)
(1254,3.73764041359348)
(1255,3.73764018816783)
(1256,3.73763966251974)
(1257,3.73763908824867)
(1258,3.73763877938582)
(1259,3.73763837794714)
(1260,3.73763789855471)
(1261,3.73763754906325)
(1262,3.73763730365613)
(1263,3.73763685658245)
(1264,3.73763643312884)
(1265,3.73763614413284)
(1266,3.7376359232286)
(1267,3.7376353717435)
(1268,3.73763477923102)
(1269,3.73763443376275)
(1270,3.73763382788124)
(1271,3.73763337114625)
(1272,3.73763285506992)
(1273,3.73763250102621)
(1274,3.73763209665208)
(1275,3.7376318347477)
(1276,3.73763155821631)
(1277,3.73763119144495)
(1278,3.7376308504812)
(1279,3.73763047535611)
(1280,3.7376302488955)
(1281,3.73763005418233)
(1282,3.73762973302334)
(1283,3.73762942843885)
(1284,3.73762893171723)
(1285,3.73762855153197)
(1286,3.73762808575924)
(1287,3.73762780320721)
(1288,3.73762744753946)
(1289,3.73762710459974)
(1290,3.73762683573962)
(1291,3.73762657828504)
(1292,3.73762642643795)
(1293,3.73762617329268)
(1294,3.73762601050756)
(1295,3.73762588957408)
(1296,3.73762562122054)
(1297,3.73762523073697)
(1298,3.73762508665577)
(1299,3.73762489891204)
(1300,3.7376230974625)
(1301,3.73762267012478)
(1302,3.73762224143321)
(1303,3.73762149729731)
(1304,3.73762114889393)
(1305,3.7376206416011)
(1306,3.73762044655844)
(1307,3.73762023830673)
(1308,3.73761970486658)
(1309,3.7376194455041)
(1310,3.73761897529248)
(1311,3.73761864373451)
(1312,3.73761842198267)
(1313,3.73761827908011)
(1314,3.73761812939962)
(1315,3.73761796319856)
(1316,3.73761775045959)
(1317,3.73761747039154)
(1318,3.73761731116085)
(1319,3.73761701036397)
(1320,3.73761684238523)
(1321,3.73761671439021)
(1322,3.73761654255161)
(1323,3.73761635447267)
(1324,3.73761590614413)
(1325,3.73761566735629)
(1326,3.7376154400938)
(1327,3.73761529731819)
(1328,3.73761492989069)
(1329,3.73761468866737)
(1330,3.73761440300676)
(1331,3.73761424104455)
(1332,3.73761408614492)
(1333,3.73761390943591)
(1334,3.73761362922652)
(1335,3.73761342484814)
(1336,3.73761324022176)
(1337,3.73761312459037)
(1338,3.73761289900038)
(1339,3.73761271531347)
(1340,3.7376123032293)
(1341,3.73761209866479)
(1342,3.73761192263487)
(1343,3.73761170424469)
(1344,3.73761150600832)
(1345,3.73761131762441)
(1346,3.73761119345184)
(1347,3.73761108882423)
(1348,3.73761097487601)
(1349,3.73761084588636)
(1350,3.73761072363959)
(1351,3.73761064215836)
(1352,3.73761053209004)
(1353,3.73761044130857)
(1354,3.73761029183933)
(1355,3.73761015247928)
(1356,3.73760996685499)
(1357,3.73760988775832)
(1358,3.73760978738946)
(1359,3.7376090020316)
(1360,3.7376089035519)
(1361,3.73760881206537)
(1362,3.73760861602219)
(1363,3.73760842656771)
(1364,3.73760832177562)
(1365,3.73760818150878)
(1366,3.73760809890135)
(1367,3.73760797151546)
(1368,3.73760787161063)
(1369,3.73760756961736)
(1370,3.73760747116711)
(1371,3.73760736011604)
(1372,3.73760723202149)
(1373,3.73760712947204)
(1374,3.73760701600486)
(1375,3.73760687731659)
(1376,3.73760679008259)
(1377,3.73760670106301)
(1378,3.73760663864601)
(1379,3.7376065350028)
(1380,3.73760642026817)
(1381,3.73760632415419)
(1382,3.73760623833773)
(1383,3.73760618753237)
(1384,3.73760610578057)
(1385,3.73760604648956)
(1386,3.73760597689513)
(1387,3.73760586593274)
(1388,3.73760575484324)
(1389,3.73760546334045)
(1390,3.73760510095436)
(1391,3.737604956683)
(1392,3.73760490445019)
(1393,3.73760468410573)
(1394,3.73760444259203)
(1395,3.73760430410201)
(1396,3.73760409273969)
(1397,3.73760397099405)
(1398,3.73760382561482)
(1399,3.73760373220187)
(1400,3.73760364059813)
(1401,3.73760355782486)
(1402,3.73760348082942)
(1403,3.73760340073912)
(1404,3.73760332667934)
(1405,3.73760324347233)
(1406,3.73760318147964)
(1407,3.73760307996749)
(1408,3.7376029898135)
(1409,3.73760278335389)
(1410,3.73760270823199)
(1411,3.73760264710096)
(1412,3.73760242922555)
(1413,3.73760193988499)
(1414,3.73760143838954)
(1415,3.73760109213916)
(1416,3.73760098132394)
(1417,3.7376005070204)
(1418,3.73760038565941)
(1419,3.73760025857145)
(1420,3.73760010073035)
(1421,3.73759999377192)
(1422,3.73759984564174)
(1423,3.73759977410477)
(1424,3.73759969744466)
(1425,3.7375995128137)
(1426,3.73759942087224)
(1427,3.73759926648255)
(1428,3.73759900859604)
(1429,3.73759894122858)
(1430,3.73759889091598)
(1431,3.73759878592945)
(1432,3.73759870726108)
(1433,3.73759860482734)
(1434,3.7375985560469)
(1435,3.73759848994877)
(1436,3.73759836673548)
(1437,3.73759830340948)
(1438,3.73759826165397)
(1439,3.73759821807697)
(1440,3.73759812914216)
(1441,3.73759806283282)
(1442,3.73759802330871)
(1443,3.73759799204713)
(1444,3.73759790880162)
(1445,3.73759785022702)
(1446,3.7375975859094)
(1447,3.73759756369907)
(1448,3.73759751055027)
(1449,3.73759749673014)
(1450,3.73759742949714)
(1451,3.73759731236853)
(1452,3.73759727864989)
(1453,3.73759721879489)
(1454,3.73759715662844)
(1455,3.7375971194844)
(1456,3.73759709491096)
(1457,3.73759706867463)
(1458,3.7375970320241)
(1459,3.73759698421843)
(1460,3.73759696493256)
(1461,3.73759694817078)
(1462,3.73759692153828)
(1463,3.73759688118051)
(1464,3.73759686435211)
(1465,3.73759682020134)
(1466,3.73759674792578)
(1467,3.73759671372138)
(1468,3.73759670557411)
(1469,3.73759666530939)
(1470,3.73759658906249)
(1471,3.73759653305572)
(1472,3.73759648627582)
(1473,3.73759645098238)
(1474,3.73759641837893)
(1475,3.73759639873333)
(1476,3.73759636800384)
(1477,3.73759634768158)
(1478,3.73759632900134)
(1479,3.73759631346046)
(1480,3.73759629297124)
(1481,3.73759627519922)
(1482,3.73759626390486)
(1483,3.73759624653445)
(1484,3.73759623070539)
(1485,3.73759620949157)
(1486,3.73759618572849)
(1487,3.73759615275141)
(1488,3.7375961137877)
(1489,3.73759608989624)
(1490,3.7375960715564)
(1491,3.73759600599166)
(1492,3.73759596763902)
(1493,3.7375959020056)
(1494,3.73759587258152)
(1495,3.73759581548826)
(1496,3.73759578491382)
(1497,3.73759575465653)
(1498,3.73759573273887)
(1499,3.73759571318224)
(1500,3.73759569690808)
(1501,3.73759567680252)
(1502,3.73759564875242)
(1503,3.73759559550243)
(1504,3.737595575628)
(1505,3.73759554620513)
(1506,3.73759548423987)
(1507,3.73759545080701)
(1508,3.73759543956058)
(1509,3.73759535200989)
(1510,3.73759531904693)
(1511,3.73759516601144)
(1512,3.73759501692505)
(1513,3.73759497910142)
(1514,3.73759494664961)
(1515,3.73759488374651)
(1516,3.73759485856696)
(1517,3.73759483267793)
(1518,3.73759478656688)
(1519,3.73759474630067)
(1520,3.73759471942792)
(1521,3.73759469760238)
(1522,3.73759467574851)
(1523,3.73759465371966)
(1524,3.73759463090968)
(1525,3.73759461416489)
(1526,3.73759458981272)
(1527,3.73759457100742)
(1528,3.73759455188972)
(1529,3.73759452211803)
(1530,3.73759448909325)
(1531,3.73759440265633)
(1532,3.73759435980884)
(1533,3.73759431673363)
(1534,3.73759429823854)
(1535,3.73759424922084)
(1536,3.73759421141819)
(1537,3.73759414264468)
(1538,3.73759411267539)
(1539,3.73759407811473)
(1540,3.73759405336316)
(1541,3.73759401424284)
(1542,3.73759398973353)
(1543,3.73759396835957)
(1544,3.73759391325268)
(1545,3.73759388295736)
(1546,3.73759385693513)
(1547,3.73759383687114)
(1548,3.73759381003253)
(1549,3.73759378730555)
(1550,3.7375937564466)
(1551,3.73759370104203)
(1552,3.73759367512416)
(1553,3.73759365622307)
(1554,3.73759362562735)
(1555,3.73759358626582)
(1556,3.73759347653489)
(1557,3.73759345024392)
(1558,3.73759342511212)
(1559,3.73759341132892)
(1560,3.73759335898225)
(1561,3.73759313473795)
(1562,3.73759305764427)
(1563,3.73759293198748)
(1564,3.73759289553699)
(1565,3.7375928579892)
(1566,3.73759282784336)
(1567,3.73759280045312)
(1568,3.73759276790708)
(1569,3.73759274988997)
(1570,3.73759273290781)
(1571,3.73759268669778)
(1572,3.73759266738146)
(1573,3.73759263290324)
(1574,3.73759255886586)
(1575,3.73759251383096)
(1576,3.7375925070418)
(1577,3.73759242893098)
(1578,3.7375923742678)
(1579,3.73759233939731)
(1580,3.73759229802056)
(1581,3.73759223710942)
(1582,3.73759218883031)
(1583,3.73759217046377)
(1584,3.73759215824898)
(1585,3.73759214050205)
(1586,3.73759212422926)
(1587,3.73759210838957)
(1588,3.7375920977849)
(1589,3.7375920806783)
(1590,3.73759206509513)
(1591,3.7375920512522)
(1592,3.7375920427595)
(1593,3.73759203566467)
(1594,3.73759201164258)
(1595,3.73759200326938)
(1596,3.73759199855686)
(1597,3.73759198496145)
(1598,3.73759193719825)
(1599,3.73759191104112)
(1600,3.73759188761455)
(1601,3.73759187757655)
(1602,3.73759184500687)
(1603,3.73759182561239)
(1604,3.73759181019494)
(1605,3.73759180430032)
(1606,3.73759177453126)
(1607,3.73759175400277)
(1608,3.73759174229279)
(1609,3.73759173315729)
(1610,3.73759172253214)
(1611,3.73759171082768)
(1612,3.73759169446135)
(1613,3.73759168454076)
(1614,3.73759165435977)
(1615,3.73759164035126)
(1616,3.73759159928345)
(1617,3.73759159291005)
(1618,3.73759158465185)
(1619,3.73759158192753)
(1620,3.73759156969715)
(1621,3.73759151600569)
(1622,3.73759148988752)
(1623,3.7375914706473)
(1624,3.73759145927664)
(1625,3.73759144218356)
(1626,3.73759143116457)
(1627,3.73759142435886)
(1628,3.73759141310619)
(1629,3.73759140664004)
(1630,3.73759140123351)
(1631,3.73759139867231)
(1632,3.73759139030815)
(1633,3.73759138701396)
(1634,3.73759138347821)
(1635,3.73759133868459)
(1636,3.73759133254748)
(1637,3.73759132614577)
(1638,3.73759132348959)
(1639,3.73759131674587)
(1640,3.73759131416952)
(1641,3.73759131154659)
(1642,3.73759130977906)
(1643,3.73759130692713)
(1644,3.73759130446745)
(1645,3.73759130232601)
(1646,3.73759130043594)
(1647,3.73759129731258)
(1648,3.73759129422323)
(1649,3.73759129093343)
(1650,3.73759128750128)
(1651,3.73759128103571)
(1652,3.73759127148391)
(1653,3.73759126692418)
(1654,3.73759126367369)
(1655,3.73759123957256)
(1656,3.73759121715229)
(1657,3.73759121362837)
(1658,3.73759120500385)
(1659,3.73759118534987)
(1660,3.73759117905565)
(1661,3.7375911715676)
(1662,3.73759114205985)
(1663,3.7375911296927)
(1664,3.73759112242247)
(1665,3.73759109324951)
(1666,3.7375910839578)
(1667,3.73759106826165)
(1668,3.73759104985113)
(1669,3.73759103751238)
(1670,3.73759102897253)
(1671,3.73759101654705)
(1672,3.73759100585387)
(1673,3.73759098455513)
(1674,3.73759097121983)
(1675,3.73759093213954)
(1676,3.73759091525101)
(1677,3.73759089829387)
(1678,3.73759086096342)
(1679,3.73759082893052)
(1680,3.737590796788)
(1681,3.73759077442082)
(1682,3.73759075586395)
(1683,3.73759072643293)
(1684,3.73759060919327)
(1685,3.73759053295097)
(1686,3.73759049330429)
(1687,3.73759045862152)
(1688,3.73759041816716)
(1689,3.73759038411041)
(1690,3.73759031493546)
(1691,3.73759026531969)
(1692,3.73759023236924)
(1693,3.73759021148846)
(1694,3.73759018624055)
(1695,3.73759016704711)
(1696,3.73759015163669)
(1697,3.73759013078604)
(1698,3.73759010222169)
(1699,3.73759004772387)
(1700,3.73759002866821)
(1701,3.73759001495885)
(1702,3.73758998552235)
(1703,3.73758990908409)
(1704,3.73758981090818)
(1705,3.7375897391046)
(1706,3.73758971982188)
(1707,3.73758958869862)
(1708,3.73758956483426)
(1709,3.73758952139986)
(1710,3.73758949144691)
(1711,3.7375894175407)
(1712,3.73758937572804)
(1713,3.73758930651014)
(1714,3.73758927962953)
(1715,3.73758924224832)
(1716,3.73758917367011)
(1717,3.73758914582746)
(1718,3.73758913643835)
(1719,3.73758907126005)
(1720,3.73758905054521)
(1721,3.73758900816774)
(1722,3.73758898061566)
(1723,3.73758896051808)
(1724,3.73758894322232)
(1725,3.73758891995705)
(1726,3.7375888949516)
(1727,3.7375888645591)
(1728,3.73758884634554)
(1729,3.73758881829787)
(1730,3.73758880539018)
(1731,3.73758879586456)
(1732,3.73758872655574)
(1733,3.73758870953495)
(1734,3.73758869697307)
(1735,3.73758869116605)
(1736,3.73758863969246)
(1737,3.73758860853091)
(1738,3.73758858457416)
(1739,3.73758856546749)
(1740,3.737588538409)
(1741,3.73758850959037)
(1742,3.7375884882394)
(1743,3.73758846185977)
(1744,3.73758843680878)
(1745,3.73758840779896)
(1746,3.73758839137689)
(1747,3.73758838196593)
(1748,3.73758833144478)
(1749,3.73758831282487)
(1750,3.7375882950344)
(1751,3.73758828347222)
(1752,3.73758826764018)
(1753,3.73758825533004)
(1754,3.73758823443736)
(1755,3.73758822491147)
(1756,3.73758821606131)
(1757,3.73758819846738)
(1758,3.73758818643535)
(1759,3.73758816046739)
(1760,3.73758815304434)
(1761,3.73758814697514)
(1762,3.73758813482113)
(1763,3.73758810359343)
(1764,3.73758806042036)
(1765,3.73758803428063)
(1766,3.73758802804734)
(1767,3.73758799420251)
(1768,3.73758798110165)
(1769,3.73758795410743)
(1770,3.73758794263395)
(1771,3.73758793133417)
(1772,3.73758791458983)
(1773,3.73758790058685)
(1774,3.73758788234625)
(1775,3.73758787314109)
(1776,3.73758786330207)
(1777,3.73758785015541)
(1778,3.73758784004478)
(1779,3.73758783140052)
(1780,3.73758782532189)
(1781,3.7375878214215)
(1782,3.73758781596135)
(1783,3.73758780832482)
(1784,3.73758780068191)
(1785,3.73758779655553)
(1786,3.73758778946524)
(1787,3.73758777600372)
(1788,3.73758776880541)
(1789,3.73758776533992)
(1790,3.73758773800892)
(1791,3.73758772826122)
(1792,3.73758770965911)
(1793,3.73758769304365)
(1794,3.73758768226737)
(1795,3.73758767278667)
(1796,3.73758766509388)
(1797,3.73758765588906)
(1798,3.73758764419383)
(1799,3.73758762851168)
(1800,3.7375876186642)
(1801,3.73758761245765)
(1802,3.73758758441205)
(1803,3.73758756867851)
(1804,3.73758751219017)
(1805,3.73758750232962)
(1806,3.73758748829042)
(1807,3.73758747973202)
(1808,3.7375874585515)
(1809,3.7375873153233)
(1810,3.73758726747758)
(1811,3.73758723671357)
(1812,3.73758721933354)
(1813,3.73758719110762)
(1814,3.73758716037321)
(1815,3.73758714541441)
(1816,3.73758713521314)
(1817,3.73758711254284)
(1818,3.73758709908092)
(1819,3.73758705395136)
(1820,3.73758703909125)
(1821,3.73758702435067)

};
\addplot [thick, blue]
coordinates {
(0,23930.000884189)
(1,7180.45388604586)
(2,3617.86527996726)
(3,2206.80793392946)
(4,1491.00767628939)
(5,1073.99586354884)
(6,808.278854547473)
(7,628.014922400163)
(8,499.960644278589)
(9,405.752764676303)
(10,334.52634655712)
(11,279.477813847677)
(12,236.164491398145)
(13,201.577684383878)
(14,173.611856832551)
(15,150.754899712087)
(16,131.896367104543)
(17,116.20926011348)
(18,103.06516983272)
(19,91.9789779656275)
(20,82.5721972749523)
(21,74.5480919899652)
(22,67.6703468931341)
(23,61.7492516151206)
(24,56.6308080266971)
(25,52.1895417181108)
(26,48.3230738518772)
(27,44.9473811889776)
(28,41.9908958545994)
(29,39.3943307887263)
(30,37.1076648133911)
(31,35.0895582332656)
(32,33.3047098499916)
(33,31.7230458561848)
(34,30.3185369243055)
(35,29.0691392036425)
(36,27.9557168948839)
(37,26.9619473758925)
(38,26.0735819006833)
(39,25.2785051309949)
(40,24.5658953302441)
(41,23.9261996677162)
(42,23.3512128923731)
(43,22.8338555363931)
(44,22.3679803523897)
(45,21.9478508869475)
(46,21.5686369463874)
(47,21.2260325999356)
(48,20.9161983824383)
(49,20.6357394037178)
(50,20.3816052485759)
(51,20.1511846245288)
(52,19.942073913322)
(53,19.7522963849272)
(54,19.5798157329886)
(55,19.4229051369124)
(56,19.2800427060627)
(57,19.1499227239291)
(58,19.0313113044641)
(59,18.9230788312623)
(60,18.8242439624631)
(61,18.7339264319528)
(62,18.6513238709167)
(63,18.5757271356125)
(64,18.5064950547755)
(65,18.4430397931058)
(66,18.3848534199294)
(67,18.3314621000904)
(68,18.282410130483)
(69,18.2373078847479)
(70,18.195837809721)
(71,18.1576476003302)
(72,18.1224457540708)
(73,18.0899698163812)
(74,18.0599809133384)
(75,18.032269842075)
(76,18.0066439340179)
(77,17.9829130684296)
(78,17.9609160188493)
(79,17.9405020453336)
(80,17.9215358365279)
(81,17.9038985280102)
(82,17.8874801575507)
(83,17.8721768211419)
(84,17.8578894936326)
(85,17.8445328788038)
(86,17.83202928813)
(87,17.8203078087826)
(88,17.8093094635342)
(89,17.7989701384791)
(90,17.7892345518416)
(91,17.7800530702916)
(92,17.7713803468455)
(93,17.7631748927227)
(94,17.755398724103)
(95,17.7480170503084)
(96,17.7409979934438)
(97,17.7343123346806)
(98,17.7279332839925)
(99,17.7218362708545)
(100,17.7160071042362)

};
\addplot [thick, green!50.0!black]
coordinates {
(0,23930.000884189)
(1,12083.1529253535)
(2,1962.70076992746)
(3,950.485602345173)
(4,331.959633328347)
(5,243.914762566925)
(6,134.329835058351)
(7,101.908974098636)
(8,71.430490269532)
(9,53.8008057292489)
(10,49.2070909809805)
(11,47.2377989642264)
(12,45.8394992598531)
(13,44.9246042975355)
(14,44.7440629754329)
(15,44.39329787842)
(16,44.3236229723654)
(17,44.2184239820041)
(18,44.1931882049634)
(19,44.1267629597911)
(20,44.1051433432534)
(21,44.0736620641302)
(22,44.0227863049802)
(23,43.9604053846835)
(24,43.8383240409788)
(25,43.7012466950384)
(26,43.4900906898139)
(27,43.1528664627698)
(28,42.5004315629993)
(29,41.8013543791512)
(30,40.8296742339329)
(31,39.900023272227)
(32,38.5823534699433)
(33,37.4156430458995)
(34,36.2719668390337)
(35,35.0776601950381)
(36,34.1837687644875)
(37,33.2698645775237)
(38,32.1962015032741)
(39,31.3082549503913)
(40,30.4262826422634)
(41,29.5345560514281)
(42,28.6995811800974)
(43,27.8507440638526)
(44,26.8903832876716)
(45,26.0249848856949)
(46,25.0716863775733)
(47,24.1555764755733)
(48,23.4070450974266)
(49,22.7049953977644)
(50,22.0172886765464)
(51,21.3371046021488)
(52,20.6002987471974)
(53,19.9394587462096)
(54,19.3137752887323)
(55,18.6682287336718)
(56,18.1127667913341)
(57,17.5853401003318)
(58,16.9831567654981)
(59,16.4442553456098)
(60,15.9565899295854)
(61,15.4524799469826)
(62,15.0154266084085)
(63,14.6158710219892)
(64,14.1749219112577)
(65,13.8168371772112)
(66,13.4635649950696)
(67,13.0778586597113)
(68,12.766579104139)
(69,12.4211022836258)
(70,12.0819160127479)
(71,11.7910516256896)
(72,11.5003137939156)
(73,11.234474233497)
(74,10.9852905269443)
(75,10.7240520078585)
(76,10.4627083449499)
(77,10.2286926962082)
(78,9.95946000368242)
(79,9.74002365928325)
(80,9.51656034513117)
(81,9.29145988899816)
(82,9.07267351528514)
(83,8.88697268772216)
(84,8.66699360507882)
(85,8.48608469643221)
(86,8.31017432250414)
(87,8.12849250541886)
(88,7.95268498774693)
(89,7.80156034591783)
(90,7.63432584968156)
(91,7.4884815251341)
(92,7.34172281544927)
(93,7.2094978287168)
(94,7.08523071948241)
(95,6.97027976094049)
(96,6.85746063617941)
(97,6.74882343897467)
(98,6.6395369792259)
(99,6.54752507330449)
(100,6.451547775383)
(101,6.36162519097851)
(102,6.27636659871515)
(103,6.19576519855579)
(104,6.1105722950634)
(105,6.02718102167213)
(106,5.95082535982036)
(107,5.87578582284964)
(108,5.81233595151331)
(109,5.73816283625261)
(110,5.67001782292722)
(111,5.59877510876061)
(112,5.53334659609807)
(113,5.47110793085513)
(114,5.40991942506864)
(115,5.36059759085225)
(116,5.30003871138622)
(117,5.25644195605928)
(118,5.2079330088986)
(119,5.16682679077942)
(120,5.12543393447383)
(121,5.08741167787886)
(122,5.04874894827296)
(123,5.01259404523217)
(124,4.97309259872131)
(125,4.93509187871559)
(126,4.89996412221353)
(127,4.86620424754931)
(128,4.83259517430058)
(129,4.79999147089031)
(130,4.76493049905084)
(131,4.73325694446595)
(132,4.70416695570773)
(133,4.67888476619529)
(134,4.65262739639857)
(135,4.62762957717384)
(136,4.60369921903072)
(137,4.57990557443312)
(138,4.55729925905472)
(139,4.53696012173993)
(140,4.51517453524093)
(141,4.49495399856832)
(142,4.47542802021928)
(143,4.45406482650698)
(144,4.43656339391869)
(145,4.4160779300751)
(146,4.40306779014329)
(147,4.38606923868873)
(148,4.37190599538975)
(149,4.3577969376911)
(150,4.34318842794173)
(151,4.32919007359116)
(152,4.31158257764717)
(153,4.29677898650613)
(154,4.28305080552483)
(155,4.267086204204)
(156,4.2556660992037)
(157,4.23959397046955)
(158,4.22830194655125)
(159,4.21514694220386)
(160,4.205095004479)
(161,4.19570757464146)
(162,4.18438526562959)
(163,4.17710246770172)
(164,4.16764634202139)
(165,4.15979207046328)
(166,4.15082356546804)
(167,4.1413433964435)
(168,4.12834874500854)
(169,4.11980319580987)
(170,4.10692696875007)
(171,4.10011473718714)
(172,4.09123078620307)
(173,4.08648205789493)
(174,4.0787785079874)
(175,4.07174538454882)
(176,4.06528955168316)
(177,4.05784389039987)
(178,4.05216401723596)
(179,4.0450031924621)
(180,4.03969141173713)
(181,4.03158734139422)
(182,4.02736646948498)
(183,4.02067225174152)
(184,4.01607415543408)
(185,4.01021462980189)
(186,4.00613493575736)
(187,3.99983563711056)
(188,3.99648862233057)
(189,3.99125870742269)
(190,3.98846829193574)
(191,3.9831902487313)
(192,3.98007499293121)
(193,3.97524163934253)
(194,3.97215460976433)
(195,3.96632082396578)
(196,3.96254319211348)
(197,3.95708910777948)
(198,3.95383850180867)
(199,3.94852121439348)
(200,3.94568940290271)
(201,3.94072141291591)
(202,3.93782105410923)
(203,3.93305155506064)
(204,3.93019398139779)
(205,3.92532870600185)
(206,3.92209170875403)
(207,3.91810749726888)
(208,3.91534363714)
(209,3.91186221718249)
(210,3.90928785970875)
(211,3.90606044995817)
(212,3.90349266209369)
(213,3.89958691920946)
(214,3.897287214864)
(215,3.8936821450916)
(216,3.89139327046655)
(217,3.88793774047786)
(218,3.88557234615864)
(219,3.88228682452595)
(220,3.88051617709015)
(221,3.87729796257439)
(222,3.87513932736856)
(223,3.87216538408708)
(224,3.87028102692261)
(225,3.86773271296983)
(226,3.8662517928345)
(227,3.86408512950891)
(228,3.86259043297563)
(229,3.86035480014273)
(230,3.85884153265201)
(231,3.85705879546716)
(232,3.85539309545359)
(233,3.85410561380124)
(234,3.85242128398972)
(235,3.85102128013812)
(236,3.84945327587287)
(237,3.84781293402378)
(238,3.84604008183371)
(239,3.84417200140538)
(240,3.84274745653806)
(241,3.84098541886405)
(242,3.83971776306798)
(243,3.83798034973253)
(244,3.83683190988865)
(245,3.83548293350208)
(246,3.83424116842435)
(247,3.83283540727062)
(248,3.8313033218924)
(249,3.82981663297735)
(250,3.82848929462528)
(251,3.8267681263994)
(252,3.82556631007534)
(253,3.8237236519563)
(254,3.82266083539909)
(255,3.82121486467379)
(256,3.82027029191084)
(257,3.81914306693968)
(258,3.81788632471194)
(259,3.81681665579276)
(260,3.815442034354)
(261,3.81415800671505)
(262,3.81297447001371)
(263,3.81152340702821)
(264,3.81045569222702)
(265,3.80941642619398)
(266,3.80846482708664)
(267,3.80743060710025)
(268,3.80619376358734)
(269,3.80522889452713)
(270,3.80401851256108)
(271,3.80299889538528)
(272,3.80205030908258)
(273,3.80106889898641)
(274,3.79988412022831)
(275,3.7991917822155)
(276,3.79807124376999)
(277,3.797482715769)
(278,3.7965630211434)
(279,3.79595838279024)
(280,3.79510609114931)
(281,3.79446580203799)
(282,3.79370736410598)
(283,3.79306436028809)
(284,3.79237663383407)
(285,3.79176445214526)
(286,3.79109035607477)
(287,3.79035981049862)
(288,3.7898544928744)
(289,3.78909664395785)
(290,3.78852891958106)
(291,3.78780672972926)
(292,3.78705331214125)
(293,3.78632831068458)
(294,3.78561088257541)
(295,3.78494126319544)
(296,3.78424645159915)
(297,3.78362091545339)
(298,3.78279288500698)
(299,3.78225339809526)
(300,3.78155230857642)
(301,3.78100235288025)
(302,3.78031075503927)
(303,3.77981799763149)
(304,3.77917037382963)
(305,3.77870749903801)
(306,3.77819085472531)
(307,3.77761639759165)
(308,3.77715341323193)
(309,3.77664655414066)
(310,3.77623097367384)
(311,3.77576984061671)
(312,3.77536295173602)
(313,3.77503195579037)
(314,3.77465982150755)
(315,3.77434166336993)
(316,3.77396276713725)
(317,3.77363004159907)
(318,3.77316841264676)
(319,3.77276421897502)
(320,3.77225566919488)
(321,3.77189031199103)
(322,3.7714721403252)
(323,3.77109490607283)
(324,3.77077672848363)
(325,3.77040763452058)
(326,3.77014261566575)
(327,3.76981112057641)
(328,3.76950132945398)
(329,3.76917320470531)
(330,3.76883523362868)
(331,3.76848952063809)
(332,3.76815760320699)
(333,3.76782985213799)
(334,3.76754482814153)
(335,3.76729175462336)
(336,3.76702112772215)
(337,3.76676071816487)
(338,3.76650710166178)
(339,3.76624529610397)
(340,3.7660012369687)
(341,3.76576871156933)
(342,3.76548216567509)
(343,3.76525100190972)
(344,3.76494754504821)
(345,3.76469876108863)
(346,3.76442697240637)
(347,3.76416546542455)
(348,3.76393475400801)
(349,3.76363455908212)
(350,3.76337640316017)
(351,3.76311301911484)
(352,3.76284771453187)
(353,3.76253703911119)
(354,3.76232167142751)
(355,3.76202573480877)
(356,3.76185425270663)
(357,3.76156246706336)
(358,3.76138344332866)
(359,3.76117289703059)
(360,3.76100270837172)
(361,3.76082446550187)
(362,3.7606289573713)
(363,3.76044311543043)
(364,3.76024629428892)
(365,3.76006954475379)
(366,3.75985640443088)
(367,3.75969235902624)
(368,3.75951702274354)
(369,3.75931657626994)
(370,3.75910121730723)
(371,3.75891229381617)
(372,3.75867411566804)
(373,3.75849673663124)
(374,3.75820699197885)
(375,3.75804020440156)
(376,3.75779450624678)
(377,3.75761621409044)
(378,3.75742597013973)
(379,3.75727316308266)
(380,3.75707887403196)
(381,3.75691804333044)
(382,3.75671995029196)
(383,3.75657361317903)
(384,3.75636936444537)
(385,3.75623420136118)
(386,3.75605211770352)
(387,3.75591023229299)
(388,3.75572308588805)
(389,3.75559401401372)
(390,3.75539650343974)
(391,3.75527251214585)
(392,3.75510164488203)
(393,3.75496109680896)
(394,3.75481262853061)
(395,3.75469609001998)
(396,3.75454271565955)
(397,3.75440771834154)
(398,3.75426445318479)
(399,3.75411262043161)
(400,3.75399171515408)
(401,3.75384700299386)
(402,3.75373601510596)
(403,3.75358952992677)
(404,3.75346851370497)
(405,3.75331368838498)
(406,3.75321270157323)
(407,3.75304360830039)
(408,3.75293439432026)
(409,3.75276541695022)
(410,3.75262054753382)
(411,3.75247200037655)
(412,3.7523475081682)
(413,3.75219140568697)
(414,3.75206671474024)
(415,3.75189307979836)
(416,3.75173763675237)
(417,3.75156268024528)
(418,3.75142599690311)
(419,3.75125107517422)
(420,3.75111450674077)
(421,3.7509454269434)
(422,3.75081900849697)
(423,3.75064999403041)
(424,3.75052161893989)
(425,3.75036528789738)
(426,3.75026399910255)
(427,3.75010917698355)
(428,3.75001596172723)
(429,3.7498732375621)
(430,3.74977431046982)
(431,3.74964562735367)
(432,3.74954942303358)
(433,3.74944652403868)
(434,3.7493681752848)
(435,3.74927872296471)
(436,3.74919218837585)
(437,3.74911693308352)
(438,3.74904535603257)
(439,3.74898662304563)
(440,3.74891812214901)
(441,3.74886006547367)
(442,3.74878864997678)
(443,3.74873493000449)
(444,3.74866742227346)
(445,3.74860335566448)
(446,3.74853886909196)
(447,3.7484689239028)
(448,3.74839482936899)
(449,3.7483291956542)
(450,3.74825233183658)
(451,3.74818854703671)
(452,3.74811401122678)
(453,3.7480493651035)
(454,3.74797064112921)
(455,3.74790148235455)
(456,3.74782462089375)
(457,3.74773651887886)
(458,3.74766042018675)
(459,3.74757585211129)
(460,3.74750236376181)
(461,3.74743283260543)
(462,3.74736352734666)
(463,3.74729553947663)
(464,3.74722657349309)
(465,3.74717382868019)
(466,3.74710339329863)
(467,3.74704646640288)
(468,3.74697943582619)
(469,3.74691128760395)
(470,3.74685245681109)
(471,3.7467801417168)
(472,3.74673345968546)
(473,3.74665729852526)
(474,3.74659552995826)
(475,3.74652509912179)
(476,3.74645390176811)
(477,3.74637731978863)
(478,3.74631515250338)
(479,3.74622957037861)
(480,3.74616387803336)
(481,3.74608521368325)
(482,3.74603074191055)
(483,3.74595306429884)
(484,3.74589855954142)
(485,3.74583462599904)
(486,3.74577077734902)
(487,3.74571831393103)
(488,3.74565593313321)
(489,3.74560545139983)
(490,3.74554468118862)
(491,3.74548637616047)
(492,3.74542800526426)
(493,3.74536301874227)
(494,3.74530919324732)
(495,3.74524474401653)
(496,3.74519043103749)
(497,3.74513261719584)
(498,3.7450748078865)
(499,3.74501474774243)
(500,3.74495562664613)
(501,3.74489141848624)
(502,3.74482564836354)
(503,3.74476995007867)
(504,3.74470217498241)
(505,3.74466013798536)
(506,3.74459241735894)
(507,3.74454516156133)
(508,3.74449130315165)
(509,3.74443834806183)
(510,3.74438116607466)
(511,3.74432569129946)
(512,3.74427277852877)
(513,3.74421468938001)
(514,3.74416898588479)
(515,3.74411319810954)
(516,3.74407042018594)
(517,3.74401511214301)
(518,3.74396160607964)
(519,3.74390840544191)
(520,3.74384421006221)
(521,3.74378485738189)
(522,3.74371527275934)
(523,3.74364856496327)
(524,3.74359798408825)
(525,3.74352553396694)
(526,3.74347413806783)
(527,3.7434139073384)
(528,3.74336809436373)
(529,3.74331388671594)
(530,3.74326481803457)
(531,3.74321706082349)
(532,3.7431667639514)
(533,3.74311644970281)
(534,3.74306550277262)
(535,3.74301612338473)
(536,3.74296511114281)
(537,3.74292679709068)
(538,3.7428840961028)
(539,3.74284083603382)
(540,3.74279934013523)
(541,3.74275079549761)
(542,3.74271288375513)
(543,3.74265985488221)
(544,3.74262619436029)
(545,3.74257588637796)
(546,3.74253814252726)
(547,3.74249694296591)
(548,3.74246291698524)
(549,3.74242574050466)
(550,3.74239141277331)
(551,3.74235198117102)
(552,3.74231594388783)
(553,3.74228000186351)
(554,3.74225188910381)
(555,3.74222155872168)
(556,3.74219576824704)
(557,3.74216113099798)
(558,3.7421291097801)
(559,3.74209025127021)
(560,3.74205645012881)
(561,3.742021328813)
(562,3.74198671723298)
(563,3.74195160010508)
(564,3.74192422502758)
(565,3.74189243587585)
(566,3.74186892772694)
(567,3.74184104773447)
(568,3.74181717989176)
(569,3.74178758912634)
(570,3.74176170676605)
(571,3.74172994771925)
(572,3.7417043910203)
(573,3.74167522898495)
(574,3.74164707029444)
(575,3.74161954493233)
(576,3.74159400714088)
(577,3.74156332337963)
(578,3.74153752882663)
(579,3.74151144021312)
(580,3.74148733439032)
(581,3.74146244079137)
(582,3.74143818563236)
(583,3.74141818184761)
(584,3.74139000959428)
(585,3.74137234442168)
(586,3.7413478078677)
(587,3.74132927231714)
(588,3.74130673362227)
(589,3.74128907232733)
(590,3.74127362276867)
(591,3.74124914971593)
(592,3.74123446674069)
(593,3.74121632466596)
(594,3.74120328924757)
(595,3.74118909198467)
(596,3.74117680023017)
(597,3.74116386712557)
(598,3.74115020641502)
(599,3.74113609115005)
(600,3.74112102775107)
(601,3.74110749144362)
(602,3.74109314215094)
(603,3.74107917636756)
(604,3.74106530064851)
(605,3.74105188044314)
(606,3.74104058745549)
(607,3.74102722452002)
(608,3.74101564643921)
(609,3.7410008112848)
(610,3.74098842846916)
(611,3.74097421955018)
(612,3.74096139595054)
(613,3.74094836322696)
(614,3.74093659798061)
(615,3.74092290311171)
(616,3.74091219227729)
(617,3.74089814241077)
(618,3.7408861800869)
(619,3.74087294730084)
(620,3.7408597322999)
(621,3.74084563998964)
(622,3.74082985004121)
(623,3.74081335151383)
(624,3.74079551969156)
(625,3.74077958068779)
(626,3.74076175305565)
(627,3.74074628545173)
(628,3.74072490354971)
(629,3.74070646601223)
(630,3.74068413792834)
(631,3.74066539847874)
(632,3.74064288016811)
(633,3.74062455595132)
(634,3.74060225253168)
(635,3.74058485230222)
(636,3.74056720432599)
(637,3.74055080307954)
(638,3.7405355904744)
(639,3.74051831025509)
(640,3.74049967410211)
(641,3.7404805820288)
(642,3.7404614181446)
(643,3.7404429937889)
(644,3.74042496445342)
(645,3.74040890097807)
(646,3.74038926776979)
(647,3.74037152353002)
(648,3.74035252625746)
(649,3.74033603939086)
(650,3.74031681998737)
(651,3.74029990664518)
(652,3.74028027147941)
(653,3.74026014164315)
(654,3.74024156506649)
(655,3.74022117047007)
(656,3.74020356747955)
(657,3.74018450850117)
(658,3.74016633904289)
(659,3.74014958282022)
(660,3.7401301214953)
(661,3.74011029350001)
(662,3.74009019470237)
(663,3.74006904952426)
(664,3.74004884025743)
(665,3.74002541632284)
(666,3.7400060571524)
(667,3.7399824110428)
(668,3.7399631255681)
(669,3.73994258380541)
(670,3.73992532750292)
(671,3.73990245314094)
(672,3.73988269064243)
(673,3.73986097450602)
(674,3.73983834320463)
(675,3.73982346149618)
(676,3.739808806616)
(677,3.7397894434327)
(678,3.73977530314519)
(679,3.73975817222208)
(680,3.73974305035264)
(681,3.73972789694524)
(682,3.73971465170234)
(683,3.73969961708132)
(684,3.73968447763494)
(685,3.73966606560046)
(686,3.73964990507722)
(687,3.73963166453637)
(688,3.73961687748722)
(689,3.73960278539316)
(690,3.73958896406016)
(691,3.73957411917519)
(692,3.73955883151207)
(693,3.73954533116955)
(694,3.73953066789554)
(695,3.73951893758622)
(696,3.73950519910128)
(697,3.73949317816142)
(698,3.73948031930539)
(699,3.7394688100309)
(700,3.73945584122258)
(701,3.73944436754757)
(702,3.73943138711405)
(703,3.73941893881355)
(704,3.73940744492966)
(705,3.73939534553567)
(706,3.73938459506139)
(707,3.73937227444512)
(708,3.73936214675933)
(709,3.73935006882915)
(710,3.73934039984275)
(711,3.7393292378273)
(712,3.73931901661485)
(713,3.73930878956023)
(714,3.73929906521196)
(715,3.7392875530797)
(716,3.739277916638)
(717,3.73926730332421)
(718,3.73925774458292)
(719,3.7392479932643)
(720,3.73923874038437)
(721,3.73922981791569)
(722,3.73922154686658)
(723,3.73921167046033)
(724,3.73920331711842)
(725,3.73919456397858)
(726,3.73918631566387)
(727,3.73917834474922)
(728,3.73917042270363)
(729,3.73916237844795)
(730,3.73915299143573)
(731,3.73914427576656)
(732,3.73913543069423)
(733,3.7391267066131)
(734,3.73911726655116)
(735,3.73910819104071)
(736,3.73909971441933)
(737,3.73909149254162)
(738,3.73908401550042)
(739,3.7390763009519)
(740,3.73906843692166)
(741,3.73905968627323)
(742,3.73905065076069)
(743,3.73904192751378)
(744,3.73903357636314)
(745,3.73902604377139)
(746,3.73901843709885)
(747,3.73900889415562)
(748,3.73900160309667)
(749,3.73899319368035)
(750,3.73898650568641)
(751,3.73897822879846)
(752,3.73897089324449)
(753,3.73896301795268)
(754,3.73895517076489)
(755,3.73894765985481)
(756,3.73894012512977)
(757,3.73893298043012)
(758,3.73892562206563)
(759,3.73891939171553)
(760,3.73891149068061)
(761,3.73890577199222)
(762,3.73889911737097)
(763,3.73889281153022)
(764,3.73888634911453)
(765,3.73887975607411)
(766,3.73887342211177)
(767,3.73886617331568)
(768,3.73885876920499)
(769,3.73885000520325)
(770,3.73884207976913)
(771,3.7388343769273)
(772,3.7388274048653)
(773,3.73882060786759)
(774,3.73881355154971)
(775,3.73880633393491)
(776,3.73879796483344)
(777,3.73879031328148)
(778,3.73878204196246)
(779,3.73877368954621)
(780,3.7387640563381)
(781,3.73875435569102)
(782,3.73874441370142)
(783,3.73873577632889)
(784,3.73872607226107)
(785,3.73871813795245)
(786,3.73870868666259)
(787,3.73870018513551)
(788,3.73869178497775)
(789,3.73868341674856)
(790,3.73867597355206)
(791,3.73866757594054)
(792,3.73866073760676)
(793,3.73865220868223)
(794,3.73864498636212)
(795,3.73863867871789)
(796,3.73863129113675)
(797,3.7386256295367)
(798,3.73861884723437)
(799,3.73861248452653)
(800,3.73860612432279)
(801,3.73860026767612)
(802,3.73859455441694)
(803,3.73858820576177)
(804,3.73858127593893)
(805,3.73857346277667)
(806,3.73856626769453)
(807,3.73855891495757)
(808,3.7385528572699)
(809,3.73854581476239)
(810,3.73853905091437)
(811,3.73853219868304)
(812,3.73852646389334)
(813,3.73852015640883)
(814,3.738514909154)
(815,3.73850853000471)
(816,3.73850284708773)
(817,3.73849721927389)
(818,3.73849239098663)
(819,3.73848737215917)
(820,3.73848264838812)
(821,3.7384774604575)
(822,3.73847211337382)
(823,3.73846617751523)
(824,3.73846124619406)
(825,3.73845554427388)
(826,3.73845028752177)
(827,3.73844497493247)
(828,3.73843973399836)
(829,3.73843455112398)
(830,3.73842932788865)
(831,3.7384248373687)
(832,3.73841926490839)
(833,3.73841382887902)
(834,3.73840825490896)
(835,3.73840316697943)
(836,3.73839725571907)
(837,3.73839259373779)
(838,3.73838684888716)
(839,3.73838157655308)
(840,3.73837646790498)
(841,3.73837097186236)
(842,3.73836663183237)
(843,3.73836108335151)
(844,3.73835602080752)
(845,3.73835055704804)
(846,3.7383453271969)
(847,3.73834024667299)
(848,3.73833463207854)
(849,3.73832918703276)
(850,3.73832291413042)
(851,3.73831687036546)
(852,3.73831125938319)
(853,3.73830514651685)
(854,3.73829991680505)
(855,3.73829440892003)
(856,3.73828878673197)
(857,3.73828492855531)
(858,3.73827867671623)
(859,3.73827541310651)
(860,3.73827013837977)
(861,3.73826546873338)
(862,3.73826032775041)
(863,3.73825597165023)
(864,3.7382512648817)
(865,3.73824696765095)
(866,3.73824254523716)
(867,3.7382379994685)
(868,3.73823371601288)
(869,3.73822972549451)
(870,3.73822632560494)
(871,3.73822244872733)
(872,3.73821918459051)
(873,3.73821545528507)
(874,3.73821302782372)
(875,3.73820938709438)
(876,3.73820629217906)
(877,3.73820275150971)
(878,3.73819902044947)
(879,3.73819619970524)
(880,3.73819282857863)
(881,3.73818966762606)
(882,3.7381860574362)
(883,3.73818265774078)
(884,3.73817922485455)
(885,3.73817584588696)
(886,3.73817224097197)
(887,3.73816854419949)
(888,3.73816513570857)
(889,3.73816169701156)
(890,3.73815822907931)
(891,3.738154984763)
(892,3.7381514622296)
(893,3.73814881262211)
(894,3.73814528322617)
(895,3.73814222112149)
(896,3.73813896525792)
(897,3.73813595598848)
(898,3.73813340634269)
(899,3.73813059803052)
(900,3.7381277791384)
(901,3.73812488939121)
(902,3.73812223485436)
(903,3.738119523562)
(904,3.73811680377472)
(905,3.73811364019401)
(906,3.73811058778631)
(907,3.73810768352435)
(908,3.73810482442504)
(909,3.73810208533219)
(910,3.73809903221664)
(911,3.73809600569151)
(912,3.73809255900566)
(913,3.73808923457356)
(914,3.73808606315053)
(915,3.73808249898209)
(916,3.73807919745919)
(917,3.7380758203985)
(918,3.7380730837772)
(919,3.73806991960299)
(920,3.73806706663024)
(921,3.73806439392417)
(922,3.73806144849824)
(923,3.73805888068136)
(924,3.73805574565061)
(925,3.73805283912954)
(926,3.73804986567239)
(927,3.73804727935088)
(928,3.73804461017825)
(929,3.738042270261)
(930,3.73803924710539)
(931,3.73803617191435)
(932,3.73803334611889)
(933,3.73803077743158)
(934,3.73802840684875)
(935,3.7380254392557)
(936,3.73802278937475)
(937,3.73801986469609)
(938,3.73801735068798)
(939,3.7380152051394)
(940,3.73801277545141)
(941,3.73801038977061)
(942,3.73800794877972)
(943,3.738005806859)
(944,3.73800336383142)
(945,3.73800155049602)
(946,3.73799943022637)
(947,3.7379975465971)
(948,3.73799545398977)
(949,3.73799329822184)
(950,3.7379913470985)
(951,3.73798894849061)
(952,3.73798661737822)
(953,3.73798395675553)
(954,3.7379811405716)
(955,3.73797845319369)
(956,3.73797576076701)
(957,3.73797395783296)
(958,3.73797136749773)
(959,3.73796935482986)
(960,3.7379668332172)
(961,3.73796445905763)
(962,3.73796222369774)
(963,3.73795983591007)
(964,3.73795762023251)
(965,3.73795502080226)
(966,3.73795254930971)
(967,3.73794994355625)
(968,3.73794740372277)
(969,3.73794484234453)
(970,3.73794235330776)
(971,3.73793976003286)
(972,3.73793738086568)
(973,3.73793478184286)
(974,3.73793241889013)
(975,3.73792973722227)
(976,3.73792759920222)
(977,3.73792499382056)
(978,3.73792301105486)
(979,3.73792085709439)
(980,3.73791863842735)
(981,3.73791644598183)
(982,3.73791424258246)
(983,3.73791210217146)
(984,3.73790993142622)
(985,3.7379079095607)
(986,3.73790596917292)
(987,3.73790406917185)
(988,3.7379021328438)
(989,3.73790010390277)
(990,3.73789798257262)
(991,3.73789573849813)
(992,3.73789332214728)
(993,3.73789099321558)
(994,3.73788872315638)
(995,3.73788659227978)
(996,3.7378844678229)
(997,3.73788238443127)
(998,3.73788001244578)
(999,3.73787777361877)
(1000,3.73787559857741)
(1001,3.73787363657249)
(1002,3.73787134590187)
(1003,3.73786962462704)
(1004,3.73786732096394)
(1005,3.73786543193935)
(1006,3.73786341171314)
(1007,3.73786143982061)
(1008,3.73785933256784)
(1009,3.73785727496065)
(1010,3.7378549937527)
(1011,3.73785327666587)
(1012,3.73785120321333)
(1013,3.73784936251772)
(1014,3.73784744529523)
(1015,3.73784545629565)
(1016,3.73784367740762)
(1017,3.73784174955561)
(1018,3.73784013250464)
(1019,3.73783820960295)
(1020,3.73783656398327)
(1021,3.73783475691151)
(1022,3.73783290794361)
(1023,3.73783092634172)
(1024,3.73782895564209)
(1025,3.73782706026381)
(1026,3.7378251225944)
(1027,3.73782290273813)
(1028,3.73782103014618)
(1029,3.73781944793876)
(1030,3.73781743459305)
(1031,3.73781606714392)
(1032,3.73781439517502)
(1033,3.73781260601281)
(1034,3.73781077490974)
(1035,3.73780907424129)
(1036,3.73780732914821)
(1037,3.73780586743132)
(1038,3.73780396584181)
(1039,3.73780249381773)
(1040,3.73780080961943)
(1041,3.7377995018793)
(1042,3.73779803826332)
(1043,3.73779657575801)
(1044,3.73779491621912)
(1045,3.7377931979854)
(1046,3.73779161775134)
(1047,3.73779033791641)
(1048,3.73778887316342)
(1049,3.7377877897758)
(1050,3.73778620500413)
(1051,3.73778506843068)
(1052,3.73778416471776)
(1053,3.73778260549099)
(1054,3.73778174262242)
(1055,3.73778069006054)
(1056,3.73777969017323)
(1057,3.73777880875506)
(1058,3.73777783454191)
(1059,3.73777695419631)
(1060,3.73777600634838)
(1061,3.7377750327335)
(1062,3.73777398920056)
(1063,3.73777306393962)
(1064,3.73777215564349)
(1065,3.73777142353804)
(1066,3.73777041282186)
(1067,3.73776957896351)
(1068,3.73776859524338)
(1069,3.73776779469995)
(1070,3.7377669318555)
(1071,3.73776614028247)
(1072,3.73776528438792)
(1073,3.73776444469955)
(1074,3.73776366949448)
(1075,3.73776295787662)
(1076,3.73776229564913)
(1077,3.73776153431477)
(1078,3.73776080537183)
(1079,3.73776001308687)
(1080,3.73775925332851)
(1081,3.73775848189563)
(1082,3.73775770819448)
(1083,3.73775692860469)
(1084,3.73775606777868)
(1085,3.7377551993551)
(1086,3.73775426707449)
(1087,3.7377533883616)
(1088,3.73775249517319)
(1089,3.73775159515056)
(1090,3.73775062651259)
(1091,3.73774964547174)
(1092,3.7377487280397)
(1093,3.73774783450545)
(1094,3.73774696245769)
(1095,3.7377460913254)
(1096,3.73774521027453)
(1097,3.73774437109701)
(1098,3.73774352008182)
(1099,3.73774265814851)
(1100,3.73774176048393)
(1101,3.73774084119726)
(1102,3.73773990751915)
(1103,3.73773900572545)
(1104,3.73773806794322)
(1105,3.73773719278047)
(1106,3.73773624508499)
(1107,3.73773521718683)
(1108,3.73773412527184)
(1109,3.73773313687008)
(1110,3.73773214220548)
(1111,3.73773144429205)
(1112,3.73773055924199)
(1113,3.73772974500404)
(1114,3.73772887768844)
(1115,3.73772805425816)
(1116,3.73772733127552)
(1117,3.73772646498446)
(1118,3.73772584015222)
(1119,3.73772502866064)
(1120,3.73772442444391)
(1121,3.73772376832508)
(1122,3.73772310878609)
(1123,3.73772237340603)
(1124,3.73772156767873)
(1125,3.73772078112301)
(1126,3.73771991074325)
(1127,3.73771924954994)
(1128,3.73771856522111)
(1129,3.73771795111994)
(1130,3.73771733073921)
(1131,3.73771679049915)
(1132,3.7377162780014)
(1133,3.73771582706726)
(1134,3.73771535828913)
(1135,3.73771490951183)
(1136,3.73771446921926)
(1137,3.73771403111815)
(1138,3.73771358812464)
(1139,3.73771316301315)
(1140,3.73771279031696)
(1141,3.73771237189736)
(1142,3.73771203750067)
(1143,3.7377116599452)
(1144,3.73771127172099)
(1145,3.73771085822908)
(1146,3.73771044543171)
(1147,3.73771003003628)
(1148,3.73770961395682)
(1149,3.73770923764058)
(1150,3.73770889922817)
(1151,3.73770859452372)
(1152,3.73770827919211)
(1153,3.73770797626651)
(1154,3.73770767117597)
(1155,3.73770736857502)
(1156,3.73770706574281)
(1157,3.7377067357417)
(1158,3.73770644452416)
(1159,3.7377061624599)
(1160,3.73770591618359)
(1161,3.73770565186327)
(1162,3.73770540349742)
(1163,3.7377051498843)
(1164,3.73770489828832)
(1165,3.73770464728056)
(1166,3.73770439070047)
(1167,3.73770415575694)
(1168,3.73770391578511)
(1169,3.73770368437559)
(1170,3.73770345536176)
(1171,3.73770322558527)
(1172,3.73770298686895)
(1173,3.73770273028132)
(1174,3.73770245482769)
(1175,3.73770217413684)
(1176,3.7377019331503)
(1177,3.73770169907311)
(1178,3.7377015139343)
(1179,3.73770127989218)
(1180,3.73770105128693)
(1181,3.73770079954456)
(1182,3.73770054955769)
(1183,3.73770029304605)
(1184,3.73770004125366)
(1185,3.73769977530764)
(1186,3.73769951295856)
(1187,3.73769926266991)
(1188,3.73769900336662)
(1189,3.73769871651389)
(1190,3.73769839480367)
(1191,3.737698074719)
(1192,3.73769774179859)
(1193,3.73769748221526)
(1194,3.73769719518852)
(1195,3.73769690637777)
(1196,3.73769663167631)
(1197,3.73769633574138)
(1198,3.73769607853776)
(1199,3.7376958144059)
(1200,3.73769554493178)
(1201,3.73769526084129)
(1202,3.73769497588662)
(1203,3.7376947008467)
(1204,3.73769439126651)
(1205,3.73769413306921)
(1206,3.73769380868222)
(1207,3.73769350853485)
(1208,3.73769317393108)
(1209,3.73769283009798)
(1210,3.73769247135905)
(1211,3.73769209858761)
(1212,3.73769171699872)
(1213,3.73769132551348)
(1214,3.73769094369066)
(1215,3.73769054557034)
(1216,3.73769011067508)
(1217,3.73768962369373)
(1218,3.73768910674606)
(1219,3.73768860218847)
(1220,3.73768810351532)
(1221,3.73768761414745)
(1222,3.73768711915892)
(1223,3.73768659307928)
(1224,3.73768607392712)
(1225,3.73768554996432)
(1226,3.73768501421687)
(1227,3.73768446010595)
(1228,3.73768388223272)
(1229,3.73768327250755)
(1230,3.73768265894461)
(1231,3.73768206471346)
(1232,3.73768146312711)
(1233,3.737680829661)
(1234,3.73768015617143)
(1235,3.73767948454796)
(1236,3.73767880574565)
(1237,3.7376781884305)
(1238,3.73767757991299)
(1239,3.73767702860267)
(1240,3.73767640534893)
(1241,3.73767589011394)
(1242,3.73767534393083)
(1243,3.73767481378436)
(1244,3.73767425344704)
(1245,3.73767369977887)
(1246,3.73767312260475)
(1247,3.73767250402789)
(1248,3.73767185901266)
(1249,3.73767124867968)
(1250,3.7376707212574)
(1251,3.73767020360522)
(1252,3.73766979089013)
(1253,3.73766931914386)
(1254,3.73766888418013)
(1255,3.7376684384465)
(1256,3.73766802886091)
(1257,3.73766763370181)
(1258,3.73766722439677)
(1259,3.73766678772803)
(1260,3.73766633837701)
(1261,3.73766593285758)
(1262,3.73766557431165)
(1263,3.73766522909212)
(1264,3.73766486655208)
(1265,3.73766447741273)
(1266,3.73766407889038)
(1267,3.7376636872703)
(1268,3.73766330797546)
(1269,3.73766293878962)
(1270,3.73766256917968)
(1271,3.73766217565688)
(1272,3.73766178751651)
(1273,3.73766138093199)
(1274,3.73766097219518)
(1275,3.73766054868311)
(1276,3.73766012064062)
(1277,3.73765968773187)
(1278,3.73765929004936)
(1279,3.7376588983647)
(1280,3.73765857729948)
(1281,3.73765824298788)
(1282,3.73765791391428)
(1283,3.73765754589012)
(1284,3.73765716449733)
(1285,3.73765675708174)
(1286,3.73765636136766)
(1287,3.73765592835829)
(1288,3.73765549899108)
(1289,3.73765507510088)
(1290,3.73765465342127)
(1291,3.73765421736184)
(1292,3.73765375636746)
(1293,3.73765328605144)
(1294,3.7376528031705)
(1295,3.73765230855558)
(1296,3.73765182613358)
(1297,3.73765138604914)
(1298,3.73765090601406)
(1299,3.73765049491484)
(1300,3.73765005768532)
(1301,3.7376496716272)
(1302,3.73764923433049)
(1303,3.73764886032749)
(1304,3.73764846126802)
(1305,3.73764806695725)
(1306,3.73764766494201)
(1307,3.73764733409389)
(1308,3.73764695345439)
(1309,3.73764667951635)
(1310,3.73764636466859)
(1311,3.73764606151295)
(1312,3.73764572119771)
(1313,3.73764539506465)
(1314,3.7376450736006)
(1315,3.73764475219649)
(1316,3.73764441786011)
(1317,3.73764409772302)
(1318,3.73764384546962)
(1319,3.7376435493154)
(1320,3.7376433645728)
(1321,3.73764314343803)
(1322,3.73764293393858)
(1323,3.73764270632371)
(1324,3.73764248831257)
(1325,3.73764226579396)
(1326,3.73764204223729)
(1327,3.73764181162365)
(1328,3.73764158827036)
(1329,3.73764138767165)
(1330,3.73764120492894)
(1331,3.7376410260218)
(1332,3.73764083475087)
(1333,3.73764061870902)
(1334,3.73764037767109)
(1335,3.73764012360505)
(1336,3.73763987771454)
(1337,3.7376396517877)
(1338,3.73763943309605)
(1339,3.73763920315945)
(1340,3.73763896945233)
(1341,3.73763873555047)
(1342,3.73763851357945)
(1343,3.7376382983679)
(1344,3.73763807808236)
(1345,3.73763784460468)
(1346,3.73763762497371)
(1347,3.73763741941811)
(1348,3.73763727192147)
(1349,3.73763710826708)
(1350,3.73763694811504)
(1351,3.73763675994024)
(1352,3.73763657289571)
(1353,3.73763639096094)
(1354,3.73763621887288)
(1355,3.73763606516677)
(1356,3.73763589480864)
(1357,3.73763575572018)
(1358,3.7376355877625)
(1359,3.73763540746619)
(1360,3.73763520898492)
(1361,3.73763501634436)
(1362,3.73763483216651)
(1363,3.73763465906271)
(1364,3.73763444415595)
(1365,3.73763426762508)
(1366,3.7376341249083)
(1367,3.73763399850648)
(1368,3.73763387003244)
(1369,3.73763372118102)
(1370,3.73763355394754)
(1371,3.73763337076347)
(1372,3.73763318226683)
(1373,3.73763300957557)
(1374,3.73763285627397)
(1375,3.73763272921596)
(1376,3.73763258036911)
(1377,3.73763242562811)
(1378,3.73763225679472)
(1379,3.73763210166237)
(1380,3.73763194127076)
(1381,3.73763177538933)
(1382,3.73763159616711)
(1383,3.73763142420793)
(1384,3.73763127614555)
(1385,3.73763114330174)
(1386,3.73763101100519)
(1387,3.73763086605537)
(1388,3.73763071601502)
(1389,3.73763056102673)
(1390,3.73763040678395)
(1391,3.73763025443388)
(1392,3.73763010651576)
(1393,3.73762996457433)
(1394,3.73762981090172)
(1395,3.73762967157435)
(1396,3.73762953207794)
(1397,3.73762940403861)
(1398,3.73762927126971)
(1399,3.73762913085929)
(1400,3.73762897165465)
(1401,3.73762881289319)
(1402,3.73762866940379)
(1403,3.73762851794946)
(1404,3.73762840128511)
(1405,3.73762827898554)
(1406,3.73762815249144)
(1407,3.73762801440821)
(1408,3.7376278791837)
(1409,3.737627741139)
(1410,3.73762760239068)
(1411,3.73762746070811)
(1412,3.73762731893616)
(1413,3.7376271795589)
(1414,3.73762704360246)
(1415,3.73762691339655)
(1416,3.73762678357674)
(1417,3.73762664896301)
(1418,3.73762650301095)
(1419,3.73762634551589)
(1420,3.73762619738923)
(1421,3.73762605917187)
(1422,3.73762592842149)
(1423,3.73762579325494)
(1424,3.73762565187908)
(1425,3.73762550779101)
(1426,3.73762536482536)
(1427,3.7376252194639)
(1428,3.73762506785865)
(1429,3.73762491092845)
(1430,3.73762474815163)
(1431,3.73762458447438)
(1432,3.73762442224002)
(1433,3.73762426148421)
(1434,3.73762409788079)
(1435,3.73762392111662)
(1436,3.73762372939314)
(1437,3.73762353033168)
(1438,3.7376233431922)
(1439,3.73762316337049)
(1440,3.73762300499781)
(1441,3.73762283337373)
(1442,3.73762265294986)
(1443,3.73762246609427)
(1444,3.73762227691635)
(1445,3.73762208489515)
(1446,3.73762188226807)
(1447,3.73762166692031)
(1448,3.73762145832925)
(1449,3.7376212527125)
(1450,3.73762109354055)
(1451,3.73762093275122)
(1452,3.73762076558291)
(1453,3.73762057658399)
(1454,3.73762037330604)
(1455,3.73762015793678)
(1456,3.73761992935639)
(1457,3.73761968848752)
(1458,3.73761944760438)
(1459,3.73761923936954)
(1460,3.73761904170001)
(1461,3.73761887684832)
(1462,3.73761867942299)
(1463,3.73761848064535)
(1464,3.73761825704742)
(1465,3.73761803861037)
(1466,3.73761780815527)
(1467,3.73761762408757)
(1468,3.73761741376535)
(1469,3.73761720951149)
(1470,3.73761699511336)
(1471,3.73761677640477)
(1472,3.73761654364145)
(1473,3.73761629476186)
(1474,3.73761603005415)
(1475,3.73761576831072)
(1476,3.73761553803933)
(1477,3.73761532837887)
(1478,3.73761509561706)
(1479,3.73761488263892)
(1480,3.73761466455759)
(1481,3.73761445499419)
(1482,3.73761423251479)
(1483,3.73761401873113)
(1484,3.73761379358154)
(1485,3.73761359131229)
(1486,3.73761336656554)
(1487,3.73761315347859)
(1488,3.73761294611287)
(1489,3.7376127445138)
(1490,3.73761254543819)
(1491,3.73761235722166)
(1492,3.73761215305585)
(1493,3.73761200841957)
(1494,3.73761185159645)
(1495,3.73761170991816)
(1496,3.73761154242304)
(1497,3.73761136162733)
(1498,3.73761117634399)
(1499,3.73761098890008)
(1500,3.73761082245617)
(1501,3.73761063768726)
(1502,3.73761047922827)
(1503,3.73761033182616)
(1504,3.73761020301296)
(1505,3.73761000742428)
(1506,3.73760986981618)
(1507,3.73760971615042)
(1508,3.73760957121213)
(1509,3.73760942139378)
(1510,3.73760926889972)
(1511,3.73760911978392)
(1512,3.73760895219256)
(1513,3.73760878556245)
(1514,3.73760862738326)
(1515,3.73760848420832)
(1516,3.73760835176741)
(1517,3.73760821591986)
(1518,3.73760807665891)
(1519,3.73760794281897)
(1520,3.73760780074793)
(1521,3.73760768653305)
(1522,3.7376075678191)
(1523,3.73760745190176)
(1524,3.73760732645725)
(1525,3.73760720331273)
(1526,3.7376070915162)
(1527,3.73760697349069)
(1528,3.73760685797723)
(1529,3.73760673826564)
(1530,3.7376066257528)
(1531,3.73760651605343)
(1532,3.73760640586584)
(1533,3.73760630379022)
(1534,3.73760620534581)
(1535,3.73760611051832)
(1536,3.73760601273951)
(1537,3.73760591134992)
(1538,3.7376058061787)
(1539,3.7376057042312)
(1540,3.73760560961895)
(1541,3.73760552315037)
(1542,3.73760544056886)
(1543,3.73760535909402)
(1544,3.73760527598597)
(1545,3.73760518856434)
(1546,3.73760509504293)
(1547,3.73760499745276)
(1548,3.73760489961755)
(1549,3.73760480296526)
(1550,3.73760470619765)
(1551,3.73760460864083)
(1552,3.7376045116938)
(1553,3.7376044157956)
(1554,3.7376043186507)
(1555,3.73760421799344)
(1556,3.73760411461452)
(1557,3.73760401132712)
(1558,3.73760390919485)
(1559,3.73760380724487)
(1560,3.73760370539528)
(1561,3.73760360498452)
(1562,3.73760350568369)
(1563,3.7376034047352)
(1564,3.73760330000792)
(1565,3.73760319386773)
(1566,3.73760309534674)
(1567,3.73760299899253)
(1568,3.73760291749075)
(1569,3.73760282680345)
(1570,3.73760273719772)
(1571,3.73760263997921)
(1572,3.73760253821961)
(1573,3.73760242776214)
(1574,3.73760230886317)
(1575,3.73760218526445)
(1576,3.73760205660743)
(1577,3.73760194672052)
(1578,3.73760184960159)
(1579,3.73760174999683)
(1580,3.73760164386156)
(1581,3.73760154318748)
(1582,3.73760143631602)
(1583,3.7376013341355)
(1584,3.73760122584062)
(1585,3.73760111936052)
(1586,3.7376010180417)
(1587,3.73760092056162)
(1588,3.73760082191887)
(1589,3.73760072204599)
(1590,3.73760062309931)
(1591,3.7376005230976)
(1592,3.73760041799236)
(1593,3.73760030903755)
(1594,3.73760020331269)
(1595,3.73760010263913)
(1596,3.73760000831251)
(1597,3.73759990605645)
(1598,3.73759981901611)
(1599,3.73759972244397)
(1600,3.73759963309246)
(1601,3.73759953859021)
(1602,3.73759944682532)
(1603,3.73759934051997)
(1604,3.73759925126324)
(1605,3.73759915527417)
(1606,3.73759907933958)
(1607,3.73759900008198)
(1608,3.7375989272918)
(1609,3.73759884266253)
(1610,3.73759875718111)
(1611,3.7375986624251)
(1612,3.73759859751473)
(1613,3.73759852887345)
(1614,3.73759846374757)
(1615,3.737598388458)
(1616,3.73759831139809)
(1617,3.73759823257879)
(1618,3.73759815282548)
(1619,3.73759806848271)
(1620,3.73759798172708)
(1621,3.73759789742966)
(1622,3.73759781075065)
(1623,3.73759771969863)
(1624,3.73759762845506)
(1625,3.73759754105537)
(1626,3.73759745671501)
(1627,3.73759737377796)
(1628,3.7375972743465)
(1629,3.73759719044345)
(1630,3.73759710801062)
(1631,3.73759703233459)
(1632,3.73759695208945)
(1633,3.7375968687333)
(1634,3.73759678093925)
(1635,3.73759668817345)
(1636,3.7375965903474)
(1637,3.73759649536603)
(1638,3.7375964087352)
(1639,3.73759632066769)
(1640,3.73759622205811)
(1641,3.73759611316816)
(1642,3.73759600556999)
(1643,3.73759590537958)
(1644,3.73759580750648)
(1645,3.73759570442166)
(1646,3.73759559725357)
(1647,3.73759549378158)
(1648,3.73759540099863)
(1649,3.7375953244327)
(1650,3.73759523858968)
(1651,3.73759517218731)
(1652,3.73759509783738)
(1653,3.73759502838762)
(1654,3.737594958195)
(1655,3.7375948944902)
(1656,3.73759483394906)
(1657,3.73759477310349)
(1658,3.73759468928392)
(1659,3.73759461627794)
(1660,3.73759455163258)
(1661,3.73759449817772)
(1662,3.73759444381947)
(1663,3.73759438472119)
(1664,3.73759432202903)
(1665,3.73759426185691)
(1666,3.73759420683496)
(1667,3.73759415627843)
(1668,3.73759410713664)
(1669,3.73759405527205)
(1670,3.7375939987328)
(1671,3.73759394046587)
(1672,3.73759388655925)
(1673,3.73759383848533)
(1674,3.73759379096115)
(1675,3.73759373764579)
(1676,3.73759367790876)
(1677,3.73759361847882)
(1678,3.73759356651393)
(1679,3.73759352238252)
(1680,3.73759348022548)
(1681,3.73759343555033)
(1682,3.73759338615417)
(1683,3.73759333710601)
(1684,3.73759328996952)
(1685,3.73759324855301)
(1686,3.73759320899794)
(1687,3.73759316732303)
(1688,3.73759312149506)
(1689,3.73759307467979)
(1690,3.73759303176461)
(1691,3.73759299401631)
(1692,3.73759295527858)
(1693,3.73759291455147)
(1694,3.737592868634)
(1695,3.73759282127566)
(1696,3.73759277628577)
(1697,3.73759273495557)
(1698,3.73759269409416)
(1699,3.73759265002587)
(1700,3.7375926011103)
(1701,3.7375925494732)
(1702,3.73759250017444)
(1703,3.73759245756809)
(1704,3.73759242040558)
(1705,3.73759238222926)
(1706,3.73759234399478)
(1707,3.73759230393978)
(1708,3.73759226777552)
(1709,3.73759224117847)
(1710,3.73759221274482)
(1711,3.73759218450323)
(1712,3.73759215258561)
(1713,3.73759212025795)
(1714,3.73759208880398)
(1715,3.73759205983218)
(1716,3.73759203015348)
(1717,3.73759199901293)
(1718,3.7375919689404)
(1719,3.7375919381267)
(1720,3.73759190717107)
(1721,3.73759187659196)
(1722,3.73759184221879)
(1723,3.73759180994768)
(1724,3.73759177551439)
(1725,3.73759174031505)
(1726,3.73759170489796)
(1727,3.73759167089929)
(1728,3.73759163656053)
(1729,3.7375916004347)
(1730,3.7375915630783)
(1731,3.7375915282947)
(1732,3.73759149809411)
(1733,3.73759146362884)
(1734,3.73759143386064)
(1735,3.73759140115549)
(1736,3.73759136793234)
(1737,3.73759133096433)
(1738,3.73759129161984)
(1739,3.73759124804235)
(1740,3.73759120083625)
(1741,3.73759114731531)
(1742,3.73759109352298)
(1743,3.7375910446454)
(1744,3.73759100027234)
(1745,3.73759095600196)
(1746,3.73759090707817)
(1747,3.73759085215696)
(1748,3.73759079503385)
(1749,3.73759074406772)
(1750,3.73759069253628)
(1751,3.73759065676692)
(1752,3.73759061428324)
(1753,3.73759057499449)
(1754,3.73759052645851)
(1755,3.73759048064913)
(1756,3.73759043455754)
(1757,3.73759038827876)
(1758,3.73759033712082)
(1759,3.73759028639018)
(1760,3.73759024157345)
(1761,3.73759019948979)
(1762,3.73759015618433)
(1763,3.73759010850923)
(1764,3.73759005680098)
(1765,3.7375900017183)
(1766,3.73758994411765)
(1767,3.73758988682915)
(1768,3.73758983203402)
(1769,3.73758978032802)
(1770,3.7375897260059)
(1771,3.7375896762351)
(1772,3.7375896203976)
(1773,3.73758957089212)
(1774,3.73758951796816)
(1775,3.73758946328458)
(1776,3.73758940384114)
(1777,3.73758934322023)
(1778,3.73758928272387)
(1779,3.7375892256255)
(1780,3.73758917537962)
(1781,3.73758912699768)
(1782,3.73758908072046)
(1783,3.73758903985921)
(1784,3.73758897858366)
(1785,3.73758894248497)
(1786,3.73758890463421)
(1787,3.73758887145974)
(1788,3.73758883391051)
(1789,3.73758879580625)
(1790,3.73758875521873)
(1791,3.73758871383408)
(1792,3.73758867142734)
(1793,3.73758862957653)
(1794,3.73758859023648)
(1795,3.73758855283849)
(1796,3.73758851594527)
(1797,3.7375884784074)
(1798,3.73758844092796)
(1799,3.73758840253326)
(1800,3.73758836633553)
(1801,3.73758833080204)
(1802,3.73758829566746)
(1803,3.73758825943895)
(1804,3.73758822261112)
(1805,3.73758818636827)
(1806,3.73758815188703)
(1807,3.73758811837005)
(1808,3.73758808434215)
(1809,3.73758804967374)
(1810,3.7375880141639)
(1811,3.73758797843636)
(1812,3.73758794257115)
(1813,3.73758790706634)
(1814,3.73758787145831)
(1815,3.73758783522635)
(1816,3.73758779735354)
(1817,3.73758775838679)
(1818,3.73758771995462)
(1819,3.73758768302255)
(1820,3.73758764647465)
(1821,3.73758760874016)
(1822,3.7375875698626)
(1823,3.73758753234608)
(1824,3.73758749891761)
(1825,3.73758746699951)
(1826,3.73758743997837)
(1827,3.73758741102558)
(1828,3.7375873807198)
(1829,3.73758734904967)
(1830,3.7375873159441)
(1831,3.73758728306383)
(1832,3.73758724941525)
(1833,3.73758721429308)
(1834,3.73758717894605)
(1835,3.73758714768612)
(1836,3.73758711820917)
(1837,3.7375870900657)
(1838,3.73758706294216)
(1839,3.73758703212777)
(1840,3.73758700582025)
(1841,3.73758698081453)
(1842,3.73758695723698)
(1843,3.73758693405397)
(1844,3.73758691019978)
(1845,3.73758688348379)
(1846,3.73758685442763)
(1847,3.73758682330132)
(1848,3.73758679452154)
(1849,3.73758676927051)
(1850,3.73758674517658)
(1851,3.73758671970837)
(1852,3.73758669209737)
(1853,3.73758666355251)
(1854,3.73758663585114)
(1855,3.73758661037551)
(1856,3.73758658579673)
(1857,3.73758656060622)
(1858,3.73758653405735)
(1859,3.73758650720927)
(1860,3.73758648167257)
(1861,3.73758645798092)
(1862,3.7375864346905)
(1863,3.73758640996677)
(1864,3.7375863836568)
(1865,3.73758635614852)
(1866,3.7375863316304)
(1867,3.73758630945259)
(1868,3.73758628857717)
(1869,3.73758626639471)
(1870,3.73758624174587)
(1871,3.73758621505475)
(1872,3.73758618847645)
(1873,3.73758616339974)
(1874,3.73758613902944)
(1875,3.73758611352485)
(1876,3.73758608456499)
(1877,3.73758605190861)
(1878,3.73758601717574)
(1879,3.73758598278636)
(1880,3.73758594961869)
(1881,3.7375859165012)
(1882,3.73758588189546)
(1883,3.73758584570932)
(1884,3.73758580970537)
(1885,3.73758577671192)
(1886,3.73758574529538)
(1887,3.73758571897931)
(1888,3.73758568752634)
(1889,3.73758566235755)
(1890,3.73758563498478)
(1891,3.73758561179947)
(1892,3.7375855885306)
(1893,3.73758556468993)
(1894,3.73758553707286)
(1895,3.73758550590131)
(1896,3.7375854748175)
(1897,3.73758544566073)
(1898,3.73758542108586)
(1899,3.73758539761021)
(1900,3.73758537253755)
(1901,3.73758534456351)
(1902,3.73758531510785)
(1903,3.73758528569317)
(1904,3.73758525670081)
(1905,3.73758522724986)
(1906,3.73758519624261)
(1907,3.73758516396205)
(1908,3.73758513103556)
(1909,3.73758509905426)
(1910,3.73758506783006)
(1911,3.73758503589382)
(1912,3.73758500093893)
(1913,3.73758496184296)
(1914,3.73758492024349)
(1915,3.73758488025846)
(1916,3.73758484464449)
(1917,3.73758481238568)
(1918,3.73758478076888)
(1919,3.73758474834576)
(1920,3.73758471583524)
(1921,3.73758468415501)
(1922,3.7375846530251)
(1923,3.737584621063)
(1924,3.73758458712408)
(1925,3.73758455127049)
(1926,3.73758451508774)
(1927,3.73758448074828)
(1928,3.73758444900891)
(1929,3.73758441827352)
(1930,3.73758438636364)
(1931,3.73758435315471)
(1932,3.73758431776736)
(1933,3.73758428677122)
(1934,3.73758425661823)
(1935,3.73758422748775)
(1936,3.73758419605623)
(1937,3.73758416357973)
(1938,3.73758413089271)
(1939,3.73758409918838)
(1940,3.73758406771552)
(1941,3.73758403658106)
(1942,3.73758400854234)
(1943,3.73758398078771)
(1944,3.7375839517959)
(1945,3.73758392029415)
(1946,3.73758388761671)
(1947,3.73758385512754)
(1948,3.73758382280765)
(1949,3.73758378950549)
(1950,3.73758375483989)
(1951,3.73758371990952)
(1952,3.73758368631191)
(1953,3.73758365389461)
(1954,3.73758362264007)
(1955,3.73758359257549)
(1956,3.73758356369335)
(1957,3.73758353541574)
(1958,3.73758350816288)
(1959,3.73758348413104)
(1960,3.73758346080752)
(1961,3.73758344343344)
(1962,3.73758342475346)
(1963,3.73758340574346)
(1964,3.73758338437881)
(1965,3.73758336269404)
(1966,3.73758334033704)
(1967,3.73758331619967)
(1968,3.73758328894128)
(1969,3.73758326141635)
(1970,3.7375832388656)
(1971,3.73758321866958)
(1972,3.73758319773051)
(1973,3.73758317340571)
(1974,3.73758314693893)
(1975,3.73758312067294)
(1976,3.73758309583166)
(1977,3.73758307259917)
(1978,3.73758305081917)
(1979,3.7375830300488)
(1980,3.73758300936925)
(1981,3.73758298828815)
(1982,3.73758296812426)
(1983,3.73758295151039)
(1984,3.73758293723906)
(1985,3.73758292621097)
(1986,3.73758291413958)
(1987,3.7375829041734)
(1988,3.73758289526207)
(1989,3.73758288736179)

};
\path [draw=black, fill opacity=0] (axis cs:13,1)--(axis cs:13,1);

\path [draw=black, fill opacity=0] (axis cs:13,0)--(axis cs:13,0);

\path [draw=black, fill opacity=0] (axis cs:1,13)--(axis cs:1,13);

\path [draw=black, fill opacity=0] (axis cs:0,13)--(axis cs:0,13);

\end{axis}

\end{tikzpicture}}
   		\end{column}
   	\end{columns}
   \end{frame}
   
   \begin{frame}{Proximal Method}
   	\alert{Effect of regularization parameter $\lambda$ on solution:}\\
   	\centering\includegraphics[width = 0.7\textwidth]{lambda1.png}\\
   	\centering\includegraphics[width = 0.7\textwidth]{lambda2.png}
   \end{frame}

\section{Logistic Regression: An Example}
  \begin{frame}\frametitle{Task}
    Explain what we want to do, and explain the dataset,
    and why using both SQN and Prox makes sense   
  \end{frame}

  \begin{frame}\frametitle{Results}
    Nice table with SQN, SGD (no reg, L2), (Lasso,) Prox (L1) showing
    Obj. value in found optimum, CPU time, Iterations, F1 score of prediction model

    Use different reg. parameters??
    Stop after fixed time? after fixed iters? after insign. improvements  
  \end{frame}

\section{Conclusion}

  \begin{frame}{Summary}
    \begin{center}\ccbysa\end{center}
  \end{frame}
  

  \plain{Questions?}

  \begin{frame}[allowframebreaks]\frametitle{Main References}

    \bibliography{refs}
    \bibliographystyle{abbrv}

  \end{frame}

\end{document}
