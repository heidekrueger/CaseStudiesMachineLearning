%!TEX program = xelatex
\documentclass[10pt]{beamer}

\usetheme[progressbar=frametitle, noframetitleoffset, block=fill]{m}
%\definecolor{TUMblue}{RGB}{55,55,255}
%\setbeamercolor{alerted text}{fg=TUMblue}

\usepackage{booktabs}
\usepackage[scale=2]{ccicons}

\usepackage{pgfplots}
\usepackage{tikz}
\usepgfplotslibrary{dateplot}
\usepackage{caption}

\newlength\figureheight
\newlength\figurewidth
\DeclareMathOperator{\prox}{prox}
\DeclareMathOperator{\argmin}{argmin}

\title{Stochastic Optimization in Machine Learning}
\subtitle{Case Studies in Nonlinear Optimization}
\date{\today}
\author{F. Bauer \and S. Chambon \and R. Halbig \and S. Heidekrüger \and J. Heuke}
\institute{Technische Universität München}
%\titlegraphic{\hfill\includegraphics[height=1.5cm]{logo.eps}}

\begin{document}

\maketitle

\plain{
  \begin{quote}
    We're not running out of data anytime soon. It's maybe the only resource that grows exponentially.
    \\
    \flushright{\alert{Andreas Weigend}}
  \end{quote}
  }


\begin{frame}
  \frametitle{Outline}
  \setbeamertemplate{section in toc}[sections numbered]
  \tableofcontents[hideallsubsections]
\end{frame}

\section{Introduction}

  \begin{frame}[t]\frametitle{Introduction: What is Machine Learning (ML) ?}
	  	Implementation of autonomously learning software for:
        \begin{itemize}
        	\item Discovery of patterns and relationships in data
        	\item Prediction of future events
        \end{itemize}
        \vspace{5pt}
        \alert{Examples:}
        \begin{columns}
        	\begin{column}{.5\linewidth}
        		Electroence-phalography (EEG)\\
        		\includegraphics[width = 0.8\linewidth]{eeg_pic.jpg}\\
        		\alert{Section 4}
        	\end{column}\hspace{-10pt}
        	\begin{column}{.5\linewidth}
        		Image Denoising\\
        		\includegraphics[width = 0.8\linewidth]{lena_pic.jpg}\\
        		\alert{Section 5}
        	\end{column}
        \end{columns}
  \end{frame}

  \begin{frame}\frametitle{Introduction: ML and Optimization I}
    \alert{Training} a Machine Learning model means finding optimal parameters $\omega$:

    $$ \omega^* = \argmin_{\omega} F(\omega, X, z)$$
    
    \begin{itemize}
      \item \alert{$F$}: Loss function of chosen ML-model
      \item \alert{$X$}: The training data ($N:=\#$samples $\times$ $\#$features matrix)
      \item \alert{$z$}: Training labels (only in classification models; vector of size $N$)
      
      \item The dimension $n$ of $\omega$ is model dependent, often $\#$features$+1$
    \end{itemize}   
  \end{frame}

  \begin{frame}\frametitle{Introduction: ML and Optimization II}
    After we have found $\omega^*$, we can do \alert{Prediction} on new data points:

    $$ \hat {z_i} := h(\omega^*, x_i)$$
    
    \begin{itemize}
      \item \alert{$x_i$}: new data point with \emph{unknown} label \alert{$z_i$}
      \item \alert{$h$}: hypothesis function of the ML model
    \end{itemize}   
  \end{frame}

  \begin{frame}
    \frametitle{Introduction: Challenges in Machine Learning}
      \begin{itemize}
        \item Massive amounts of training data 
        \item Construction of very large models
        \item Handling high memory/computational demands
      \end{itemize}
      \vspace{36pt}
      
    \centering \large{Ansatz: \alert{Stochastic Methods}}
  \end{frame}
  
  \begin{frame}{Introduction: Stochastic Framework}
    $$ F(\omega) := \mathbb{E}\left[f(\omega, \xi)\right] \uncover<3>{= \frac{1}{N}\sum_{i=1}^N f(\omega, x_i, z_i)}$$
    \begin{itemize}
      \item<2-> \alert{$\xi$}: Random variable; takes the form of an input-output-pair $(x_i, z_i)$
      \item<3-> \alert{$f$}: Partial loss function corresponding to a single data point.
    \end{itemize}
  \end{frame}

  \begin{frame}{Introduction: Stochastic Methods}
    \begin{columns}[T]
      \begin{column}{.5\textwidth}
        \centering \alert{Gradient Method}
        $$\min F(\omega) $$

        \uncover<2->{
        $$\omega^{(k+1)}:= \omega^{k}-\alpha_k \nabla F(\omega^{k})$$\\
        \phantom{zeile}
        }
      \end{column}\hfill
      \begin{column}{.5\textwidth}
        \centering \alert{Stochastic Gradient Descent}
        $$\min \mathbb E \left [f(\omega, \xi)\right]$$
        \uncover<3>{
          $$\omega^{k+1}:= \omega^{k}-\alpha_k \alert{\nabla \hat F(\omega^{k})} $$
          with
          $$\alert{\nabla \hat F(\omega^{k})} := \frac{1}{b}\sum_{i\in \mathcal S_k}f(\omega, x_i, z_i)$$
          where $\mathcal S_k \subset [N], \quad b:=|\mathcal S_k| \ll N$\\\alert{"Mini Batch"}
        }
      \end{column}
    \end{columns}
  \end{frame}

\section{SQN: A Stochastic Quasi-Newton Method}

  \begin{frame}\frametitle{Stochastic Quasi-Newton Method (SQN)}
      \begin{itemize}
        \item \alert{Stochastically} use second-order information
        \item Based on BFGS-method.
        \
        \item Basic idea: $$ \omega^{k+1} = \omega^{k} - \alpha_k \alert{H_t} \nabla \hat F(\omega^{k})$$
        
        \item $t$ running on slower time-scale than $k$. 
        \item $H_t$ update in $\mathcal O(n)$ time and constant memory, using several tricks
      \end{itemize}
  \end{frame}

  \subsection*{Behavior}

  \begin{frame}
    \frametitle{Behavior I}

      \begin{columns}[T]
      \begin{column}{.5\textwidth}
        \resizebox{\linewidth}{!}{\input{'EEG Sample Objective vs. Iterations.tikz'}}
      \end{column}\hfill
      \begin{column}{.5\textwidth}
        \resizebox{\linewidth}{!}{\input{'EEG Fixed Subset Objective vs. Iterations.tikz'}}
      \end{column}
    \end{columns}
    \center{Performance on EEG Dataset, Problem size: $69550 \times 600$\\
    \tiny{Armijo-stepsizes, Further SQN-parameters: $L=10$, $M=5$}}
  \end{frame}

  \begin{frame}
    \frametitle{Behavior II}
    \resizebox{\linewidth}{!}{\input{'EEG Fixed Subset Objective vs. Accessed Data Points.tikz'}}
    \center{Performance on EEG Dataset, Problem size: $69550 \times 600$\\
    \tiny{Armijo-stepsizes, Further SQN-parameters: $L=10$, $M=5$}}
  \end{frame}

  \begin{frame}\frametitle{Results}
    \begin{itemize}
      \item Can be faster than SGD on appropriate Datasets
      \item Requires tedious, manual tuning of hyperparameters to be efficient!
    \end{itemize}
  
  \end{frame}

 \section{Proximal Method}

   \begin{frame}{Proximal Method}
       \begin{flalign*}
       	\text{\alert{Problem}}&&
       	\min_x &\;F(x) := \underbrace{f(x)}_{smooth} \quad + \underbrace{h(x)}_{non-smooth}&
       \end{flalign*}
       
       \begin{flalign*}
       	\text{\alert{Proximity Operator}}&&\prox_f(v) = &\underset{x}{\argmin} \; \bigl( f(x) + \frac{1}{2} \lVert x - v \rVert^2_2 \bigr)&
       \end{flalign*}
		\begin{figure}[t] 
			
			\centering\includegraphics[width = 0.5\textwidth]{prox_boyd.jpg}
			\caption{\footnotesize Evaluating a proximal operator at various points. \textit{N Parikh, S Boyd, Proximal Methods,
					Foundations and Trends in Optimization 1, 2014}}
		\end{figure} 	
   \end{frame}
   
   \begin{frame}{Proximal Method}
   	\alert{Traditional Proximal Gradient Step:}
   	\begin{equation*}
   	x_{k+1} = \prox_{\lambda_kh}(x_k - \lambda_k\nabla f(x_k))
   	\end{equation*}
   	\alert{Quasi-Newton Proximal Step:}
   	\begin{equation*}
   	x_{k+1} = \prox_h^{B_k}(x_k - B_k^{-1}\nabla f(x_k)),
   	\end{equation*}
   	with $B_k = \underbrace{D_k}_{diag} + \underbrace{u_k}_{\in\mathbb{R}^n}u_k^T$.
   \end{frame}
   
   \begin{frame}{Proximal Method}
   	\begin{columns}[T]
   		\begin{column}{.5\textwidth}
   			$F(x) = \lVert Ax - b \rVert + \lambda \lVert x \rVert_1$\\
   			$A \in \mathbb{R}^{1500 \times 3000},\:b \in \mathbb{R}^{1500}$\\
   			$A_{ij},\:b_i\:$ \textasciitilde $\:\mathcal{N}(0,1)$, $\:\lambda = 0.1$\\
   			\vspace{15pt}
   			\resizebox{\linewidth}{!}{% This file was created by matplotlib v0.1.0.
% Copyright (c) 2010--2014, Nico Schl�mer <nico.schloemer@gmail.com>
% All rights reserved.
% 
% The lastest updates can be retrieved from
% 
% https://github.com/nschloe/matplotlib2tikz
% 
% where you can also submit bug reports and leavecomments.
% 
\begin{tikzpicture}

\begin{axis}[
xlabel={Number of Iterations},
ylabel={Function Value},
xmin=0, xmax=60,
ymin=8.97211792181375, ymax=100000,
ymode=log,
axis on top,
legend entries={{0SR1},{ProxGrad},{L-BFGS-B}}
]
\addplot [thick, red]
coordinates {
(0,23930.000884189)
(1,7180.45388604586)
(2,1602.10980923053)
(3,580.81114454538)
(4,519.996842220967)
(5,203.222944097077)
(6,121.922299073436)
(7,51.6523543005163)
(8,59.2695972484554)
(9,30.819611673989)
(10,25.614751730395)
(11,21.7592732486774)
(12,22.654003528062)
(13,20.15049001864)
(14,19.850460675318)
(15,18.3912111072782)
(16,23.73822362313)
(17,17.8866027335036)
(18,17.8196671916423)
(19,17.7775123499805)
(20,18.0615078838142)
(21,17.7506027746239)
(22,17.7424949039984)
(23,17.7297358775166)
(24,18.0476551294605)
(25,17.7168284338285)
(26,17.7118227102102)
(27,17.897336767244)
(28,17.7264425474283)
(29,17.6583909070735)
(30,17.6466963489978)
(31,17.5726603204587)
(32,17.9785494963624)
(33,17.5007993751903)
(34,17.4774762719696)
(35,17.4693572907426)
(36,18.0844886660129)
(37,17.4544077868014)
(38,17.4488948638463)
(39,17.3764214969115)
(40,17.5899917220372)
(41,17.1689206218387)
(42,17.0926131263914)
(43,17.193691075103)
(44,19.0707895757655)
(45,16.9630825309593)
(46,16.9036371416858)
(47,16.8413981063643)
(48,16.8743377801165)
(49,16.6801578404075)
(50,16.5909092149981)
(51,15.9390711318578)
(52,15.2212036355576)
(53,13.1777255991126)
(54,11.3509943341348)
(55,10.0944946484959)
(56,9.71465222794144)
(57,9.40843618327981)
(58,9.31008072896225)
(59,9.12293542675185)
(60,8.97211792181375)

};
\addplot [thick, blue]
coordinates {
(0,23930.000884189)
(1,7180.45388604586)
(2,3617.86527996726)
(3,2206.80793392946)
(4,1491.00767628939)
(5,1073.99586354884)
(6,808.278854547473)
(7,628.014922400163)
(8,499.960644278589)
(9,405.752764676303)
(10,334.52634655712)
(11,279.477813847677)
(12,236.164491398145)
(13,201.577684383878)
(14,173.611856832551)
(15,150.754899712087)
(16,131.896367104543)
(17,116.20926011348)
(18,103.06516983272)
(19,91.9789779656275)
(20,82.5721972749523)
(21,74.5480919899652)
(22,67.6703468931341)
(23,61.7492516151206)
(24,56.6308080266971)
(25,52.1895417181108)
(26,48.3230738518772)
(27,44.9473811889776)
(28,41.9908958545994)
(29,39.3943307887263)
(30,37.1076648133911)
(31,35.0895582332656)
(32,33.3047098499916)
(33,31.7230458561848)
(34,30.3185369243055)
(35,29.0691392036425)
(36,27.9557168948839)
(37,26.9619473758925)
(38,26.0735819006833)
(39,25.2785051309949)
(40,24.5658953302441)
(41,23.9261996677162)
(42,23.3512128923731)
(43,22.8338555363931)
(44,22.3679803523897)
(45,21.9478508869475)
(46,21.5686369463874)
(47,21.2260325999356)
(48,20.9161983824383)
(49,20.6357394037178)
(50,20.3816052485759)
(51,20.1511846245288)
(52,19.942073913322)
(53,19.7522963849272)
(54,19.5798157329886)
(55,19.4229051369124)
(56,19.2800427060627)
(57,19.1499227239291)
(58,19.0313113044641)
(59,18.9230788312623)
(60,18.8242439624631)

};
\addplot [thick, green!50.0!black]
coordinates {
(0,23930.000884189)
(1,12083.1529253535)
(2,1962.70076992746)
(3,950.485602345173)
(4,331.959633328347)
(5,243.914762566925)
(6,134.329835058351)
(7,101.908974098636)
(8,71.430490269532)
(9,53.8008057292489)
(10,49.2070909809805)
(11,47.2377989642264)
(12,45.8394992598531)
(13,44.9246042975355)
(14,44.7440629754329)
(15,44.39329787842)
(16,44.3236229723654)
(17,44.2184239820041)
(18,44.1931882049634)
(19,44.1267629597911)
(20,44.1051433432534)
(21,44.0736620641302)
(22,44.0227863049802)
(23,43.9604053846835)
(24,43.8383240409788)
(25,43.7012466950384)
(26,43.4900906898139)
(27,43.1528664627698)
(28,42.5004315629993)
(29,41.8013543791512)
(30,40.8296742339329)
(31,39.900023272227)
(32,38.5823534699433)
(33,37.4156430458995)
(34,36.2719668390337)
(35,35.0776601950381)
(36,34.1837687644875)
(37,33.2698645775237)
(38,32.1962015032741)
(39,31.3082549503913)
(40,30.4262826422634)
(41,29.5345560514281)
(42,28.6995811800974)
(43,27.8507440638526)
(44,26.8903832876716)
(45,26.0249848856949)
(46,25.0716863775733)
(47,24.1555764755733)
(48,23.4070450974266)
(49,22.7049953977644)
(50,22.0172886765464)
(51,21.3371046021488)
(52,20.6002987471974)
(53,19.9394587462096)
(54,19.3137752887323)
(55,18.6682287336718)
(56,18.1127667913341)
(57,17.5853401003318)
(58,16.9831567654981)
(59,16.4442553456098)
(60,15.9565899295854)
(61,15.4524799469826)

};
\path [draw=black, fill opacity=0] (axis cs:1,13)--(axis cs:1,13);

\path [draw=black, fill opacity=0] (axis cs:13,0)--(axis cs:13,0);

\path [draw=black, fill opacity=0] (axis cs:13,1)--(axis cs:13,1);

\path [draw=black, fill opacity=0] (axis cs:0,13)--(axis cs:0,13);

\end{axis}

\end{tikzpicture}}
   			\begin{center}
   				\hspace{-3pt}
   				\scalebox{0.85}{
   					\begin{tabular}{|c|c|c|c|}
   						\hline                       
   						&\tiny \textbf{0SR1} & \tiny \textbf{ProxGrad} &  \tiny \textbf{L-BFGS-B} \\ \hline
   						\tiny \textbf{Iterations} &\tiny  1,822	& \tiny 135,328 & \tiny 1,989 \\	\hline  
   						\tiny \textbf{Run-Time}&\tiny 68 s & \tiny 1,144 s &\tiny 56 s  \\ \hline
   						
   					\end{tabular}
   				}
   			\end{center}
   		
   		\end{column}\hfill
   		\begin{column}{.5\textwidth}
   			$F(x) = \lVert Ax - b \rVert + \lambda \lVert x \rVert_1$\\
   			$A \in \mathbb{R}^{2197 \times 2197},\:b \in \mathbb{R}^{2197}$\\
   			$A$: \small Discretization of 3D Laplacian\\
   			\normalsize$\lambda = 1$\\
   			\vspace{8pt}
   			\resizebox{\linewidth}{!}{% This file was created by matplotlib v0.1.0.
% Copyright (c) 2010--2014, Nico Schl�mer <nico.schloemer@gmail.com>
% All rights reserved.
% 
% The lastest updates can be retrieved from
% 
% https://github.com/nschloe/matplotlib2tikz
% 
% where you can also submit bug reports and leavecomments.
% 
\begin{tikzpicture}

\begin{axis}[
xlabel={Number of Iterations},
ylabel={Function Value},
xmin=0, xmax=20,
ymin=1e-06, ymax=1000,
ymode=log,
axis on top,
legend entries={{0SR1},{ProxGrad},{L-BFGS-B}}
]
\addplot [thick, red]
coordinates {
(0,333.264307802431)
(1,215.357154371654)
(2,100.123663957516)
(3,21.4837427619828)
(4,3.0897485708941)
(5,0.187433891370735)
(6,6.51020673790687e-06)

};
\addplot [thick, blue]
coordinates {
(0,333.264307802431)
(1,215.357154371654)
(2,170.003421267992)
(3,141.522150053744)
(4,119.156668702769)
(5,99.6101770700706)
(6,81.6789720647588)
(7,64.9604634429121)
(8,49.5057952403396)
(9,35.6628328020705)
(10,23.8398203524616)
(11,14.5158425646087)
(12,7.80320158995826)
(13,3.60559254143513)
(14,1.46309958044681)
(15,0.474434423191431)
(16,0.129794470233551)
(17,0.0155473980678074)
(18,6.51020673590862e-06)

};
\addplot [thick, green!50.0!black]
coordinates {
(0,333.264307802431)
(1,261.727510586159)
(2,224.241384163549)
(3,157.245982662659)
(4,46.7748392207664)
(5,9.43034561455249)
(6,0.80243045230926)
(7,0.080090115880621)
(8,0.0258285251052386)
(9,6.51020673590862e-06)

};
\path [draw=black, fill opacity=0] (axis cs:1,13)--(axis cs:1,13);

\path [draw=black, fill opacity=0] (axis cs:13,0)--(axis cs:13,0);

\path [draw=black, fill opacity=0] (axis cs:13,1)--(axis cs:13,1);

\path [draw=black, fill opacity=0] (axis cs:0,13)--(axis cs:0,13);

\end{axis}

\end{tikzpicture}}
   			\begin{center}
   				\hspace{5pt}
   				\scalebox{0.85}{
   					\begin{tabular}{|c|c|c|c|}
   						\hline                       
   						&\tiny \textbf{0SR1} & \tiny \textbf{ProxGrad} &  \tiny \textbf{L-BFGS-B} \\ \hline
   						\tiny \textbf{Iterations}& \tiny  7	& \tiny 18 & \tiny 10 \\	\hline  
   						\tiny \textbf{Run-Time}& \tiny 0.037 s &\tiny 0.004 s &\tiny 0.022 s\\ \hline
   						
   					\end{tabular}
   				}
   			\end{center}
   		\end{column}
   	\end{columns}
   \end{frame}
   
   \begin{frame}{Proximal Method: Stochastic Extension}
   High-dimensional data:
   Extension to stochastic framework\\
   \vspace{10pt}
   \alert{Effect of batch size}
   	\begin{columns}[T]
   		\hspace{-16pt}
   		\begin{column}{.3\textwidth}
   			\hspace{30pt} \scriptsize Batch size = 1
   			\vspace{10pt}
   			\resizebox{1.18\linewidth}{!}{% This file was created by matplotlib v0.1.0.
% Copyright (c) 2010--2014, Nico Schl�mer <nico.schloemer@gmail.com>
% All rights reserved.
% 
% The lastest updates can be retrieved from
% 
% https://github.com/nschloe/matplotlib2tikz
% 
% where you can also submit bug reports and leavecomments.
% 
\begin{tikzpicture}

\begin{axis}[
xlabel={Number of Iterations},
ylabel={Function Value},
xmin=0, xmax=100,
ymin=1, ymax=100000,
ymode=log,
axis on top,
legend entries={{0SR1},{PG},{S0SR1},{SPG}}
]
\addplot [thick, red]
coordinates {
(0,23930.000884189)
(1,7180.45388604586)
(2,1602.10980923053)
(3,580.81114454538)
(4,519.996842220967)
(5,203.222944097077)
(6,121.922299073436)
(7,51.6523543005163)
(8,59.2695972484554)
(9,30.819611673989)
(10,25.614751730395)
(11,21.7592732486774)
(12,22.654003528062)
(13,20.15049001864)
(14,19.850460675318)
(15,18.3912111072782)
(16,23.73822362313)
(17,17.8866027335036)
(18,17.8196671916423)
(19,17.7775123499805)
(20,18.0615078838142)
(21,17.7506027746239)
(22,17.7424949039984)
(23,17.7297358775166)
(24,18.0476551294605)
(25,17.7168284338285)
(26,17.7118227102102)
(27,17.897336767244)
(28,17.7264425474283)
(29,17.6583909070735)
(30,17.6466963489978)
(31,17.5726603204587)
(32,17.9785494963624)
(33,17.5007993751903)
(34,17.4774762719696)
(35,17.4693572907426)
(36,18.0844886660129)
(37,17.4544077868014)
(38,17.4488948638463)
(39,17.3764214969115)
(40,17.5899917220372)
(41,17.1689206218387)
(42,17.0926131263914)
(43,17.193691075103)
(44,19.0707895757655)
(45,16.9630825309593)
(46,16.9036371416858)
(47,16.8413981063643)
(48,16.8743377801165)
(49,16.6801578404075)
(50,16.5909092149981)
(51,15.9390711318578)
(52,15.2212036355576)
(53,13.1777255991126)
(54,11.3509943341348)
(55,10.0944946484959)
(56,9.71465222794144)
(57,9.40843618327981)
(58,9.31008072896225)
(59,9.12293542675185)
(60,8.97211792181375)
(61,9.96955580951941)
(62,8.8971778312592)
(63,8.89032770510936)
(64,8.846532027156)
(65,9.14275424383174)
(66,8.76428614407871)
(67,8.74549967678879)
(68,8.81264423855369)
(69,8.92224051152909)
(70,8.71372433712952)
(71,8.69686120584995)
(72,8.86066032777056)
(73,8.64994348063148)
(74,8.64177834237841)
(75,8.61074429525064)
(76,8.97413158797716)
(77,8.57239195780947)
(78,8.56084322589053)
(79,8.56535473569669)
(80,8.70378144008176)
(81,8.55111754917089)
(82,8.54693544975204)
(83,8.66605947747419)
(84,8.49089417601578)
(85,8.48541554539908)
(86,8.44561206967175)
(87,8.58014556574557)
(88,8.40335231266141)
(89,8.3639219974089)
(90,8.30542294510122)
(91,8.53249537616582)
(92,8.29202808913906)
(93,8.25359007491097)
(94,8.22495187877562)
(95,8.2493220091081)
(96,8.31589617161871)
(97,8.19976279420752)
(98,8.18967260723272)
(99,8.37069393810889)
(100,8.23037470713979)

};
\addplot [thick, blue]
coordinates {
(0,23930.000884189)
(1,7180.45388604586)
(2,3617.86527996726)
(3,2206.80793392946)
(4,1491.00767628939)
(5,1073.99586354884)
(6,808.278854547473)
(7,628.014922400163)
(8,499.960644278589)
(9,405.752764676303)
(10,334.52634655712)
(11,279.477813847677)
(12,236.164491398145)
(13,201.577684383878)
(14,173.611856832551)
(15,150.754899712087)
(16,131.896367104543)
(17,116.20926011348)
(18,103.06516983272)
(19,91.9789779656275)
(20,82.5721972749523)
(21,74.5480919899652)
(22,67.6703468931341)
(23,61.7492516151206)
(24,56.6308080266971)
(25,52.1895417181108)
(26,48.3230738518772)
(27,44.9473811889776)
(28,41.9908958545994)
(29,39.3943307887263)
(30,37.1076648133911)
(31,35.0895582332656)
(32,33.3047098499916)
(33,31.7230458561848)
(34,30.3185369243055)
(35,29.0691392036425)
(36,27.9557168948839)
(37,26.9619473758925)
(38,26.0735819006833)
(39,25.2785051309949)
(40,24.5658953302441)
(41,23.9261996677162)
(42,23.3512128923731)
(43,22.8338555363931)
(44,22.3679803523897)
(45,21.9478508869475)
(46,21.5686369463874)
(47,21.2260325999356)
(48,20.9161983824383)
(49,20.6357394037178)
(50,20.3816052485759)
(51,20.1511846245288)
(52,19.942073913322)
(53,19.7522963849272)
(54,19.5798157329886)
(55,19.4229051369124)
(56,19.2800427060627)
(57,19.1499227239291)
(58,19.0313113044641)
(59,18.9230788312623)
(60,18.8242439624631)
(61,18.7339264319528)
(62,18.6513238709167)
(63,18.5757271356125)
(64,18.5064950547755)
(65,18.4430397931058)
(66,18.3848534199294)
(67,18.3314621000904)
(68,18.282410130483)
(69,18.2373078847479)
(70,18.195837809721)
(71,18.1576476003302)
(72,18.1224457540708)
(73,18.0899698163812)
(74,18.0599809133384)
(75,18.032269842075)
(76,18.0066439340179)
(77,17.9829130684296)
(78,17.9609160188493)
(79,17.9405020453336)
(80,17.9215358365279)
(81,17.9038985280102)
(82,17.8874801575507)
(83,17.8721768211419)
(84,17.8578894936326)
(85,17.8445328788038)
(86,17.83202928813)
(87,17.8203078087826)
(88,17.8093094635342)
(89,17.7989701384791)
(90,17.7892345518416)
(91,17.7800530702916)
(92,17.7713803468455)
(93,17.7631748927227)
(94,17.755398724103)
(95,17.7480170503084)
(96,17.7409979934438)
(97,17.7343123346806)
(98,17.7279332839925)
(99,17.7218362708545)
(100,17.7160071042362)

};
\addplot [thick, green!50.0!black]
coordinates {
(0,23930.000884189)
(1,23923.334047724)
(2,23928.7635276026)
(3,23903.2228942111)
(4,23880.551116968)
(5,23870.1438694121)
(6,23840.8591097331)
(7,23800.5465231594)
(8,23693.5652072578)
(9,23631.2019253979)
(10,23595.7049294721)
(11,23557.3218465756)
(12,23539.2186492497)
(13,23473.584727042)
(14,23400.1812437473)
(15,23388.0720749918)
(16,23318.1474310779)
(17,23292.0698546307)
(18,23269.068415991)
(19,23210.8791988646)
(20,23193.4863330836)
(21,23181.6599796314)
(22,23062.4792306439)
(23,23035.3733843549)
(24,23032.8599854377)
(25,22939.6846047895)
(26,22929.2618733069)
(27,22867.1481481817)
(28,22818.677731829)
(29,22807.9258198671)
(30,22785.6420041927)
(31,22755.6532908351)
(32,22742.582665471)
(33,22725.8825926485)
(34,22684.1564937657)
(35,22606.4456396911)
(36,22572.1742036179)
(37,22554.3985911884)
(38,22522.8369496402)
(39,22469.3743043561)
(40,22448.1243780356)
(41,22414.5732307433)
(42,22391.7091715499)
(43,22379.7010436894)
(44,22323.1997260491)
(45,22288.681718955)
(46,22282.2596286649)
(47,22284.2692515112)
(48,22235.9625129174)
(49,22213.2722470128)
(50,22178.9652092496)
(51,22157.7238236404)
(52,22153.6879665373)
(53,22140.6889877556)
(54,22123.4074016335)
(55,22116.046144858)
(56,22111.9024374621)
(57,22105.9198104965)
(58,21964.1343917105)
(59,21929.7826902415)
(60,21919.2485920075)
(61,21830.9844862054)
(62,21821.6299921176)
(63,21808.0503615675)
(64,21797.9385095849)
(65,21784.6608946421)
(66,21769.8680785984)
(67,21748.6833833625)
(68,21674.4511468646)
(69,21661.9376809697)
(70,21507.9874738557)
(71,21486.1511327575)
(72,21443.970383019)
(73,21439.7646878615)
(74,21420.0980397293)
(75,21306.20071164)
(76,21273.4158974333)
(77,21230.9495093902)
(78,21220.6918082251)
(79,21209.498741541)
(80,21178.4949344347)
(81,21162.2436290185)
(82,21141.5973807421)
(83,21101.8211631996)
(84,21032.6648578308)
(85,21024.2079580824)
(86,20999.327086295)
(87,20968.6752743001)
(88,20931.3260880202)
(89,20886.1066199987)
(90,20863.1270327032)
(91,20850.5835406883)
(92,20791.7487241044)
(93,20780.2586328704)
(94,20667.450523141)
(95,20436.9578418679)
(96,20321.419385273)
(97,20306.0602165487)
(98,20230.0396648291)
(99,20199.7871509236)
(100,20187.8616636121)

};
\addplot [thick, black]
coordinates {
(0,23930.000884189)
(1,23890.7203112094)
(2,23878.2170334922)
(3,23872.9686180075)
(4,23866.5346978871)
(5,23850.8764373553)
(6,23787.6950107308)
(7,23775.2516717042)
(8,23762.1300268247)
(9,23748.3910080289)
(10,23742.4176277215)
(11,23737.2794479725)
(12,23712.9424883472)
(13,23708.2694816423)
(14,23695.5428051205)
(15,23690.6246255974)
(16,23687.9657700372)
(17,23682.3953826871)
(18,23653.6012341404)
(19,23649.4236352872)
(20,23604.2971428242)
(21,23599.4062344063)
(22,23569.6297513714)
(23,23556.0011900061)
(24,23549.9773609808)
(25,23537.2426445563)
(26,23525.0819362898)
(27,23517.6903376486)
(28,23494.4847770765)
(29,23488.1010901902)
(30,23464.1734774381)
(31,23458.8655867368)
(32,23456.126867557)
(33,23441.5265289338)
(34,23436.4208834939)
(35,23401.2764127029)
(36,23372.2130357179)
(37,23345.1245749014)
(38,23335.6753012462)
(39,23302.2316059798)
(40,23293.8727188189)
(41,23296.2855197425)
(42,23277.2518022093)
(43,23248.185842466)
(44,23227.8846968562)
(45,23211.1802037091)
(46,23165.8983053437)
(47,23144.5178907953)
(48,23134.0165866522)
(49,23107.3347892962)
(50,23099.1373874557)
(51,23072.4548553106)
(52,23067.2888466665)
(53,23054.9606379638)
(54,23044.7351663614)
(55,23037.7511586)
(56,23032.536131999)
(57,23013.8989655525)
(58,22993.535118503)
(59,22988.6210309045)
(60,22981.4397361105)
(61,22976.4924591746)
(62,22971.6682267528)
(63,22967.0362740143)
(64,22961.8842584499)
(65,22879.4871764632)
(66,22870.2420512314)
(67,22869.5838747355)
(68,22859.9466968665)
(69,22854.4506719323)
(70,22833.5518644733)
(71,22829.9975916655)
(72,22749.1493997893)
(73,22738.4968508306)
(74,22722.0365538912)
(75,22717.4156339043)
(76,22701.8070654364)
(77,22699.1574630432)
(78,22640.8706502276)
(79,22636.8312482253)
(80,22622.284223347)
(81,22544.9797946472)
(82,22536.1429896047)
(83,22529.6266187036)
(84,22524.6932752277)
(85,22512.0230096588)
(86,22493.5562713872)
(87,22442.2856687459)
(88,22432.8918264189)
(89,22391.350837251)
(90,22345.9832234223)
(91,22259.3110049608)
(92,22253.2353619672)
(93,22206.5548772089)
(94,22200.5719097136)
(95,22186.6237812275)
(96,22161.1787011814)
(97,22140.2994978854)
(98,22134.8075457781)
(99,22124.9060394855)
(100,22121.1279874691)

};
\path [draw=black, fill opacity=0] (axis cs:1,13)--(axis cs:1,13);

\path [draw=black, fill opacity=0] (axis cs:13,0)--(axis cs:13,0);

\path [draw=black, fill opacity=0] (axis cs:13,1)--(axis cs:13,1);

\path [draw=black, fill opacity=0] (axis cs:0,13)--(axis cs:0,13);

\end{axis}

\end{tikzpicture}}
   		\end{column}\hspace{-16pt}
   		\begin{column}{.3\textwidth}
   			\hspace{30pt} \scriptsize Batch size = 50
   			\vspace{10pt}
   			\resizebox{1.18\linewidth}{!}{% This file was created by matplotlib v0.1.0.
% Copyright (c) 2010--2014, Nico Schl�mer <nico.schloemer@gmail.com>
% All rights reserved.
% 
% The lastest updates can be retrieved from
% 
% https://github.com/nschloe/matplotlib2tikz
% 
% where you can also submit bug reports and leavecomments.
% 
\begin{tikzpicture}

\begin{axis}[
xlabel={Number of Iterations},
ylabel={Function Value},
xmin=0, xmax=100,
ymin=1, ymax=100000,
ymode=log,
axis on top,
legend entries={{0SR1},{PG},{S0SR1},{SPG}}
]
\addplot [thick, red]
coordinates {
(0,23930.000884189)
(1,7180.45388604586)
(2,1602.10980923053)
(3,580.81114454538)
(4,519.996842220967)
(5,203.222944097077)
(6,121.922299073436)
(7,51.6523543005163)
(8,59.2695972484554)
(9,30.819611673989)
(10,25.614751730395)
(11,21.7592732486774)
(12,22.654003528062)
(13,20.15049001864)
(14,19.850460675318)
(15,18.3912111072782)
(16,23.73822362313)
(17,17.8866027335036)
(18,17.8196671916423)
(19,17.7775123499805)
(20,18.0615078838142)
(21,17.7506027746239)
(22,17.7424949039984)
(23,17.7297358775166)
(24,18.0476551294605)
(25,17.7168284338285)
(26,17.7118227102102)
(27,17.897336767244)
(28,17.7264425474283)
(29,17.6583909070735)
(30,17.6466963489978)
(31,17.5726603204587)
(32,17.9785494963624)
(33,17.5007993751903)
(34,17.4774762719696)
(35,17.4693572907426)
(36,18.0844886660129)
(37,17.4544077868014)
(38,17.4488948638463)
(39,17.3764214969115)
(40,17.5899917220372)
(41,17.1689206218387)
(42,17.0926131263914)
(43,17.193691075103)
(44,19.0707895757655)
(45,16.9630825309593)
(46,16.9036371416858)
(47,16.8413981063643)
(48,16.8743377801165)
(49,16.6801578404075)
(50,16.5909092149981)
(51,15.9390711318578)
(52,15.2212036355576)
(53,13.1777255991126)
(54,11.3509943341348)
(55,10.0944946484959)
(56,9.71465222794144)
(57,9.40843618327981)
(58,9.31008072896225)
(59,9.12293542675185)
(60,8.97211792181375)
(61,9.96955580951941)
(62,8.8971778312592)
(63,8.89032770510936)
(64,8.846532027156)
(65,9.14275424383174)
(66,8.76428614407871)
(67,8.74549967678879)
(68,8.81264423855369)
(69,8.92224051152909)
(70,8.71372433712952)
(71,8.69686120584995)
(72,8.86066032777056)
(73,8.64994348063148)
(74,8.64177834237841)
(75,8.61074429525064)
(76,8.97413158797716)
(77,8.57239195780947)
(78,8.56084322589053)
(79,8.56535473569669)
(80,8.70378144008176)
(81,8.55111754917089)
(82,8.54693544975204)
(83,8.66605947747419)
(84,8.49089417601578)
(85,8.48541554539908)
(86,8.44561206967175)
(87,8.58014556574557)
(88,8.40335231266141)
(89,8.3639219974089)
(90,8.30542294510122)
(91,8.53249537616582)
(92,8.29202808913906)
(93,8.25359007491097)
(94,8.22495187877562)
(95,8.2493220091081)
(96,8.31589617161871)
(97,8.19976279420752)
(98,8.18967260723272)
(99,8.37069393810889)
(100,8.23037470713979)

};
\addplot [thick, blue]
coordinates {
(0,23930.000884189)
(1,7180.45388604586)
(2,3617.86527996726)
(3,2206.80793392946)
(4,1491.00767628939)
(5,1073.99586354884)
(6,808.278854547473)
(7,628.014922400163)
(8,499.960644278589)
(9,405.752764676303)
(10,334.52634655712)
(11,279.477813847677)
(12,236.164491398145)
(13,201.577684383878)
(14,173.611856832551)
(15,150.754899712087)
(16,131.896367104543)
(17,116.20926011348)
(18,103.06516983272)
(19,91.9789779656275)
(20,82.5721972749523)
(21,74.5480919899652)
(22,67.6703468931341)
(23,61.7492516151206)
(24,56.6308080266971)
(25,52.1895417181108)
(26,48.3230738518772)
(27,44.9473811889776)
(28,41.9908958545994)
(29,39.3943307887263)
(30,37.1076648133911)
(31,35.0895582332656)
(32,33.3047098499916)
(33,31.7230458561848)
(34,30.3185369243055)
(35,29.0691392036425)
(36,27.9557168948839)
(37,26.9619473758925)
(38,26.0735819006833)
(39,25.2785051309949)
(40,24.5658953302441)
(41,23.9261996677162)
(42,23.3512128923731)
(43,22.8338555363931)
(44,22.3679803523897)
(45,21.9478508869475)
(46,21.5686369463874)
(47,21.2260325999356)
(48,20.9161983824383)
(49,20.6357394037178)
(50,20.3816052485759)
(51,20.1511846245288)
(52,19.942073913322)
(53,19.7522963849272)
(54,19.5798157329886)
(55,19.4229051369124)
(56,19.2800427060627)
(57,19.1499227239291)
(58,19.0313113044641)
(59,18.9230788312623)
(60,18.8242439624631)
(61,18.7339264319528)
(62,18.6513238709167)
(63,18.5757271356125)
(64,18.5064950547755)
(65,18.4430397931058)
(66,18.3848534199294)
(67,18.3314621000904)
(68,18.282410130483)
(69,18.2373078847479)
(70,18.195837809721)
(71,18.1576476003302)
(72,18.1224457540708)
(73,18.0899698163812)
(74,18.0599809133384)
(75,18.032269842075)
(76,18.0066439340179)
(77,17.9829130684296)
(78,17.9609160188493)
(79,17.9405020453336)
(80,17.9215358365279)
(81,17.9038985280102)
(82,17.8874801575507)
(83,17.8721768211419)
(84,17.8578894936326)
(85,17.8445328788038)
(86,17.83202928813)
(87,17.8203078087826)
(88,17.8093094635342)
(89,17.7989701384791)
(90,17.7892345518416)
(91,17.7800530702916)
(92,17.7713803468455)
(93,17.7631748927227)
(94,17.755398724103)
(95,17.7480170503084)
(96,17.7409979934438)
(97,17.7343123346806)
(98,17.7279332839925)
(99,17.7218362708545)
(100,17.7160071042362)

};
\addplot [thick, green!50.0!black]
coordinates {
(0,23930.000884189)
(1,23171.0346959744)
(2,21784.3535900778)
(3,20501.4769070109)
(4,19543.6665497666)
(5,18988.7934814984)
(6,18215.4373574971)
(7,17272.1454863206)
(8,16427.6602658124)
(9,15867.5434466811)
(10,14894.918119963)
(11,14116.9994755899)
(12,13726.9429460544)
(13,13327.6846049478)
(14,12811.0074148469)
(15,12439.6946147915)
(16,11982.3722133073)
(17,11569.3180134221)
(18,11039.307674824)
(19,10658.5139274633)
(20,10233.3532190102)
(21,9830.49141422135)
(22,9538.50386463762)
(23,9292.46215015677)
(24,8922.53723282377)
(25,8689.62695889746)
(26,8332.21531651746)
(27,8060.22179928855)
(28,7542.81062742628)
(29,7345.62352876317)
(30,7122.59280048562)
(31,6911.71867839273)
(32,6783.96852282346)
(33,6597.39813805498)
(34,6421.08199318435)
(35,6051.4440357494)
(36,5864.06903393834)
(37,5710.841520059)
(38,5465.94544022695)
(39,5295.42752247915)
(40,5083.37809781545)
(41,5014.56044826766)
(42,4956.89249180127)
(43,4798.11876413539)
(44,4767.91069849755)
(45,4619.00590806872)
(46,4538.59417059465)
(47,4433.87381330478)
(48,4343.63399486368)
(49,4249.49366868748)
(50,4147.51613532838)
(51,4008.73791472631)
(52,3920.33204121809)
(53,3799.16925271651)
(54,3724.07754306651)
(55,3652.86033114161)
(56,3609.46730550209)
(57,3390.97125217073)
(58,3290.83637561082)
(59,3241.56939297359)
(60,3127.06324030011)
(61,3051.82316650495)
(62,3019.50281905113)
(63,2962.39855462584)
(64,2879.09281679541)
(65,2821.50325676303)
(66,2754.65445009351)
(67,2716.91027069895)
(68,2612.14382872169)
(69,2563.32013516176)
(70,2520.70611327144)
(71,2476.5004250781)
(72,2374.27877130335)
(73,2344.16839504726)
(74,2312.75922747124)
(75,2277.12912174529)
(76,2269.54728601055)
(77,2205.90555954617)
(78,2147.79636872422)
(79,2123.52923930368)
(80,2097.70583017859)
(81,2079.83915939177)
(82,2001.2167455977)
(83,1958.97259960861)
(84,1932.49471599967)
(85,1910.78626157815)
(86,1888.27388183557)
(87,1822.21242558697)
(88,1787.25681803998)
(89,1737.2226512776)
(90,1724.81413377049)
(91,1698.50141287942)
(92,1664.30055132707)
(93,1632.82687249937)
(94,1557.6162509941)
(95,1522.83828155334)
(96,1464.39049704361)
(97,1433.11448377846)
(98,1417.71333251217)
(99,1407.86044704754)
(100,1344.81616763266)

};
\addplot [thick, black]
coordinates {
(0,23930.000884189)
(1,23293.1463670467)
(2,22553.387902309)
(3,21670.4864884541)
(4,20750.5513890745)
(5,20195.5275977338)
(6,19748.5819979135)
(7,19383.5430314739)
(8,18780.5418113084)
(9,18448.36860308)
(10,17959.6641837916)
(11,17442.0382999429)
(12,17013.0300143071)
(13,16529.8968496443)
(14,16056.6251422336)
(15,15808.1473900858)
(16,15491.2327974059)
(17,15160.0547738003)
(18,14839.8815790667)
(19,14475.8853962597)
(20,13994.7673749869)
(21,13735.2666879705)
(22,13271.7153440532)
(23,12772.5132825963)
(24,12495.3540557656)
(25,12171.4905558794)
(26,11963.292344086)
(27,11651.528551017)
(28,11303.7409235124)
(29,10950.1265296381)
(30,10684.058074616)
(31,10462.6972990673)
(32,10260.749041893)
(33,9942.4477782177)
(34,9755.90949540406)
(35,9543.06727328901)
(36,9375.21884479077)
(37,9225.93150626459)
(38,9017.44188999023)
(39,8700.02041152976)
(40,8572.1977841637)
(41,8416.28349705419)
(42,8243.77586374936)
(43,8012.48872705011)
(44,7896.27520735391)
(45,7771.15490274834)
(46,7695.1076200625)
(47,7583.17136596472)
(48,7333.10767096815)
(49,7186.55480315562)
(50,7042.13174645903)
(51,6973.849547127)
(52,6887.87235435164)
(53,6738.61901654716)
(54,6684.7944770361)
(55,6475.48700108552)
(56,6338.15776584224)
(57,6214.65158988481)
(58,6129.54348884943)
(59,6010.823316054)
(60,5869.57996643339)
(61,5761.15179212112)
(62,5708.69916893628)
(63,5622.76122546336)
(64,5544.05502537033)
(65,5478.69288104323)
(66,5403.22194656423)
(67,5341.90714131825)
(68,5222.34612750201)
(69,5121.1741698018)
(70,5087.49640237379)
(71,5014.18001774599)
(72,4899.9352399649)
(73,4822.63375697015)
(74,4742.89006151635)
(75,4677.02107086117)
(76,4592.91755663964)
(77,4530.91586617531)
(78,4420.99631661626)
(79,4376.80079193637)
(80,4279.9603821111)
(81,4227.02127969156)
(82,4188.01996451806)
(83,4138.89458389439)
(84,4074.734896419)
(85,3996.98838466048)
(86,3944.18752651912)
(87,3858.37275016319)
(88,3794.27356111269)
(89,3710.63525674627)
(90,3650.23006769274)
(91,3604.3861952728)
(92,3534.83163771068)
(93,3503.49394371603)
(94,3457.10600021213)
(95,3398.18132367632)
(96,3361.46674483991)
(97,3292.33317109272)
(98,3252.42254004838)
(99,3214.94409221704)
(100,3151.14396888867)

};
\path [draw=black, fill opacity=0] (axis cs:1,13)--(axis cs:1,13);

\path [draw=black, fill opacity=0] (axis cs:13,0)--(axis cs:13,0);

\path [draw=black, fill opacity=0] (axis cs:13,1)--(axis cs:13,1);

\path [draw=black, fill opacity=0] (axis cs:0,13)--(axis cs:0,13);

\end{axis}

\end{tikzpicture}}
   		\end{column}\hspace{-16pt}
   		\begin{column}{.3\textwidth}
   			\hspace{30pt} \scriptsize Batch size = 150
   			\vspace{10pt}
   			\resizebox{1.18\linewidth}{!}{% This file was created by matplotlib v0.1.0.
% Copyright (c) 2010--2014, Nico Schl�mer <nico.schloemer@gmail.com>
% All rights reserved.
% 
% The lastest updates can be retrieved from
% 
% https://github.com/nschloe/matplotlib2tikz
% 
% where you can also submit bug reports and leavecomments.
% 
\begin{tikzpicture}

\begin{axis}[
xlabel={Number of Iterations},
ylabel={Function Value},
xmin=0, xmax=100,
ymin=1, ymax=100000,
ymode=log,
axis on top,
legend entries={{0SR1},{PG},{S0SR1},{SPG}}
]
\addplot [thick, red]
coordinates {
(0,23930.000884189)
(1,7180.45388604586)
(2,1602.10980923053)
(3,580.81114454538)
(4,519.996842220967)
(5,203.222944097077)
(6,121.922299073436)
(7,51.6523543005163)
(8,59.2695972484554)
(9,30.819611673989)
(10,25.614751730395)
(11,21.7592732486774)
(12,22.654003528062)
(13,20.15049001864)
(14,19.850460675318)
(15,18.3912111072782)
(16,23.73822362313)
(17,17.8866027335036)
(18,17.8196671916423)
(19,17.7775123499805)
(20,18.0615078838142)
(21,17.7506027746239)
(22,17.7424949039984)
(23,17.7297358775166)
(24,18.0476551294605)
(25,17.7168284338285)
(26,17.7118227102102)
(27,17.897336767244)
(28,17.7264425474283)
(29,17.6583909070735)
(30,17.6466963489978)
(31,17.5726603204587)
(32,17.9785494963624)
(33,17.5007993751903)
(34,17.4774762719696)
(35,17.4693572907426)
(36,18.0844886660129)
(37,17.4544077868014)
(38,17.4488948638463)
(39,17.3764214969115)
(40,17.5899917220372)
(41,17.1689206218387)
(42,17.0926131263914)
(43,17.193691075103)
(44,19.0707895757655)
(45,16.9630825309593)
(46,16.9036371416858)
(47,16.8413981063643)
(48,16.8743377801165)
(49,16.6801578404075)
(50,16.5909092149981)
(51,15.9390711318578)
(52,15.2212036355576)
(53,13.1777255991126)
(54,11.3509943341348)
(55,10.0944946484959)
(56,9.71465222794144)
(57,9.40843618327981)
(58,9.31008072896225)
(59,9.12293542675185)
(60,8.97211792181375)
(61,9.96955580951941)
(62,8.8971778312592)
(63,8.89032770510936)
(64,8.846532027156)
(65,9.14275424383174)
(66,8.76428614407871)
(67,8.74549967678879)
(68,8.81264423855369)
(69,8.92224051152909)
(70,8.71372433712952)
(71,8.69686120584995)
(72,8.86066032777056)
(73,8.64994348063148)
(74,8.64177834237841)
(75,8.61074429525064)
(76,8.97413158797716)
(77,8.57239195780947)
(78,8.56084322589053)
(79,8.56535473569669)
(80,8.70378144008176)
(81,8.55111754917089)
(82,8.54693544975204)
(83,8.66605947747419)
(84,8.49089417601578)
(85,8.48541554539908)
(86,8.44561206967175)
(87,8.58014556574557)
(88,8.40335231266141)
(89,8.3639219974089)
(90,8.30542294510122)
(91,8.53249537616582)
(92,8.29202808913906)
(93,8.25359007491097)
(94,8.22495187877562)
(95,8.2493220091081)
(96,8.31589617161871)
(97,8.19976279420752)
(98,8.18967260723272)
(99,8.37069393810889)
(100,8.23037470713979)

};
\addplot [thick, blue]
coordinates {
(0,23930.000884189)
(1,7180.45388604586)
(2,3617.86527996726)
(3,2206.80793392946)
(4,1491.00767628939)
(5,1073.99586354884)
(6,808.278854547473)
(7,628.014922400163)
(8,499.960644278589)
(9,405.752764676303)
(10,334.52634655712)
(11,279.477813847677)
(12,236.164491398145)
(13,201.577684383878)
(14,173.611856832551)
(15,150.754899712087)
(16,131.896367104543)
(17,116.20926011348)
(18,103.06516983272)
(19,91.9789779656275)
(20,82.5721972749523)
(21,74.5480919899652)
(22,67.6703468931341)
(23,61.7492516151206)
(24,56.6308080266971)
(25,52.1895417181108)
(26,48.3230738518772)
(27,44.9473811889776)
(28,41.9908958545994)
(29,39.3943307887263)
(30,37.1076648133911)
(31,35.0895582332656)
(32,33.3047098499916)
(33,31.7230458561848)
(34,30.3185369243055)
(35,29.0691392036425)
(36,27.9557168948839)
(37,26.9619473758925)
(38,26.0735819006833)
(39,25.2785051309949)
(40,24.5658953302441)
(41,23.9261996677162)
(42,23.3512128923731)
(43,22.8338555363931)
(44,22.3679803523897)
(45,21.9478508869475)
(46,21.5686369463874)
(47,21.2260325999356)
(48,20.9161983824383)
(49,20.6357394037178)
(50,20.3816052485759)
(51,20.1511846245288)
(52,19.942073913322)
(53,19.7522963849272)
(54,19.5798157329886)
(55,19.4229051369124)
(56,19.2800427060627)
(57,19.1499227239291)
(58,19.0313113044641)
(59,18.9230788312623)
(60,18.8242439624631)
(61,18.7339264319528)
(62,18.6513238709167)
(63,18.5757271356125)
(64,18.5064950547755)
(65,18.4430397931058)
(66,18.3848534199294)
(67,18.3314621000904)
(68,18.282410130483)
(69,18.2373078847479)
(70,18.195837809721)
(71,18.1576476003302)
(72,18.1224457540708)
(73,18.0899698163812)
(74,18.0599809133384)
(75,18.032269842075)
(76,18.0066439340179)
(77,17.9829130684296)
(78,17.9609160188493)
(79,17.9405020453336)
(80,17.9215358365279)
(81,17.9038985280102)
(82,17.8874801575507)
(83,17.8721768211419)
(84,17.8578894936326)
(85,17.8445328788038)
(86,17.83202928813)
(87,17.8203078087826)
(88,17.8093094635342)
(89,17.7989701384791)
(90,17.7892345518416)
(91,17.7800530702916)
(92,17.7713803468455)
(93,17.7631748927227)
(94,17.755398724103)
(95,17.7480170503084)
(96,17.7409979934438)
(97,17.7343123346806)
(98,17.7279332839925)
(99,17.7218362708545)
(100,17.7160071042362)

};
\addplot [thick, green!50.0!black]
coordinates {
(0,23930.000884189)
(1,21975.8838823078)
(2,18879.7841116922)
(3,15960.0457997545)
(4,14447.1279373071)
(5,12820.1061318088)
(6,11421.3371094484)
(7,10030.9824217186)
(8,9262.35660898206)
(9,8585.4187508991)
(10,7688.96195002778)
(11,7132.34755549664)
(12,6525.2951182746)
(13,6027.09456653836)
(14,5597.23641138016)
(15,5187.22473075384)
(16,4772.05172482115)
(17,4471.68310363214)
(18,4209.37840725187)
(19,3983.04898585604)
(20,3667.91089385107)
(21,3482.61347510547)
(22,3204.98381841887)
(23,3005.41210512299)
(24,2789.48549555158)
(25,2596.21702301637)
(26,2432.36541193231)
(27,2273.67287544755)
(28,2121.48647499734)
(29,2027.04112150971)
(30,1967.84958661728)
(31,1893.26095480486)
(32,1827.94065349038)
(33,1742.67674887332)
(34,1620.77454333673)
(35,1554.37718117258)
(36,1459.00663185291)
(37,1367.10109079839)
(38,1272.10736049544)
(39,1203.39451094449)
(40,1106.49715073096)
(41,1057.04158786243)
(42,1008.40211835047)
(43,984.638316507915)
(44,958.578010233065)
(45,938.54976705447)
(46,898.464275861728)
(47,856.248510438271)
(48,782.28256338342)
(49,751.537914041033)
(50,735.781269501783)
(51,702.087572095052)
(52,670.348753496428)
(53,638.884372266288)
(54,618.771715950506)
(55,591.78179993537)
(56,553.545045310196)
(57,532.950511157198)
(58,525.270346627585)
(59,510.529645358556)
(60,478.261789311396)
(61,459.281523498543)
(62,441.941510943923)
(63,415.448885858652)
(64,408.870940091032)
(65,401.1861184411)
(66,375.52461046425)
(67,372.681837545989)
(68,360.648466528654)
(69,352.28337065206)
(70,329.30465767911)
(71,323.07630747367)
(72,307.498367140355)
(73,292.990481000015)
(74,282.478193315603)
(75,267.622742904421)
(76,264.93722965994)
(77,260.089696319324)
(78,250.507062682144)
(79,247.585988075163)
(80,242.134950230611)
(81,234.335317962934)
(82,220.50711900911)
(83,213.530194057356)
(84,213.158605459869)
(85,213.141689439488)
(86,209.134723198002)
(87,199.695600914175)
(88,191.479330978208)
(89,187.14012645406)
(90,183.283163341309)
(91,178.90686036297)
(92,176.978464251234)
(93,170.845146422236)
(94,160.318900295219)
(95,157.211520186089)
(96,155.730319261701)
(97,151.571310130795)
(98,147.790856841592)
(99,143.55714276405)
(100,139.731493391039)

};
\addplot [thick, black]
coordinates {
(0,23930.000884189)
(1,21965.7991947082)
(2,19839.111950648)
(3,18245.6603495555)
(4,16680.7461951488)
(5,15199.3720907056)
(6,14152.359426101)
(7,13249.1603187729)
(8,12106.876403638)
(9,11485.23537136)
(10,10624.0319832373)
(11,9998.23460040279)
(12,9422.70787630272)
(13,8821.25365208863)
(14,8245.35429931971)
(15,7889.63825592885)
(16,7451.142033771)
(17,6937.15210380027)
(18,6476.43919173826)
(19,6112.14465293491)
(20,5824.87229334082)
(21,5537.71260954863)
(22,5164.26550849617)
(23,4856.08827568902)
(24,4535.56040118895)
(25,4352.85351404823)
(26,4150.38773540579)
(27,3924.09232227433)
(28,3768.55661985695)
(29,3608.24629052429)
(30,3483.07998046816)
(31,3292.42823657164)
(32,3114.81719431215)
(33,2996.40682909122)
(34,2899.49477811012)
(35,2787.43820146127)
(36,2667.23824459715)
(37,2563.89074774176)
(38,2444.70473753739)
(39,2344.6641423738)
(40,2260.99267495662)
(41,2157.00635924888)
(42,2087.49262063616)
(43,2011.70388808009)
(44,1932.923696894)
(45,1885.79170445579)
(46,1829.26768287028)
(47,1771.93930339019)
(48,1682.8918326135)
(49,1629.22649488807)
(50,1591.83922210222)
(51,1556.38098178355)
(52,1508.11002119433)
(53,1462.74610811809)
(54,1431.13210322756)
(55,1366.41296056735)
(56,1333.16366228832)
(57,1299.31001305012)
(58,1263.86691844559)
(59,1234.98137203071)
(60,1204.51569914451)
(61,1164.96101924318)
(62,1120.26780370492)
(63,1100.66384629364)
(64,1076.0628189333)
(65,1044.44986761874)
(66,1014.34636686488)
(67,997.946852528628)
(68,961.128725048813)
(69,928.348639721778)
(70,909.686502121871)
(71,885.109210228062)
(72,861.830155219097)
(73,851.928861535559)
(74,827.245932841754)
(75,810.315966824559)
(76,788.915867943771)
(77,763.738854329862)
(78,742.715751501901)
(79,731.372101748885)
(80,704.591240684903)
(81,691.16383223837)
(82,674.66337409703)
(83,659.98195754727)
(84,648.240978785052)
(85,629.451397090187)
(86,616.966388764945)
(87,605.587561115596)
(88,599.04293861654)
(89,579.120434777272)
(90,571.689128156412)
(91,558.100643406932)
(92,545.52737212588)
(93,537.543618022914)
(94,528.687777844095)
(95,521.201881492171)
(96,512.015601796819)
(97,492.598839642987)
(98,480.437805137752)
(99,463.536973625467)
(100,458.21681935089)

};
\path [draw=black, fill opacity=0] (axis cs:1,13)--(axis cs:1,13);

\path [draw=black, fill opacity=0] (axis cs:13,0)--(axis cs:13,0);

\path [draw=black, fill opacity=0] (axis cs:13,1)--(axis cs:13,1);

\path [draw=black, fill opacity=0] (axis cs:0,13)--(axis cs:0,13);

\end{axis}

\end{tikzpicture}}
   		\end{column}
   	\end{columns}
   \end{frame}

\section{Logistic Regression: An Example}

\plain{Electroencephalography (EEG)\\
	\vspace{10pt}
	\alert{How deep is your sleep?}
	\vspace{15pt}\\
	\includegraphics[width=0.7\textwidth]{EEG_oscillation.png}\\
	\vspace{15pt}
	\small Sleeping patient / 20 nights of EEG recordings\\
	\small Predict next slow wave
	}

\begin{frame}{EEG: Logistic Regression}
	
\end{frame}

  \begin{frame}\frametitle{Results}
    Nice table with SQN, SGD (no reg, L2), (Lasso,) Prox (L1) showing
    Obj. value in found optimum, CPU time, Iterations, F1 score of prediction model

    \begin{table}[t]
    \centering
      \begin{tabular}{r|c|c|c}
        \phantom 0 & \textbf{$F(\omega^*)$} & \textbf{Model Score} & \textbf{Cost}\\
      \hline \alert{No regularization}   &      & &  \\
        SGD   & $0.01$  & $96\%$  & $x$ sec, $y$ AP\\
        SQN   & $0.5$  & $96\%$  & $x$ sec, $y$ AP\\
        Prox   & $0.01$  & $96\%$  & $x$ sec, $y$ AP\\
      \hline \hline \alert{L1}        &    &   &  \\
      LASSO & $.71$  &  $55\%$ & blablabla \\
        Prox   & $0.01$  & $96\%$  & $x$ sec, $y$ AP\\
      \hline \hline \alert{L2}        &    &   &  \\
       SGD & $.71$  &  $55\%$ & blablabla \\
        SQN   & $0.01$  & $96\%$  & $x$ sec, $y$ AP\\
      \end{tabular}
    \end{table}
  \end{frame}

\section{Dictionary Learning}
\plain{Dictionary Learning\\
	\vspace{10pt}
	\alert{Can we recover the image?}
	\vspace{15pt}\\
	\includegraphics[width=0.4\textwidth]{lena_pic.jpg}\\
	\vspace{15pt}
	\small Image is partially destroyed\\
	\small Reconstruct image
}
\begin{frame}{Dictionary Learning}
	bla
\end{frame}

\section{Conclusion}

  \begin{frame}{Summary}
    \begin{center}\ccbysa\end{center}
  \end{frame}
  

  \plain{Questions?}

  \begin{frame}[allowframebreaks]\frametitle{Main References}

    \bibliography{refs}
    \bibliographystyle{abbrv}

  \end{frame}

\section{Appendix}
\begin{frame}{Proximal Method}
	\centering\includegraphics[width = 0.9\textwidth]{ProxNormal_full.png}
\end{frame}
\begin{frame}{Proximal Method}
	\begin{columns}[T]
		\begin{column}{.5\textwidth}
			$F(x) = \lVert Ax - b \rVert + \lambda \lVert x \rVert_1$\\
			$A \in \mathbb{R}^{1500 \times 3000},\:b \in \mathbb{R}^{1500}$\\
			$A_{ij},\:b_i\:$ \textasciitilde $\:\mathcal{N}(0,1)$, $\:\lambda = 0.1$\\
			\vspace{28pt}
			\resizebox{\linewidth}{!}{% This file was created by matplotlib v0.1.0.
% Copyright (c) 2010--2014, Nico Schl�mer <nico.schloemer@gmail.com>
% All rights reserved.
% 
% The lastest updates can be retrieved from
% 
% https://github.com/nschloe/matplotlib2tikz
% 
% where you can also submit bug reports and leavecomments.
% 
\begin{tikzpicture}

\begin{axis}[
xlabel={Number of Iterations},
ylabel={Convergence Factor},
xmin=0, xmax=600,
ymin=0, ymax=1.4,
axis on top,
legend entries={{0SR1},{ProxGrad},{L-BFGS-B}}
]
\addplot [thick, red]
coordinates {
(0,0.998470080733672)
(1,0.994940962302861)
(2,0.996959799062693)
(3,0.99630996025186)
(4,0.998816099268038)
(5,0.995857741035573)
(6,0.997376826797568)
(7,0.995297191400582)
(8,0.997383759440146)
(9,0.997768991964815)
(10,0.997632334345562)
(11,0.995576421949935)
(12,0.996359926467174)
(13,0.995688185660593)
(14,0.998423243785616)
(15,0.995921746550061)
(16,0.998338146767779)
(17,0.993989060037129)
(18,0.997338012424794)
(19,0.997991514626505)
(20,0.996533244338816)
(21,0.997366821976355)
(22,0.996533811895126)
(23,0.997312603151186)
(24,0.997926073128897)
(25,0.997377218965451)
(26,0.99726926021805)
(27,0.996812743651606)
(28,0.998147000126286)
(29,0.996388130601992)
(30,0.998267912078671)
(31,0.996483875261679)
(32,0.997849757680263)
(33,0.993436588226564)
(34,0.999250659271823)
(35,0.999227161121785)
(36,0.989309786826051)
(37,0.998056786757621)
(38,0.999264151421749)
(39,0.997213258599441)
(40,0.995506407271311)
(41,0.99778407378576)
(42,0.995511598456749)
(43,0.997968883682375)
(44,0.995939542884016)
(45,0.998590654988278)
(46,0.992300619367407)
(47,0.997461831498417)
(48,0.997583296327934)
(49,0.996012482088423)
(50,0.997300651718407)
(51,0.997491517758997)
(52,0.995480362732228)
(53,0.99713521431069)
(54,0.996008389515033)
(55,0.998467131133506)
(56,0.996985249129562)
(57,0.995955871850529)
(58,0.995829404826631)
(59,0.996030223396214)
(60,0.997720804658173)
(61,0.994558564669144)
(62,0.996918518733037)
(63,0.997733957599643)
(64,0.99292969745297)
(65,0.998436789703839)
(66,0.995488519490871)
(67,0.988585569567711)
(68,0.995282039781278)
(69,0.999473236159331)
(70,0.996422624492982)
(71,0.993032824553451)
(72,0.996857067502211)
(73,0.992718979481587)
(74,0.996711951366108)
(75,0.995278062849048)
(76,0.996232738181895)
(77,0.995832947288513)
(78,0.995638018991299)
(79,0.996126495505451)
(80,0.99491792527774)
(81,0.995892422085085)
(82,0.993757355438417)
(83,0.996891937432182)
(84,0.992539173346602)
(85,0.997113609717012)
(86,0.98269336058232)
(87,0.999036391309447)
(88,0.999777054874061)
(89,0.998519981685207)
(90,0.995976272199881)
(91,0.990067920948323)
(92,0.997773563901056)
(93,0.998755921102475)
(94,0.991516972579533)
(95,0.998060315212439)
(96,0.997886162207093)
(97,0.995679179758377)
(98,0.997712797432933)
(99,0.993618392422604)
(100,0.997781387833242)
(101,0.998265546483169)
(102,0.991992839480001)
(103,0.998190273408278)
(104,0.99683865556405)
(105,0.990831493012037)
(106,0.996088155440601)
(107,0.998422460948507)
(108,0.994041277997111)
(109,0.99664399309482)
(110,0.992086203914159)
(111,0.997937165121879)
(112,0.997289161919175)
(113,0.994086368798453)
(114,0.993960423051011)
(115,0.997344851173483)
(116,0.997001895873719)
(117,0.99479416109568)
(118,0.992637662806528)
(119,0.996322279961984)
(120,0.998335724288531)
(121,0.991263226472158)
(122,0.99825718268897)
(123,0.982949409198468)
(124,0.996428232236067)
(125,1.00095873801046)
(126,1.00027489444991)
(127,1.00173544142378)
(128,1.0054814344426)
(129,1.00110517408765)
(130,1.00259961856169)
(131,1.00296555421992)
(132,1.00311415181748)
(133,1.00153550229598)
(134,1.00194511752155)
(135,1.00172657004743)
(136,1.00310082234005)
(137,1.00203517877576)
(138,1.00043201413998)
(139,1.00112995097693)
(140,1.00325252762415)
(141,1.00026507666983)
(142,1.00079028374499)
(143,1.00394698332505)
(144,0.999747488604653)
(145,0.999946827155586)
(146,0.999540225217286)
(147,0.998641320050316)
(148,0.998749406896581)
(149,0.998284906127838)
(150,0.998383157217318)
(151,0.998418708342679)
(152,0.999155248483573)
(153,0.998147428275314)
(154,0.998595186923791)
(155,0.998588451303644)
(156,0.998647965425365)
(157,0.998074576540096)
(158,0.998215186728932)
(159,0.998885243087912)
(160,0.997608555473223)
(161,0.998271852569424)
(162,0.997925084376533)
(163,0.997154152379926)
(164,0.997171686180147)
(165,0.993929503325571)
(166,0.997522728083163)
(167,0.998232398165397)
(168,0.991094426117647)
(169,0.996424778208847)
(170,0.989965559588179)
(171,0.99701234708412)
(172,0.991047267431153)
(173,0.995261878103358)
(174,0.993946855475755)
(175,0.994617074163186)
(176,0.994141429467023)
(177,0.996365205985127)
(178,0.993825242339399)
(179,0.996060702125544)
(180,0.988118354143931)
(181,0.99731730156443)
(182,0.996083492871505)
(183,0.984805382030801)
(184,0.994171984751282)
(185,0.998699439071764)
(186,0.979346997688537)
(187,0.998358099809218)
(188,0.989339684584011)
(189,0.98352502934184)
(190,0.997108014987962)
(191,0.996630337785789)
(192,0.988787726389681)
(193,0.995238967270926)
(194,0.994393382859809)
(195,0.991250890081155)
(196,0.98977843656732)
(197,0.991712799602542)
(198,0.993173820503475)
(199,0.991512720330977)
(200,0.991643575267925)
(201,0.990999765034199)
(202,0.993778574240679)
(203,0.987530149746226)
(204,0.991951656829993)
(205,0.994025674268546)
(206,0.984963736407547)
(207,0.993958251748623)
(208,0.980331457129376)
(209,0.977208508324933)
(210,0.993962250614333)
(211,0.994344848472363)
(212,0.978625169672577)
(213,0.991100043052331)
(214,0.977097561037346)
(215,0.988428449896462)
(216,0.982820967547067)
(217,0.990848687789691)
(218,0.979799799021555)
(219,0.99151067695444)
(220,0.991901555131946)
(221,0.979708707693253)
(222,0.984496861088189)
(223,0.985107931505002)
(224,0.988682759255747)
(225,0.981123670510407)
(226,0.993085571434269)
(227,0.989203426471157)
(228,0.974729580794793)
(229,0.983482352740098)
(230,0.992011934049144)
(231,0.979520246290111)
(232,0.991803558677558)
(233,0.943707353251044)
(234,0.993831504088062)
(235,1.00083250016996)
(236,1.00066662295716)
(237,1.00289183773102)
(238,1.02056996464158)
(239,1.00674464631744)
(240,1.02050467396545)
(241,1.00604007130889)
(242,1.00662791808469)
(243,1.00810144520017)
(244,1.00582144296329)
(245,1.01396606668458)
(246,1.00418052898777)
(247,1.00289781904686)
(248,1.01531177970227)
(249,1.00218484404827)
(250,1.00348931435468)
(251,1.02096776862141)
(252,1.00245875534772)
(253,1.00017725923797)
(254,1.00333918517117)
(255,1.00455323245988)
(256,1.00286410426716)
(257,1.00423598231876)
(258,1.00482462833005)
(259,1.00596263516757)
(260,1.00357154950055)
(261,1.00115394939353)
(262,1.00288366309317)
(263,1.00264222661459)
(264,1.0035923358198)
(265,1.00161731358895)
(266,1.00302395346993)
(267,1.00230668191929)
(268,1.00416340995028)
(269,1.00092603859146)
(270,1.0007456901956)
(271,1.00408660258647)
(272,1.00191595645929)
(273,1.00004664816735)
(274,1.00017157005921)
(275,1.00073952390294)
(276,1.00108205400284)
(277,1.00015694447412)
(278,1.00011251677952)
(279,1.00047499974379)
(280,1.00077386383289)
(281,1.00008133784146)
(282,1.00001888531866)
(283,1.00012590740727)
(284,1.0001256638814)
(285,1.00010670622764)
(286,1.00004944428723)
(287,1.00008006588617)
(288,0.999995353392804)
(289,1.0000853469937)
(290,0.999955021708739)
(291,0.999974288504309)
(292,0.999887534547668)
(293,0.999551763256655)
(294,0.999893969431871)
(295,0.999987370733744)
(296,0.999958976067061)
(297,0.999739104817383)
(298,0.998432328781285)
(299,0.998714349521607)
(300,0.998724916651262)
(301,0.999061640471079)
(302,0.998318819822458)
(303,0.998419819794849)
(304,0.99958446377999)
(305,0.998859703767087)
(306,0.998486525575541)
(307,0.999626341883683)
(308,0.999698939559748)
(309,0.997786547029421)
(310,0.999781317505322)
(311,0.999773819026931)
(312,0.99382567366253)
(313,0.998068172866546)
(314,1.00035655492373)
(315,1.00020133176143)
(316,1.00075369999957)
(317,1.0001403339329)
(318,1.00013084292531)
(319,1.00000313762645)
(320,0.999888299295017)
(321,0.999765111010468)
(322,0.999728937348015)
(323,0.999705797627205)
(324,0.999365973625546)
(325,0.999204696793602)
(326,0.999325665511314)
(327,0.998953179694342)
(328,0.998912603775142)
(329,0.995634920736584)
(330,0.99954551556531)
(331,0.99983755285114)
(332,0.997847783588267)
(333,0.995463175083982)
(334,0.99979346004436)
(335,0.99932915678051)
(336,0.996723073253087)
(337,0.999398623497015)
(338,0.999208871210041)
(339,0.993095151745247)
(340,0.997828037295084)
(341,0.998951145311384)
(342,0.99131177129673)
(343,0.998015248382988)
(344,0.995431526774172)
(345,0.992493472469157)
(346,0.991958142482971)
(347,0.995378797667188)
(348,0.98987237231725)
(349,0.995192771559426)
(350,0.984696444277552)
(351,0.994987253589296)
(352,0.979898931056147)
(353,0.990650194406847)
(354,0.99034068231553)
(355,0.97494849314194)
(356,0.977258221318208)
(357,0.961728156311865)
(358,0.981101717605691)
(359,0.983959243624222)
(360,0.995478083686534)
(361,0.973394750502988)
(362,0.972704800644013)
(363,0.981606505713978)
(364,0.980418893902811)
(365,0.979096638766405)
(366,0.991605888499665)
(367,0.97405577927728)
(368,0.978225455437565)
(369,0.980669020984107)
(370,0.985756155864266)
(371,0.979436842624239)
(372,0.981454047294948)
(373,0.989689135177071)
(374,0.977959926603034)
(375,0.988634150931736)
(376,0.94042532295267)
(377,0.994415050542384)
(378,0.999437649301255)
(379,0.997643014885507)
(380,0.992581229636837)
(381,0.969750409120536)
(382,0.998367050294913)
(383,0.999402001198623)
(384,0.991892743937923)
(385,0.997358511569066)
(386,0.99166516610632)
(387,0.995995583544588)
(388,0.983268338930222)
(389,0.992735683196101)
(390,0.974732225034852)
(391,0.995557923595185)
(392,0.993262041990218)
(393,0.973584422841693)
(394,0.992676886560982)
(395,0.998323334364726)
(396,0.975533132767848)
(397,0.994930728346065)
(398,0.983544195669238)
(399,0.983012528390598)
(400,0.98329629100799)
(401,0.986740162421778)
(402,0.976435363185254)
(403,0.980919152730342)
(404,0.969288797827066)
(405,0.988405615205541)
(406,0.958279450982159)
(407,0.994692917775428)
(408,0.998046392608261)
(409,0.973767798883652)
(410,0.975586027947106)
(411,0.999635873608726)
(412,0.999607248616737)
(413,0.994937561913286)
(414,0.994917287811882)
(415,0.994417878531305)
(416,0.995004175875623)
(417,0.992972642340951)
(418,0.989284950629175)
(419,0.993466948642801)
(420,0.989502111777954)
(421,0.993341554499745)
(422,0.98289097485116)
(423,0.995823243718976)
(424,0.998317800543714)
(425,0.98414605377378)
(426,0.993158010404325)
(427,0.988786555629301)
(428,0.993719151197856)
(429,0.989913766204263)
(430,0.993828051914236)
(431,0.982767415529964)
(432,0.991872514120267)
(433,0.995965845185806)
(434,0.984575850934174)
(435,0.994674368029686)
(436,0.971876867920466)
(437,0.995916825701782)
(438,0.999464475740847)
(439,0.997815208051538)
(440,0.993096723371356)
(441,0.971729229160861)
(442,0.997630651370869)
(443,0.999425709090606)
(444,0.993092102644345)
(445,0.996678660296125)
(446,0.988232461381206)
(447,0.994120567960207)
(448,0.99433622022858)
(449,0.985622239974157)
(450,0.991018729645116)
(451,0.976592060013734)
(452,0.990709455594887)
(453,0.990430764186041)
(454,0.984182713102456)
(455,0.984997034277276)
(456,0.984938970561247)
(457,0.982277440519385)
(458,0.992634335850589)
(459,0.986089767331046)
(460,0.985231111798194)
(461,0.969292912657612)
(462,0.993291371402152)
(463,0.987140439091231)
(464,0.953555744004117)
(465,0.991365649198506)
(466,0.993782271082554)
(467,0.918447148130537)
(468,0.977095838620622)
(469,0.937439705999286)
(470,0.930562996215817)
(471,0.940390140268187)
(472,0.921999540729525)
(473,0.94585653597656)
(474,0.923277343503236)
(475,0.927555996541238)
(476,0.830916878664084)
(477,0.914563623395049)
(478,0.969953553791818)
(479,0.796048267920469)
(480,0.963320552207813)
(481,1.26189787900771)
(482,1.03900549755231)
(483,0.99612395133274)
(484,0.993871647140713)
(485,0.980915093858808)
(486,0.820873350162104)
(487,0.862115335898704)
(488,0.905506970217271)
(489,0.896483484474868)
(490,0.866382285867771)
(491,0.72374606851546)
(492,0.845837214704159)
(493,0.912623086653703)
(494,0.587883559235896)
(495,0.913209346822776)
(496,0.540610309189249)
(497,0.0824964802033573)
(498,0)

};
\addplot [thick, blue]
coordinates {
(0,0.99798751553595)
(1,0.997983492880239)
(2,0.997979454020933)
(3,0.997975398860242)
(4,0.997971327299238)
(5,0.997967239238506)
(6,0.997963134577575)
(7,0.997959013215392)
(8,0.997954875049898)
(9,0.997950719978241)
(10,0.997946547896774)
(11,0.997942358700968)
(12,0.997938152285482)
(13,0.997933928544062)
(14,0.99792968736953)
(15,0.997925428653931)
(16,0.997921152288414)
(17,0.997916858163136)
(18,0.997912546167282)
(19,0.997908216189343)
(20,0.997903868116672)
(21,0.997899501835808)
(22,0.997895117232274)
(23,0.997890714190627)
(24,0.997886292594491)
(25,0.997881852326469)
(26,0.997877393268186)
(27,0.997872915300282)
(28,0.997868418302396)
(29,0.997863902153032)
(30,0.997859366729832)
(31,0.997854811909281)
(32,0.997850237566793)
(33,0.997845643576794)
(34,0.997841029812547)
(35,0.997836396146282)
(36,0.997831742449102)
(37,0.997827068590974)
(38,0.997822374440774)
(39,0.997817659866276)
(40,0.99781292473398)
(41,0.997808168909279)
(42,0.997803392256468)
(43,0.997798594638549)
(44,0.997793775917319)
(45,0.997788935953506)
(46,0.997784074606371)
(47,0.997779191734082)
(48,0.99777428719358)
(49,0.99776936084044)
(50,0.997764412529014)
(51,0.997759442112268)
(52,0.997754449441965)
(53,0.99774943436853)
(54,0.997744396740885)
(55,0.997739336406784)
(56,0.997734253212541)
(57,0.997729147003051)
(58,0.997724017621823)
(59,0.99771886491093)
(60,0.997713688711011)
(61,0.997708488861313)
(62,0.99770326519953)
(63,0.997698017561833)
(64,0.997692745783107)
(65,0.997687449696366)
(66,0.997682129133398)
(67,0.997676783924277)
(68,0.9976714138976)
(69,0.997666018880175)
(70,0.997660598697448)
(71,0.997655153173006)
(72,0.997649682129012)
(73,0.997644185385755)
(74,0.997638662761945)
(75,0.997633114074523)
(76,0.997627539138775)
(77,0.997621937768168)
(78,0.997616309774447)
(79,0.997610654967537)
(80,0.997604973155554)
(81,0.99759926414477)
(82,0.997593527739596)
(83,0.99758776374262)
(84,0.997581971954448)
(85,0.997576152173805)
(86,0.99757030419749)
(87,0.997564427820218)
(88,0.997558522834887)
(89,0.997552589032248)
(90,0.997546626200979)
(91,0.997540634127816)
(92,0.997534612597346)
(93,0.997528561391916)
(94,0.997522480291939)
(95,0.997516369075476)
(96,0.997510227518493)
(97,0.997504055394644)
(98,0.997497852475487)
(99,0.997491618530004)
(100,0.997485353325161)
(101,0.997479056625461)
(102,0.997472728193005)
(103,0.997466367787516)
(104,0.997459975166283)
(105,0.997453550084223)
(106,0.99744709229361)
(107,0.997440601544311)
(108,0.997434077583561)
(109,0.9974275201561)
(110,0.997420929003816)
(111,0.997414303866236)
(112,0.997407644480013)
(113,0.997400950579135)
(114,0.997394221894812)
(115,0.997387458155487)
(116,0.997380659086659)
(117,0.99737382441108)
(118,0.997366953848534)
(119,0.997360047115942)
(120,0.997353103927104)
(121,0.997346123992861)
(122,0.997339107021022)
(123,0.997332052716181)
(124,0.997324960779934)
(125,0.997317830910556)
(126,0.997310662803215)
(127,0.997303456149674)
(128,0.997296210638444)
(129,0.997288925954691)
(130,0.997281601780134)
(131,0.997274237792981)
(132,0.997266833668076)
(133,0.997259389076575)
(134,0.997251903686178)
(135,0.99724437716076)
(136,0.997236809160606)
(137,0.997229199342285)
(138,0.997221547358486)
(139,0.997213852857993)
(140,0.99720611548585)
(141,0.997198334882992)
(142,0.997190510686443)
(143,0.997182642528989)
(144,0.997174730039458)
(145,0.997166772842435)
(146,0.997158770558233)
(147,0.997150722802791)
(148,0.997142629187795)
(149,0.997134489320475)
(150,0.99712630280345)
(151,0.9971180692349)
(152,0.997109788208417)
(153,0.997101459312717)
(154,0.997093082131872)
(155,0.997084656245069)
(156,0.99707618122678)
(157,0.997067656646163)
(158,0.997059082067715)
(159,0.997050457050434)
(160,0.99704178114842)
(161,0.997033053910346)
(162,0.997024274879572)
(163,0.997015443594051)
(164,0.997006559586179)
(165,0.996997622382799)
(166,0.996988631505019)
(167,0.996979586468355)
(168,0.996970486782157)
(169,0.996961331950173)
(170,0.996952121469875)
(171,0.996942854832797)
(172,0.996933531524098)
(173,0.996924151022751)
(174,0.996914712801262)
(175,0.996905216325692)
(176,0.996895661055365)
(177,0.996886046443113)
(178,0.996876371934699)
(179,0.996866636969145)
(180,0.99685684097844)
(181,0.996846983387302)
(182,0.996837063613444)
(183,0.996827081066875)
(184,0.996817035150575)
(185,0.996806925259443)
(186,0.996796750780971)
(187,0.996786511094649)
(188,0.996776205572189)
(189,0.996765833576994)
(190,0.996755394464317)
(191,0.996744887581177)
(192,0.99673431226585)
(193,0.996723667848104)
(194,0.996712953649035)
(195,0.996702168980618)
(196,0.996691313145886)
(197,0.996680385438638)
(198,0.996669385143176)
(199,0.996658311534558)
(200,0.996647163877914)
(201,0.996635941428483)
(202,0.996624643431714)
(203,0.996613269122721)
(204,0.996601817726314)
(205,0.996590288456727)
(206,0.996578680517615)
(207,0.996566993101433)
(208,0.996555225389969)
(209,0.996543376553399)
(210,0.996531445750669)
(211,0.996519432128841)
(212,0.996507334823296)
(213,0.996495152957368)
(214,0.996482885641863)
(215,0.996470531975227)
(216,0.996458090812154)
(217,0.996445566272507)
(218,0.996432957955054)
(219,0.996420258884604)
(220,0.996407468253015)
(221,0.996394585169076)
(222,0.996381608694962)
(223,0.996368537860601)
(224,0.996355371670749)
(225,0.996342109108)
(226,0.996328749135147)
(227,0.996315290695602)
(228,0.996301732714779)
(229,0.996288074099687)
(230,0.996274313739508)
(231,0.99626045050538)
(232,0.996246483250477)
(233,0.99623241080964)
(234,0.996218231999349)
(235,0.996203945617549)
(236,0.996189550443161)
(237,0.996175045236012)
(238,0.99616042873652)
(239,0.996145699665329)
(240,0.996130856722782)
(241,0.996115898589037)
(242,0.996100823923137)
(243,0.996085631363255)
(244,0.996070319525703)
(245,0.996054887004725)
(246,0.996039332372494)
(247,0.996023654178074)
(248,0.996007850947303)
(249,0.995991921182316)
(250,0.995975863361007)
(251,0.995959675936709)
(252,0.995943357337388)
(253,0.995926905965555)
(254,0.995910320197458)
(255,0.995893598382446)
(256,0.995876738843033)
(257,0.995859739873591)
(258,0.995842599740074)
(259,0.995825316679697)
(260,0.995807888899931)
(261,0.995790314578113)
(262,0.995772591860881)
(263,0.995754718863076)
(264,0.995736693667703)
(265,0.995718514324899)
(266,0.995700178851216)
(267,0.995681685228908)
(268,0.995663031405333)
(269,0.995644215292123)
(270,0.995625234764224)
(271,0.995606087659579)
(272,0.995586771777724)
(273,0.995567284879333)
(274,0.995547624685181)
(275,0.995527788875296)
(276,0.995507775088306)
(277,0.995487580919832)
(278,0.99546720392232)
(279,0.995446641603438)
(280,0.995425891425431)
(281,0.9954049508041)
(282,0.995383817107222)
(283,0.995362487654217)
(284,0.995340959714296)
(285,0.995319230505669)
(286,0.995297297194391)
(287,0.995275156892977)
(288,0.99525280665924)
(289,0.995230243494734)
(290,0.99520746434396)
(291,0.995184466092224)
(292,0.9951612455648)
(293,0.995137799525621)
(294,0.995114124674875)
(295,0.995090217648449)
(296,0.995066075015931)
(297,0.995041693278651)
(298,0.995017068868581)
(299,0.994992198146358)
(300,0.994967077399033)
(301,0.994941702839158)
(302,0.994916070602201)
(303,0.994890176744477)
(304,0.994864017241815)
(305,0.994837587986666)
(306,0.994810884786665)
(307,0.99478390336199)
(308,0.994756639342993)
(309,0.994729088268267)
(310,0.994701245581941)
(311,0.99467310663099)
(312,0.994644666663169)
(313,0.994615920823784)
(314,0.994586864153211)
(315,0.994557491584236)
(316,0.994527797938432)
(317,0.994497777923865)
(318,0.994467426131517)
(319,0.99443673703229)
(320,0.994405704973496)
(321,0.994374324175267)
(322,0.994342588727249)
(323,0.994310492584812)
(324,0.994278029564911)
(325,0.994245193342585)
(326,0.99421197744643)
(327,0.994178375254765)
(328,0.994144379990857)
(329,0.994109984718481)
(330,0.994075182337447)
(331,0.994039965578294)
(332,0.994004326997475)
(333,0.993968258972046)
(334,0.993931753694435)
(335,0.993894803166526)
(336,0.993857399193823)
(337,0.993819533379637)
(338,0.99378119711858)
(339,0.993742381590221)
(340,0.993703077751998)
(341,0.99366327633273)
(342,0.993622967824727)
(343,0.993582142476886)
(344,0.99354079028594)
(345,0.993498900989248)
(346,0.993456464055545)
(347,0.993413468676755)
(348,0.993369903758047)
(349,0.993325757909248)
(350,0.993281019433926)
(351,0.993235676319762)
(352,0.993189716227624)
(353,0.993143126480033)
(354,0.99309589405023)
(355,0.993048005549147)
(356,0.992999447213467)
(357,0.992950204892018)
(358,0.992900264032318)
(359,0.992849609666035)
(360,0.992798226393905)
(361,0.992746098370643)
(362,0.992693209288148)
(363,0.992639542358611)
(364,0.99258508029698)
(365,0.992529805301885)
(366,0.992473699036978)
(367,0.99241674261003)
(368,0.992358916552038)
(369,0.992300200794734)
(370,0.992240574647984)
(371,0.992180016774618)
(372,0.992118505165973)
(373,0.992056017114257)
(374,0.991992529184878)
(375,0.991928017187866)
(376,0.991862456146033)
(377,0.991795820263777)
(378,0.991728082892077)
(379,0.991659216494381)
(380,0.991589192608105)
(381,0.991517981805626)
(382,0.991445553653716)
(383,0.991371876669405)
(384,0.991296918274995)
(385,0.991220644748971)
(386,0.991143021176963)
(387,0.99106401139732)
(388,0.990983577945365)
(389,0.990901681993604)
(390,0.990818283290463)
(391,0.990733340092999)
(392,0.99064680909773)
(393,0.990558645367507)
(394,0.990468802253007)
(395,0.990377231310608)
(396,0.990283882215758)
(397,0.990188702670065)
(398,0.990091638303957)
(399,0.989992632573128)
(400,0.989891626648533)
(401,0.989788559300313)
(402,0.989683366772994)
(403,0.989575982655721)
(404,0.989466337739897)
(405,0.989354359872951)
(406,0.989239973799128)
(407,0.98912310098958)
(408,0.989003659464549)
(409,0.988881563600822)
(410,0.988756723926043)
(411,0.988629046901813)
(412,0.988498434688847)
(413,0.988364784897152)
(414,0.988227990317986)
(415,0.988087938638376)
(416,0.98794451213178)
(417,0.987797587329137)
(418,0.987647034664316)
(419,0.987492718092947)
(420,0.987334494682025)
(421,0.987172214168887)
(422,0.987005718485131)
(423,0.986834841242212)
(424,0.986659407178516)
(425,0.986479231558076)
(426,0.986294119522902)
(427,0.986103865389359)
(428,0.985908251886099)
(429,0.985707049325827)
(430,0.985500014707032)
(431,0.985286890732621)
(432,0.985067404745528)
(433,0.984841267563514)
(434,0.9846081722093)
(435,0.984367792517903)
(436,0.984119781617607)
(437,0.983863770255425)
(438,0.983599364967645)
(439,0.983326146057383)
(440,0.983043665374747)
(441,0.982751443863809)
(442,0.982448968848674)
(443,0.982135691035646)
(444,0.981811021181038)
(445,0.981474326394054)
(446,0.981124926018947)
(447,0.980762087040829)
(448,0.980385018954276)
(449,0.979992868013313)
(450,0.979584710777973)
(451,0.979159546855194)
(452,0.978716290712867)
(453,0.978253762423057)
(454,0.977770677176957)
(455,0.97726563336395)
(456,0.976737098991946)
(457,0.976183396163579)
(458,0.97560268328578)
(459,0.974992934604599)
(460,0.974351916597618)
(461,0.973677160633827)
(462,0.972965931203675)
(463,0.972215188852899)
(464,0.971421546759306)
(465,0.970581219652359)
(466,0.969689963440555)
(467,0.968743003520803)
(468,0.967734949211851)
(469,0.966659691075631)
(470,0.965510276993166)
(471,0.964278761683983)
(472,0.962956022790995)
(473,0.961531534516408)
(474,0.959993086934958)
(475,0.958326435134737)
(476,0.956514856825299)
(477,0.954538589302639)
(478,0.952374105537979)
(479,0.949993173094603)
(480,0.947361615865792)
(481,0.944437663054972)
(482,0.941169715444304)
(483,0.937493274007626)
(484,0.93332663998224)
(485,0.928564772098406)
(486,0.923070308700943)
(487,0.916660100908471)
(488,0.909084400255061)
(489,0.899993558882081)
(490,0.888882529880896)
(491,0.874993742899446)
(492,0.857136730229758)
(493,0.833327379030201)
(494,0.799994286186319)
(495,0.749994645470268)
(496,0.666661909012014)
(497,0.499996433200721)
(498,0)

};
\addplot [thick, green!50.0!black]
coordinates {
(0,0.998384781518948)
(1,0.997699867682473)
(2,0.998187357585595)
(3,0.997807596410336)
(4,0.997768103957278)
(5,0.996957405918742)
(6,0.996290952630612)
(7,0.996787865290649)
(8,0.995957011612332)
(9,0.996072108047891)
(10,0.995436507042511)
(11,0.996077546665806)
(12,0.996099873437628)
(13,0.997466327761431)
(14,0.996363104970037)
(15,0.997614096722684)
(16,0.9971578489208)
(17,0.996845039459264)
(18,0.996396918849183)
(19,0.9956753246799)
(20,0.995713417290736)
(21,0.995461107179469)
(22,0.995832628291039)
(23,0.996460916755896)
(24,0.996908491502579)
(25,0.997113555070006)
(26,0.996801052298859)
(27,0.996241502475207)
(28,0.997608145454741)
(29,0.997370405944701)
(30,0.998028798270018)
(31,0.998035448794678)
(32,0.998031993699224)
(33,0.998045194524733)
(34,0.997746228492014)
(35,0.997765949726726)
(36,0.997459129774693)
(37,0.997308116507932)
(38,0.997132258166095)
(39,0.997387958942205)
(40,0.997528724370568)
(41,0.997656976424167)
(42,0.997939026359903)
(43,0.998008823352153)
(44,0.997936443763312)
(45,0.997654114647196)
(46,0.997419724880778)
(47,0.997272403092703)
(48,0.997407823180747)
(49,0.997739782577134)
(50,0.998055519419163)
(51,0.998172352630704)
(52,0.998130334407873)
(53,0.997972844218418)
(54,0.997757889319767)
(55,0.997538709301114)
(56,0.997424102838149)
(57,0.997472891165438)
(58,0.997600249461673)
(59,0.997676225701462)
(60,0.997661905203758)
(61,0.997615969950625)
(62,0.997564419522969)
(63,0.997469671645153)
(64,0.997334130450607)
(65,0.997239470936358)
(66,0.997258171875869)
(67,0.997345117948317)
(68,0.997407391070496)
(69,0.997426392807244)
(70,0.997439948619547)
(71,0.997436643430399)
(72,0.997367698141214)
(73,0.997241497899825)
(74,0.997155972896138)
(75,0.998452600067913)
(76,0.998114306948086)
(77,0.998238402136959)
(78,0.997840640794334)
(79,0.997655453541482)
(80,0.997251248675484)
(81,0.997002941630694)
(82,0.996726323923324)
(83,0.996406101896211)
(84,0.996352081782581)
(85,0.996237810985823)
(86,0.996926219678768)
(87,0.997437077295969)
(88,0.997242507579983)
(89,0.996556703937273)
(90,0.996483327525878)
(91,0.996182358862309)
(92,0.996591143613714)
(93,0.996606099344076)
(94,0.996691142583514)
(95,0.996704139296973)
(96,0.996647715505784)
(97,0.996455621355628)
(98,0.996298429693616)
(99,0.996282407487306)
(100,0.996293661168397)
(101,0.996177766859907)
(102,0.99608357284785)
(103,0.996225656851218)
(104,0.996453000603465)
(105,0.998443712152678)
(106,0.997532693433339)
(107,0.997400391871133)
(108,0.996602499362802)
(109,0.996511282279048)
(110,0.995967998677802)
(111,0.996709107632169)
(112,0.996159319644053)
(113,0.997512662782193)
(114,0.996653724933696)
(115,0.996965425246386)
(116,0.996412125166309)
(117,0.996508511475285)
(118,0.99585766560705)
(119,0.998117961280931)
(120,0.997405645762546)
(121,0.998143133144839)
(122,0.997955408643873)
(123,0.997896375621748)
(124,0.997180128947099)
(125,0.996651912557414)
(126,0.996249730183656)
(127,0.996093761932689)
(128,0.995781673363921)
(129,0.995728529793342)
(130,0.995924097757074)
(131,0.995862409454078)
(132,0.995593861984421)
(133,0.995432116365067)
(134,0.995527078273382)
(135,0.995568394344524)
(136,0.997446109069208)
(137,0.996901591388874)
(138,0.997491183182825)
(139,0.997501029759252)
(140,0.997452166207095)
(141,0.996806148018568)
(142,0.996159708567274)
(143,0.995694551959907)
(144,0.995556413271252)
(145,0.9954566427242)
(146,0.995865359047067)
(147,0.996460818514857)
(148,0.99648101803761)
(149,0.995855487936512)
(150,0.994992383251484)
(151,0.9946744143346)
(152,0.994968206103914)
(153,0.995275490472129)
(154,0.995219343841706)
(155,0.995061976267353)
(156,0.995183784956773)
(157,0.995530692795784)
(158,0.997862591043182)
(159,0.997588801984085)
(160,0.998424510006609)
(161,0.998317962384758)
(162,0.998131557238683)
(163,0.997497987609516)
(164,0.997128793728238)
(165,0.99683510887025)
(166,0.997318860518098)
(167,0.996487314391713)
(168,0.997170773421849)
(169,0.997639048671045)
(170,0.99808046033049)
(171,0.997855235315236)
(172,0.997286919943096)
(173,0.996734230547117)
(174,0.996703218660432)
(175,0.997066906165238)
(176,0.997554274302899)
(177,0.99782455403664)
(178,0.997736862067405)
(179,0.99738531091983)
(180,0.997080025709149)
(181,0.997150240718376)
(182,0.997471922462594)
(183,0.997599308764454)
(184,0.997350286665952)
(185,0.996952397625428)
(186,0.996885735514453)
(187,0.997273482981815)
(188,0.997798775332353)
(189,0.997964739552956)
(190,0.997871525959458)
(191,0.997356424090678)
(192,0.997180205562016)
(193,0.997230993444253)
(194,0.997671593952296)
(195,0.997925198865519)
(196,0.99787193975521)
(197,0.997576705145978)
(198,0.997381943280553)
(199,0.997496592557164)
(200,0.997889460218723)
(201,0.997976652010217)
(202,0.997949351079961)
(203,0.997585451743338)
(204,0.997310750250904)
(205,0.997309621826471)
(206,0.997539756797388)
(207,0.997660537933903)
(208,0.997533114757438)
(209,0.997220642442758)
(210,0.996971000895236)
(211,0.997042120645198)
(212,0.997474153419417)
(213,0.998190171744121)
(214,0.997847517218605)
(215,0.997666569909356)
(216,0.998457382737903)
(217,0.998063207571829)
(218,0.998344630025018)
(219,0.998107512504448)
(220,0.998068349268437)
(221,0.997740072415478)
(222,0.997623687063899)
(223,0.997664870437636)
(224,0.997817575374831)
(225,0.99759621231985)
(226,0.997339213519465)
(227,0.997271929406875)
(228,0.997119303955015)
(229,0.997015471238763)
(230,0.997347713253117)
(231,0.997105860910655)
(232,0.997360721584803)
(233,0.997110717380458)
(234,0.996820084212384)
(235,0.996528635089123)
(236,0.996530009908172)
(237,0.996530400408067)
(238,0.996501822459463)
(239,0.996415108009169)
(240,0.996636157931826)
(241,0.998146105199502)
(242,0.998076559455952)
(243,0.9985603100359)
(244,0.99835507169115)
(245,0.997901951635412)
(246,0.996973697902774)
(247,0.996158398198932)
(248,0.995375378874931)
(249,0.995094301458073)
(250,0.994623201045929)
(251,0.994886787745533)
(252,0.995576176980365)
(253,0.996087712085989)
(254,0.995932752328786)
(255,0.99503564824955)
(256,0.993840327689575)
(257,0.992975350220866)
(258,0.996681341406868)
(259,0.996416503817021)
(260,0.997796708109408)
(261,0.997097833907638)
(262,0.996779592052224)
(263,0.99502340809662)
(264,0.99448766025634)
(265,0.994008727860059)
(266,0.993985336636285)
(267,0.993225225419099)
(268,0.993385879462884)
(269,0.994306856905714)
(270,0.994840431393975)
(271,0.994646714520034)
(272,0.993739842397)
(273,0.992719351299219)
(274,0.991946721596671)
(275,0.991567764556073)
(276,0.991853183993781)
(277,0.992592672793482)
(278,0.996090213113956)
(279,0.994594919544864)
(280,0.994261995203486)
(281,0.99262430938993)
(282,0.993013723557302)
(283,0.992325664360567)
(284,0.992059405749935)
(285,0.99138843427876)
(286,0.991176282488076)
(287,0.99077119447884)
(288,0.991157268731496)
(289,0.992137864184772)
(290,0.992418076235837)
(291,0.991863226019847)
(292,0.995876447394573)
(293,0.994488741930312)
(294,0.997481718661567)
(295,0.99750379348791)
(296,0.997179117998075)
(297,0.995561687483986)
(298,0.994317248507469)
(299,0.993386662469559)
(300,0.993494810876755)
(301,0.993818844582005)
(302,0.994589132418366)
(303,0.995284948921721)
(304,0.995505930192567)
(305,0.99515628911568)
(306,0.994390500646387)
(307,0.99442049822504)
(308,0.994067854770962)
(309,0.994460421758078)
(310,0.994724730781556)
(311,0.994891742605968)
(312,0.994684793518944)
(313,0.994416820468799)
(314,0.994326987629313)
(315,0.994538824175444)
(316,0.99468179480302)
(317,0.994654980738135)
(318,0.99454117785519)
(319,0.994383485925508)
(320,0.994372861036904)
(321,0.994386155765128)
(322,0.994488753338825)
(323,0.994467654323422)
(324,0.994317973718325)
(325,0.993950762071321)
(326,0.993632869462415)
(327,0.993627550895174)
(328,0.993903930470571)
(329,0.994053237582696)
(330,0.99386542435863)
(331,0.993539614681781)
(332,0.993488425237733)
(333,0.996350364917513)
(334,0.996235564495462)
(335,0.996642572239853)
(336,0.996055578225192)
(337,0.995159794553666)
(338,0.994283745651885)
(339,0.993611466949935)
(340,0.993670158760192)
(341,0.993719577803573)
(342,0.993304373831393)
(343,0.993196701840368)
(344,0.99386684975662)
(345,0.994131955395007)
(346,0.994067817539982)
(347,0.995650406849506)
(348,0.99517135620354)
(349,0.995896868629817)
(350,0.99601668879581)
(351,0.995827655172278)
(352,0.995247871854145)
(353,0.994970434415332)
(354,0.993781685519375)
(355,0.993189597931531)
(356,0.992761094343306)
(357,0.99366851309964)
(358,0.994878167322324)
(359,0.995341752163057)
(360,0.994864075934695)
(361,0.993845150777735)
(362,0.992994169429959)
(363,0.992938231346187)
(364,0.993595167132545)
(365,0.994136144924123)
(366,0.994198013994815)
(367,0.993821396115147)
(368,0.993485830860605)
(369,0.993557281759686)
(370,0.993937122826447)
(371,0.994049755050903)
(372,0.993581506206529)
(373,0.993586495296973)
(374,0.993079065625524)
(375,0.993806865739666)
(376,0.994439072258276)
(377,0.994732005593367)
(378,0.994179281976816)
(379,0.993193938944196)
(380,0.992360256843684)
(381,0.992371772394586)
(382,0.99303218373424)
(383,0.993662577182572)
(384,0.993561609902747)
(385,0.992499727681375)
(386,0.991050916830789)
(387,0.9899680737674)
(388,0.989831590999351)
(389,0.990308471222209)
(390,0.990550714497668)
(391,0.990164506787447)
(392,0.989413794139097)
(393,0.98909335704734)
(394,0.992529352499841)
(395,0.991046346211424)
(396,0.996071304367468)
(397,0.994502434186795)
(398,0.995214632221789)
(399,0.993979230809329)
(400,0.993965834102369)
(401,0.992905226217982)
(402,0.991802863964734)
(403,0.989678935872725)
(404,0.987892805629111)
(405,0.987428376476552)
(406,0.98840126455827)
(407,0.990678367209653)
(408,0.991358549184544)
(409,0.990326940819817)
(410,0.988106073061908)
(411,0.986330998717079)
(412,0.985824258038945)
(413,0.986240173212441)
(414,0.986589627757017)
(415,0.986224987276404)
(416,0.985684453563891)
(417,0.985092102189871)
(418,0.985366646274413)
(419,0.985547901596453)
(420,0.984855824486416)
(421,0.982785284817266)
(422,0.979981148241715)
(423,0.978097966772613)
(424,0.978923433089631)
(425,0.981908523725307)
(426,0.984435249193611)
(427,0.984775798533871)
(428,0.983416805166635)
(429,0.982001180222335)
(430,0.981546973459689)
(431,0.981755360530358)
(432,0.981609629631226)
(433,0.980702203819672)
(434,0.979459025781152)
(435,0.978871907627699)
(436,0.979640417324965)
(437,0.981099173236175)
(438,0.981559465280714)
(439,0.980021409034979)
(440,0.9841869553306)
(441,0.98095073771759)
(442,0.982113784739291)
(443,0.981559038875676)
(444,0.980961858205216)
(445,0.97782426530629)
(446,0.975495725184817)
(447,0.974454617179887)
(448,0.975124766680534)
(449,0.974680391982112)
(450,0.974974506901884)
(451,0.977031760513834)
(452,0.976694660638338)
(453,0.974667183036795)
(454,0.970733461090552)
(455,0.967663450825885)
(456,0.966569842061474)
(457,0.966491242216805)
(458,0.965547974842191)
(459,0.963255213349519)
(460,0.961459002572125)
(461,0.961189370293739)
(462,0.960913117890811)
(463,0.960614927048244)
(464,0.960187853416254)
(465,0.959878224779174)
(466,0.959142268127887)
(467,0.958232035323104)
(468,0.984009466117456)
(469,0.983477634708826)
(470,0.987725026429103)
(471,0.984361056526701)
(472,0.978157744688188)
(473,0.966458719248256)
(474,0.957010817974654)
(475,0.949739995877049)
(476,0.942799495572639)
(477,0.930287040525448)
(478,0.92488799657942)
(479,0.93573355064783)
(480,0.941470874979964)
(481,0.934505493191853)
(482,0.910491409832247)
(483,0.881363138483438)
(484,0.859718238257544)
(485,0.849496836573083)
(486,0.847025912684471)
(487,0.844656491154314)
(488,0.832610165724455)
(489,0.79708490953717)
(490,0.72789764849168)
(491,0.632373458858738)
(492,0.764447301184836)
(493,0.792374752386914)
(494,0.804320539614492)
(495,0.721913150990058)
(496,0.665738280175891)
(497,0.506060509201089)
(498,0)

};
\path [draw=black, fill opacity=0] (axis cs:0,13)--(axis cs:0,13);

\path [draw=black, fill opacity=0] (axis cs:1,13)--(axis cs:1,13);

\path [draw=black, fill opacity=0] (axis cs:13,0)--(axis cs:13,0);

\path [draw=black, fill opacity=0] (axis cs:13,1)--(axis cs:13,1);

\end{axis}

\end{tikzpicture}}
		\end{column}\hfill
		\begin{column}{.5\textwidth}
			$F(x) = \lVert Ax - b \rVert + \lambda \lVert x \rVert_1$\\
			$A \in \mathbb{R}^{2197 \times 2197},\:b \in \mathbb{R}^{2197}$\\
			$A$: \small Discretization of 3D Laplacian\\
			\normalsize$\lambda = 1$\\
			\vspace{10pt}
			\resizebox{\linewidth}{!}{% This file was created by matplotlib v0.1.0.
% Copyright (c) 2010--2014, Nico Schl�mer <nico.schloemer@gmail.com>
% All rights reserved.
% 
% The lastest updates can be retrieved from
% 
% https://github.com/nschloe/matplotlib2tikz
% 
% where you can also submit bug reports and leavecomments.
% 
\begin{tikzpicture}

\begin{axis}[
xlabel={Number of Iterations},
ylabel={Convergence Factor},
xmin=0, xmax=20,
ymin=0, ymax=1.2,
axis on top,
legend entries={{0SR1},{ProxGrad},{L-BFGS-B}}
]
\addplot [thick, red]
coordinates {
(0,0.848084289429647)
(1,0.571531400706569)
(2,0.3183225225932)
(3,0.325478497212281)
(4,0.229202978836935)
(5,0)

};
\addplot [thick, blue]
coordinates {
(0,0.848084289429647)
(1,0.8761766786044)
(2,0.880353893444499)
(3,0.87631111105461)
(4,0.86750215311503)
(5,0.854873771332013)
(6,0.838545313511358)
(7,0.818309544143084)
(8,0.79368954997227)
(9,0.764405434650521)
(10,0.730785088934228)
(11,0.693056976527169)
(12,0.648410891181648)
(13,0.596467221748105)
(14,0.534977444909731)
(15,0.437803770596473)
(16,0.237479072114322)
(17,0)

};
\addplot [thick, green!50.0!black]
coordinates {
(0,1.03508957652092)
(1,0.953934356298228)
(2,0.781748660719143)
(3,0.348075428829538)
(4,0.392758852840426)
(5,0.321571723894868)
(6,0.494833621279672)
(7,0.62180136584259)
(8,0.66911034267718)
(9,0)

};
\path [draw=black, fill opacity=0] (axis cs:13,0)--(axis cs:13,0);

\path [draw=black, fill opacity=0] (axis cs:13,1)--(axis cs:13,1);

\path [draw=black, fill opacity=0] (axis cs:0,13)--(axis cs:0,13);

\path [draw=black, fill opacity=0] (axis cs:1,13)--(axis cs:1,13);

\end{axis}

\end{tikzpicture}}
		\end{column}
	\end{columns}
\end{frame}


\begin{frame}\frametitle{SQN: CPU Time}
    \resizebox{\linewidth}{!}{\input{'EEG Fixed Subset Objective vs. CPU time.tikz'}}

\begin{frame}{Proximal Method}
	\centering\includegraphics[width=0.9\textwidth]{lambda1.png}\\
	\centering\includegraphics[width=0.9\textwidth]{lambda2.png}

\end{frame}
\end{document}
