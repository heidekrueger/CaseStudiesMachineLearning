%%%%%%%%%%%%%%%%%%%%%%%%%%%%%%%%%%%%%%%%%%%%%%%%%%%%%%%%%%%%%%%%%%%%%%%%%
%%%%%%%%%%%%%%%%%%%%%%%%%%%%%%%%%%%%%%%%%%%%%%%%%%%%%%%%%%%%%%%%%%% Fonts
%%%%%%%%%%%%%%%%%%%%%%%%%%%%%%%%%%%%%%%%%%%%%%%%%%%%%%%%%%%%%%%%%%%%%%%%%
\usepackage[utf8]{inputenc} % Einlesen von Umlauten 
\usepackage[T1]{fontenc} % Darstellung von Umlauten
\usepackage[ngerman]{babel} % Silbentrennung
\newcommand{\hide}[1]{  }

\usepackage{hyperref}  % Referenzen
\usepackage{natbib} 	% Zitieren
\usepackage{ifthen} %% Fallunterscheidungen
	%\ifthenelse{\equal{#1}{\empty}}{#then}{#else}
	%\IfFileExists{file}{then-code}{else-code}
	%\InputIfFileExists{file}{then-code}{else-code}


%\usepackage{lmodern}
%\usepackage{concrete}% fancy fonts
%\usepackage{color} % for fancy section font and color
%\allsectionsfont{\color{section_color}\sffamily\selectfont}
%\usepackage{xcolor} % colour tricks ;)
%\definecolor{section_color}{rgb}{0.35,0.0,0}
%\usepackage{textcomp} % verschiedene Sonderzeichen, z.b. Euro
%%%%%%%%%%%%%%%%%%%%%%%%%%%%%%%%%%%%%%%%%%%%%%%%%%%%%%%%%%%%%%%%%%%%%%%%%
%%%%%%%%%%%%%%%%%%%%%%%%%%%%%%%%%%%%%%%%%%%%%%%%%%%%%% Seitenformatierung
%%%%%%%%%%%%%%%%%%%%%%%%%%%%%%%%%%%%%%%%%%%%%%%%%%%%%%%%%%%%%%%%%%%%%%%%%
%\usepackage{layout}  % \layout
%\usepackage{geometry} % page layout options
%\usepackage[a4paper]{typearea} %\a6paper: DIN A6 (1051mm x 1481mm) 
%\geometry{  
%			head = 5mm, foot = 5mm, 
%			landscape,
%			papersize={74mm,52mm},
%			text={60mm, 35mm},
%			left=3cm,right=3cm,top=2cm,bottom=3cm,
%			includeheadfoot
%}
%\usepackage{fancyhdr}
%\fancyhead[LCR]{bsp} 
%\fancyfoot[LCR]{bsp}
%\renewcommand{\headrulewidth}{0.5pt}
%\renewcommand{\footrulewidth}{0.5pt}
%\fancyhf{} % clear all header and footer fields
%\setlength{\textwidth}{15.5cm}
%\setlength{\textheight}{23cm}
%\setlength{\topmargin}{0cm}
%\setlength{\oddsidemargin}{0.5cm}
%\setlength{\evensidemargin}{0cm}
%\setlength{\parskip}{1ex}
%\setlength{\parindent}{0cm}
%\setcounter{tocdepth}{4}
%\setcounter{secnumdepth}{4}
%\headheight16.0pt
%%%%%%%%%%%%%%%%%%%%%%%%%%%%%%%%%%%%%%%%%%%%%%%%%%%%%%%%%%%%%%%%%%%%%%%%%
%%%%%%%%%%%%%%%%%%%%%%%%%%%%%%%%%%%%%%%%%%%%%%%%%%%%%%%% Textformatierung
%%%%%%%%%%%%%%%%%%%%%%%%%%%%%%%%%%%%%%%%%%%%%%%%%%%%%%%%%%%%%%%%%%%%%%%%%
%\usepackage{blindtext} % \blindtext
%\usepackage{indentfirst} % indent
%\setlength{\parskip}{0.3cm}
%\setlength{\parindent}{5pt}
% \renewcommand{\baselinestretch}{1.1}
%\marginparwidth = 10pt
%\footskip0mm
%\hoffset-10mm
%\voffset-5pt
%\normalheading % Überschriftengröße auf zur Schriftgröße angepaßt
%\usepackage{microtype} % für Blocksatz \flushright \flushleft
%\clubpenalty = 10000 % Disable single lines at the start of a paragraph (Schusterjungen)
%\widowpenalty = 10000 \displaywidowpenalty = 10000 % Disable single lines at the end of a paragraph (Hurenkinder)
\usepackage{multicol} % Mehrere Spalten \begin{multicols}{2}  \end{multicols}
%\usepackage{multirow}  % In Tabelle Zeilen zusammenfassen
%\usepackage{multicolumn} % In Tabelle Spalten zusammenfassen
%\usepackage{pdflscape}% landscape environment
%\usepackage[square,numbers]{natbib}% fuer Zitate
%\usepackage[babel,german=quotes]{csquotes}% fuer Zitate
%\newcommand{\Banach}{\textsc{Banach}} % Kapitaelchen
%\usepackage{longtable} % Mehrseitige Tabellen
%\usepackage{tabularx} % Tabellen mit fester Gesamtbreite und variablen Spalten

\hide{ 
\renewcommand\cite[2][]{ \ifthenelse{\equal{#1}{\empty}}
	{(\citeauthor{#2} \citeyear{#2})}
	{(\citeauthor{#2} \citeyear{#2}, S. #1)}
}
\newcommand\fcite[2][]{\ifthenelse{\equal{#1}{\empty}}
	{\footnote{\citeauthor{#2} \citeyear{#2}}.}
	{\footnote{\citeauthor{#2} \citeyear{#2}, S. #1}.}
}
}
%%%%%%%%%%%%%%%%%%%%%%%%%%%%%%%%%%%%%%%%%%%%%%%%%%%%%%%%%%%%%%%%%%%%%%%%%
%%%%%%%%%%%%%%%%%%%%%%%%%%%%%%%%%%%%%%%%%%%%%%%%% Innere Dokumentstruktur
%%%%%%%%%%%%%%%%%%%%%%%%%%%%%%%%%%%%%%%%%%%%%%%%%%%%%%%%%%%%%%%%%%%%%%%%%
%\usepackage[nottoc]{tocbibind}
%\usepackage{tocloft} % Table of Contents
%\setcounter{tocdepth}{5}  %  = Subparagraph
%\setcounter{secnumdepth}{5} 
%\renewcommand{\cftsecleader}{\cftdotfill{\cftdotsep}}
%\renewcommand{\theenumi}{\roman{enumi}} %% enumerate in i)
%\newcommand{\romanno}[1]{\MakeUppercase{\romannumeral #1{}}} 
%\newcommand{\romannom}[1]{\text{\MakeUppercase{\romannumeral #1{}}}} 
%\usepackage{sectsty} % fancy section layouts -> google if needed
%\usepackage[colorlinks=true, pdfborder={0 0 0}]{hyperref}
%%%%%\usepackage{hyperref} 	% for internal referntial links
%\hypersetup{
%    unicode=false,          % non-Latin characters in Acrobats bookmarks
%    pdfborder={0 0 0},       % border within links
%    pdftoolbar=true,        % show Acrobats toolbar?
%    pdfmenubar=true,        % show Acrobats menu?
%    pdffitwindow=false,     % window fit to page when opened
%    pdfstartview={FitH},    % fits the width of the page to the window
%    pdftitle={\lectureName\ - \lectureProf},    % title
%    pdfauthor={\authors},     % author
%    pdfsubject={\lectureName},   % subject of the document
%    pdfnewwindow=true,      % links in new window
%    colorlinks=false,       % false: boxed links; true: colored links
%    linkcolor=red,          % color of internal links
%    citecolor=green,        % color of links to bibliography
%    filecolor=magenta,      % color of file links
%    urlcolor=cyan           % color of external links
%    pagebackref=true        % activate back references inside bibliography
%    bookmarksdepth=subsubsection %bookmarks are added to this depth
%}
% index generation: >> makeindex $NAME 
%\usepackage{makeidx} %\index{example} \index{example!subentry} % \makeindex
% 'list of notations' generation: >> makeindex $NAME.nlo -s nomencl.ist -o $NAME.nls
%\usepackage[refpage]{nomencl}  % refer to the page where notation appears
%\renewcommand{\nomname}{Nomenclature} % \makenomenclature
%\renewcommand*{\pagedeclaration}[1]{\unskip\hfill\hyperpage{#1}} %\dotfill \nomenclature{Zeichen}{Erklärung}
%\newcommand{\itemi}{\item[i)]}
%\newcommand{\itemii}{\item[ii)]}
%\newcommand{\itemiii}{\item[iii)]}
%\newcommand{\itemiv}{\item[iv)]}
%%%%%%%%%%%%%%%%%%%%%%%%%%%%%%%%%%%%%%%%%%%%%%%%%%%%%%%%%%%%%%%%%%%%%%%%%
%%%%%%%%%%%%%%%%%%%%%%%%%%%%%%%%%%%%%%%%%%%%%%%%%%%%%%%%%%% Matheumgebung
%%%%%%%%%%%%%%%%%%%%%%%%%%%%%%%%%%%%%%%%%%%%%%%%%%%%%%%%%%%%%%%%%%%%%%%%%
\usepackage{
amsmath, % math environment 
%amssymb, % Synbols 
%stmaryrd, % Additional symbols
dsfont, % For mathematical symbols
%amsthm % %\newtheorem{name}{output}[derivation]
}	
%\usepackage[expert]{mathdesign} % additional fonts. e.g. \subseteq     
%\usepackage{tikz} % TikZ ist kein Zeichenprogramm
%\usetikzlibrary{calc,through,snakes,intersections}
%\usepackage{listings} % For the usage of code
%\lstset{numbers=left, numberstyle=\tiny, numbersep=5pt} \lstset{language=C}

%\usepackage{algorithm}
%\usepackage{algorithmic}

\usepackage{listings}
\usepackage{caption}
%\usepackage{upquote}
\usepackage{xcolor}

% This makes a gray box for the caption, with white text.
\DeclareCaptionFont{white}{\color{white}}
\DeclareCaptionFormat{listing}{\colorbox{gray}{\parbox{\textwidth}{#1#2#3}}}
\captionsetup[lstlisting]{format=listing,labelfont=white,textfont=white}

\lstset{
language=Python,
keywordstyle=\bfseries\ttfamily\color[rgb]{0,0,1},
identifierstyle=\ttfamily,
commentstyle=\color[rgb]{0.133,0.545,0.133},
stringstyle=\ttfamily\color[rgb]{0.627,0.126,0.941},
showstringspaces=false,
basicstyle=\small,
numberstyle=\footnotesize,
numbers=left,
stepnumber=1,
numbersep=10pt,
tabsize=2,
breaklines=true,
prebreak = \raisebox{0ex}[0ex][0ex]{\ensuremath{\hookleftarrow}},
breakatwhitespace=false,
aboveskip={1.5\baselineskip},
columns=fixed,
upquote=true,
extendedchars=true,
frame=bottomline,
inputencoding=utf8
}
% % % %\lstinputlisting[language=Python, label=samplecode,caption=Example for code from a file]{ filename.py or text}
%\usepackage{gnuplottex}
%\begin{gnuplot}
%set term epslatex color oldstyle rounded solid size 7cm,5cm
% ... gnuplot-Befehle
%\end{gnuplot}
%%%%%%%%%%%%%%%%%%%%%%%%%%%%%%%%%%%%%%%%%%%%%%%%%%%%%%%%%%%%%%%%%%%%%%%%%
%%%%%%%%%%%%%%%%%%%%%%%%%%%%%%%%%%%%%%%%%%%%%%%%%%%%%%%%%%%%%%%%%%%Bilder
%%%%%%%%%%%%%%%%%%%%%%%%%%%%%%%%%%%%%%%%%%%%%%%%%%%%%%%%%%%%%%%%%%%%%%%%%
%\usepackage{epsfig,bm,epsf,float} % Eps-Dateien
%\usepackage[hang,small,bf]{caption}
%\usepackage{subfigure, subcaption}
%\usepackage[pdftex,dvips]{graphicx}
%\pdfcompresslevel=9
%\begin{figure}[thbp] % top head bottom page
%	\centering
%	\includegraphics[width=4cm,height=4cm,scale=0.5,bb= 0 0 100 100]{bild}
%	\caption{Das hier ist ein 3D plot}
%	\label{fig:figlabel
%\end{figure}
%\usepackage{floatflt} % Von Text umflossenes Bild
%\begin{floatingfigure}[r]{breite}
%    \centering
%    \includegraphics{graphik.pdf}
%    \caption{Bildunterschrift}
%    \label{fig:figlabel}
%\end{floatingfigure}
%\usepackage{pdfpages} % Einbinden von Pdf-Dateien
%\usepackage{eso-pic} % ShipOutPicture
% \ClearShipoutPicture
% \AddToShipoutPicture{
%		\put(180,10){
%			 \parbox[b][\paperheight]{\paperwidth}{%
%			   bsp
%	} } }


%%%%%%%%%%%%%%CheckerHeader%%%%%%%%%%%%%%%%55
%\usepackage{delarray}
%\usepackage[babel,german=quotes]{csquotes}

%\usepackage{placeins}
%\usepackage{esint}
%\usepackage{bibgerm}

%\usepackage[numbers]{natbib} 
%\def\btxandlong{und}

% optischer Randausgleich, Verbesserung des Kernels ...
%\usepackage{microtype}
% Verschiedene Bug-Fixes
%\usepackage{mparhack}
% Fixt verschiedene Darstellungen, z.B. dass unnummerierte Kapitel 
% mit hyperref die korrekten Sprungmarken bekommen
%\usepackage[listings=false]{scrhack}
% Bookmarks im pdf+


