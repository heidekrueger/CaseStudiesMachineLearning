\documentclass[11pt]{beamer}
\usetheme{Rochester}

\usepackage[utf8]{inputenc}
\usepackage[german]{babel}
\usepackage{amsmath}
\usepackage{amsfonts}
\usepackage{amssymb}
\usepackage{graphicx}
\usepackage{tikz}

\usetikzlibrary{arrows, shapes, trees}
\tikzset{every node/.style = {rectangle, draw, fill = blue, minimum size = 1cm, text centered}}
\tikzset{edge from parent/.style = {draw}}

\author{Fin Bauer, Stanislas Chambon, Roland Halbig, \\Stefan Heidekrüger, Jakob Heuke}
\title{Stochastic Optimization in Machine Learning}
%\subtitle{}
%\logo{}
\institute{Technische Universität München}
%\date{}
%\subject{}
%\setbeamercovered{transparent}
\setbeamertemplate{navigation symbols}{}

\begin{document}
	
	\maketitle
	
	\begin{frame}
		\frametitle{Outline}
		\tableofcontents
	\end{frame}
	
	
	\section{Introduction}
	
		\begin{frame}
			\frametitle{Introduction (1)}
			\begin{block}{Status Quo}
				\begin{itemize}
					\item increasing amount of data in Machine Learning \\
					\item need of more and more robust and efficient algorithms \\
					\item ... \\
				\end{itemize}
			\end{block}
			\huge $\Rightarrow$ Stochastic Optimization
		\end{frame}
		
		\begin{frame}
			\frametitle{Introduction (2)}
			\begin{block}{Proposition}
				\begin{itemize}
					\item Stochastic Optimization
				\end{itemize}
			\end{block}
		\end{frame}
		
		\begin{frame}
			\frametitle{Structure of this Case Study}
			\begin{figure}
				\begin{tikzpicture}[grow = down, sibling distance = 5cm]
					\node {Problem}
					child{node {Implementing Stochastic BFGS}
						child{node {Testing on example}}}
					child{node {Implementing Proximal}
						child{node {Testing on example}}};
				\end{tikzpicture}
			\end{figure}
		\end{frame}
	
	
	\section{Stochastic Quasi-Newton Method}
	\subsection{Algorithm}
	
		\begin{frame}{Algorithm}
			Inhalt...
		\end{frame}
		
	\subsection{Application-Problem}
		
		\begin{frame}{Application-Problem}
			Inhalt...
		\end{frame}
		
	\subsection{Algorithm Benchmarking}
	
		\begin{frame}{Algorithm-Benchmarking}
			Inhalt...
		\end{frame}
	
	
	\section{Proximal Newton-type Method}
	\subsection{Algorithm}
	
		\begin{frame}{Algorithm}
			Inhalt...
		\end{frame}
	
	\subsection{Application-Problem}
	
		\begin{frame}{Application Problem}
			Inhalt...
		\end{frame}
	
	\subsection{Algorithm Benchmarking}
	
		\begin{frame}{Algorithm Benchmarking}
			Inhalt...
		\end{frame}
		
		
\end{document}