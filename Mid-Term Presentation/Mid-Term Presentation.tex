\documentclass[11pt]{beamer}
\usetheme{Rochester}

\usepackage[utf8]{inputenc}
\usepackage[german]{babel}
\usepackage{amsmath}
\usepackage{amsfonts}
\usepackage{amssymb}
\usepackage{graphicx}
\usepackage{tikz}
\usepackage{algorithm2e}
\usepackage{algorithmic}

\usetikzlibrary{arrows, shapes, trees}
\tikzset{every node/.style = {rectangle, fill = title.bg, text = title.fg, minimum size = 0.7 cm, text centered, font = \scriptsize, align = center}}
\tikzset{edge from parent/.style = {draw,->,thick}}

\author{Fin Bauer, Stanislas Chambon, Roland Halbig, \\Stefan Heidekrüger, Jakob Heuke}
\title{Stochastic Optimization in Machine Learning}
%\subtitle{}
%\logo{}
\institute{Technische Universität München}
%\date{}
%\subject{}
%\setbeamercovered{transparent}
\setbeamertemplate{navigation symbols}{}

\begin{document}
	
	\maketitle
	
	\begin{frame}
		\frametitle{Outline}
		\tableofcontents
	\end{frame}
	
	
	\section{Introduction}
	
	\begin{frame}{Introduction (1)}
		\begin{block}{Challenges in Machine Learning}
			\begin{itemize}
				\item massive amounts of training data 
				\item construction of very large models
				\item how to handle the high memory/computational demands?
			\end{itemize}
		\end{block}
		
		\begin{center}
			\huge{   $\Downarrow$ }
		\end{center}  
		\textbf{Stochastic Methods:} Update on small amounts of training data!
	\end{frame}
	
	\begin{frame}{Introduction (2)}
		\begin{block}{Optimization Problem}
			\begin{equation*}
				\min_{w\in \mathbb{R}^n} F(w) = \mathbb{E} \lbrack f(w; \xi) \rbrack,
			\end{equation*}
			where $f(w; \xi) = f(w; x_i, z_i)=\mathcal{L}(h(w; x_i); z_i))$.
		\end{block}
		\begin{block}{Empirical Form of Objective Function}
			\begin{equation*}
				F(w) = \frac{1}{N} \sum_{i = 1}^N f(w; x_i, z_i)
			\end{equation*}
		\end{block}
	\end{frame}
	
	\begin{frame}{Introduction (3)}
		\begin{block}{Mini-batch Stochastic Gradient}
			Consider small subset $\mathcal{S} \subset \{1, \dots, N\}$, with $b := |S| \ll N$ \\
			\vspace{0.3cm}
			Construct
			\begin{equation*}
				\widehat{\nabla} F(w)=\frac{1}{b} \sum_{i \in \mathcal{S} } \nabla f(w; x_i, z_i)
			\end{equation*}
		\end{block}
	\end{frame}	
	
	\begin{frame}{Structure of this Case Study}
		\begin{figure}
			\begin{tikzpicture}[grow = down, sibling distance = 6cm]
			\node {Choose Machine Learning Problems}
			child{node {Choose Appropriate Optimization Models}
				child{node {Implement Stochastic \\ Quasi-Newton Method}
					child{node (A) {Benchmark versus \\ State-of-the-Art Algorithms}} 
					edge from parent node[fill = none, text = fg, left = 0.2cm]
					{Smooth}}
				child{node {Implement \\ Proximal Methods}
					child{node (B) {Benchmark versus \\ State-of-the-Art Algorithms}}
					edge from parent node[fill = none, text = fg, right = 0.2cm]
					{Non-Smooth}}};
			\path (A) -- node (C) [below = 1cm]{Compare Methods} (B);
			\path (A) edge[->, thick] (C);
			\path (B) edge[->, thick] (C);
			\end{tikzpicture}
		\end{figure}
	\end{frame}
	
	
	\section{Machine Learning Problems and Optimization Models}
	
	\begin{frame}{Machine Learning Problems and Optimization Models}
		Possible Applications:
		\begin{itemize}
			\item Face recognition
			\item Text classification
			\item Speech recognition
		\end{itemize}
		\vspace{0.3cm}
		\pause
		Possible Optimization Models:
		\begin{itemize}
			\item Linear Regression: $\min_w \frac{1}{N} \sum^{N}_{i = 1} \| z_i - x_i w \|^2_2$
			\item Binary Classification: 
			\begin{equation*}
				f(w; x_i, z_i) = z_i \log(c(w; x_i)) + (1 - z_i) \log(1 - c(w; x_i))
			\end{equation*}
			with $c(w; x_i) = \frac{1}{1 + \exp(-x^T_i w)}$
			\item Neural Nets: Back propagation
		\end{itemize}
	\end{frame}
	
	\section{Stochastic Quasi-Newton Method}
	\subsection{Algorithm}
	
	\begin{frame}{Stochastic Quasi-Newton Method (1)}
		Problem:
		\begin{itemize}
			\item Incorporating second-order information via full Hessian too expensive for
			large-scale problems
			\item Use of curvature information highly beneficial for algorithm performance
		\end{itemize}
		\pause
		Idea:
		\begin{itemize}
			\item Adapt BFGS method to stochastic framework
			\item Employ limited memory version of BFGS algorithm (L-BFGS)
			\item Compute gradient based on sample $\mathcal{S}$ of training set
			\item Compute Hessian update at regular intervals of length $L$ based on small 
			subsample $\mathcal{S}_H$ of training set
		\end{itemize}
	\end{frame}
	
	\begin{frame}{Stochastic Quasi-Newton Method (2)}
		\begin{block}{Iteration}
			\vspace{-0.3 cm}
			\begin{equation*}
				w_{k+1} = w_k - \alpha_k H_t \widehat{\nabla}F(w_k)
			\end{equation*}
		\end{block}
		\begin{block}{Hessian-Update}
			Choose
			\begin{equation*}
				s_t = \bar{w}_t - \bar{w}_{t-1} \qquad y_t = \widehat{\nabla}^2 F(\bar{w}_t) s_t,
			\end{equation*}
			with $\bar{w}_t := \sum^k_{i = k-L} w_i$ and $\widehat{\nabla}^2 F(w)
			:= \frac{1}{b_H} \sum_{i \in \mathcal{S}_H} \nabla^2 f(w; x_i, z_i)$.\\
			\vspace{0.3 cm}
			Compute
			\begin{equation*}
				H_{t+1} = (I - \rho_t s_t y^T_t) H_t (I - \rho_t y_t s^T_t) + \rho_t s_t s^T_t,
			\end{equation*}
			with $\rho_t = \frac{1}{y^T_t s_t}$.
		\end{block}
	\end{frame}
	
	\begin{frame}{Stochastic Quasi-Newton Method (3)}
		\begin{block}{Stochastic L-BFGS Algorithm}
			\begin{algorithm}[H]
				\begin{algorithmic}[1]
					\STATE Initialize $w_1$, $H_1$, step-length sequence $\alpha_k > 0$ \\
					\FOR{$k = 1, \dots,$}
					\STATE Choose a sample $\mathcal{S} \subset \{1, \dots, N\}$
					\STATE Compute $w_{k+1} = w_k - \alpha^k H_t \widehat{\nabla}F(w^k)$
					\IF{$\mod(k, L) = 0$}
					\STATE Choose a sample $\mathcal{S}_H \subset \{1, \dots, N\}$ \\
					\STATE Compute $H_t$ \\
					\ENDIF
					\ENDFOR
				\end{algorithmic}
			\end{algorithm}
		\end{block}
	\end{frame}
	
	\subsection{Algorithm Benchmarking}
	
	\begin{frame}{Benchmarking of Stochastic Quasi-Newton Method}
		Challenge: Economical implementation of Algorithm is necessary for meaningful 
		benchmarking
		\begin{itemize}
			\item Memory-efficient sparse coding
			\item Calculation of Hessian-Vector Product without storing the Hessian
			\item Computation of BFGS-Update via two-loop recursion
		\end{itemize}
		\vspace{0.3cm}
		\pause
		Benchmarking:
		\begin{itemize}
			\item Comparison to Stochastic Gradient Descent Method, Standard L-BFGS Method, 
			(Stochastic) Conjugate Gradient Descent
			\item Comparison of run-time, accuracy, access-points etc. under different parameter 
			regimes and objective functions
		\end{itemize}
	\end{frame}
	
	
	\section{Sparsity: Proximal Methods}
	\subsection{Algorithm}
	
	\begin{frame}{Inducing Sparsity}
		\begin{block}{Dictionary Learning}
			\begin{equation*}
				\min_{D, \alpha} \frac{1}{N} \sum_{i=1}^N \|x_i - D \alpha_i \|_2^2 + \lambda \| 
				\alpha_i \|_1
			\end{equation*}
			\begin{itemize}
				\item control on $D$ and $\alpha$
				\item better convergence
				\item modifications of the algorithms
			\end{itemize}
		\end{block}
	\end{frame}
	
	\begin{frame}{Sparsity: General Formulation}
		\begin{block}{Problem}
			\begin{equation*}
				\min \; f(x) +g(x)
			\end{equation*}
		\end{block}
		\begin{block}{Proximal Gradient Method}
			\begin{align*}
				prox_{\lambda f}(v) &:= argmin_x f(x) + \frac{1}{2 \lambda} \|x - v \|^2\\
				x^{k+1} &:= prox_{\lambda^k g} \left(x^k - \lambda^k \nabla f(x^k) \right)
			\end{align*}
		\end{block}
	\end{frame}
	
	\begin{frame}{Sparse Formulation}
		\begin{block}{Proximal Gradient Method}
			Given $x^k$, $\lambda^{k-1}$, $\beta \in (0, 1)$\\
			Let $\lambda := \lambda^{k-1}$\\
			Repeat:\\
			\begin{enumerate}
				\item Let $z:= prox_{\lambda g} \left(x^k - \lambda \nabla f(x^k) \right)$
				\item Break if $f(z) \leq \hat{f}_{\lambda}(z, x^k)$
				\item Update $\lambda := \beta \lambda $
			\end{enumerate}
			Return $\lambda^k := \lambda, x^{k+1} := z$
		\end{block}
	\end{frame}
	
	\subsection{Conclusion}
	\begin{frame}{Conclusion}
		\begin{block}{Situation}
			\begin{itemize}
				\item Increasing amount of data in Machine Learning applications
				\item Need for robust and fast algorithms for smooth and non smooth optimization
			\end{itemize}
		\end{block}
		\begin{block}{Stochastic Second-Order Methods}
			\begin{itemize}
				\item Faster convergence through curvature information
				\item Moderate computational cost through mini-batches
			\end{itemize}
		\end{block}
	\end{frame}
	
	
\end{document}