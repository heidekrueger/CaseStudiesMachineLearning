\usepackage{
amsmath, % math environment 
amssymb, % Synbols 
%stmaryrd, % Additional symbols
dsfont, % For mathematical symbols
%amsthm % %\newtheorem{name}{output}[derivation]
}	
%\usepackage[expert]{mathdesign} % additional fonts. e.g. \subseteq     
%\usepackage{tikz} % TikZ ist kein Zeichenprogramm
%\usetikzlibrary{calc,through,snakes,intersections}

%\usepackage{gnuplottex}
%\begin{gnuplot}
%set term epslatex color oldstyle rounded solid size 7cm,5cm
% ... gnuplot-Befehle
%\end{gnuplot}

%%%%%%%%%%%%%%%%%%%%%%%%%%%%%%%%%%%%%%%%%%%%%%%%%%%%%%%%%%%%%%%%%%%%%%%%%
%%%%%%%%%%%%%%%%%%%%%%%%%%%%%%%%%%%%%%%%%%%%%%%%%%%%%%%%%%%%%%% Sonstiges
%%%%%%%%%%%%%%%%%%%%%%%%%%%%%%%%%%%%%%%%%%%%%%%%%%%%%%%%%%%%%%%%%%%%%%%%%
%\setbeamertemplate{footline}[frame number]%Seitenzahl in Fußnote
%\newcommand{\blankpage}{% Leerseite ohne Seitennummer, nächste Seite rechts
% \clearpage{\pagestyle{empty}\cleardoublepage}
%}
%\newcommand{\sub}[1]{\mbox{$_{\scriptstyle \rm #1}$}}
%\newcommand{\super}[1]{\mbox{$^{\scriptstyle \rm #1}$}}
%\renewcommand{\labelenumi}{\alph{enumi})}
%\newcommand{\grad}{\ensuremath{^\circ}}
% Kommentare
% Align Umgebung
\def\ba#1\ea{\begin{align*}#1\end{align*}}
\def\ban#1\ean{\begin{align}#1\end{align}}
\numberwithin{equation}{section} % reset numbering in each section
% Theoreom-Umgebung
%\newtheorem{dfn}{Definition}[subsection]
%\newtheorem{thm}[dfn]{Theorem}
%\newtheorem{lem}[dfn]{Lemma}
%\newtheorem{dfnlem}[dfn]{Definition und Lemma}
%\newtheorem{cor}[dfn]{Corollary}
%\newtheorem{eg}[dfn]{Beispiel}
%\newtheorem{note}[dfn]{Anmerkung:}
%\def\bthm#1\ethm{\begin{thm}#1\end{thm}}
%\renewcommand{\proof}{\textbf{Beweis: }}
%\renewcommand{\qedsymbol}{q.e.d.}  % \boxempty  % stmaryd
%\newcommand{\theorem} [1] {\begin{thm} [#1] 	\end{thm}  }
% Bild einfügen
%\newcommand{\img}[3]{
%	\begin{figure}[!htbp]
%		\centering
%		\includegraphics[]{#3}
%		\caption{#1}
%		\label{#2}
%	\end{figure}
%}
%% linear functional analysis
%\newcommand{\vectorz}[2]{\begin{pmatrix}	#1\\	#2\end{pmatrix}}
%\newcommand{\vectort}[3]{\begin{pmatrix}	#1\\	#2\\	#3\end{pmatrix}}

\usepackage{ifthen} %% Fallunterscheidungen

%scalar product
%\newcommand{\skp}[2]{\left( #1, #2 \right)}
\newcommand{\dbr}[2]{\langle #1, #2 \rangle}
%absolute value
%\newcommand{\abs}[2][\empty]{\ifthenelse{\equal{#1}{\empty}}{\left| #2 \right|}{\left| #2 \right|_{#1}}}
%norm
\newcommand{\norm}[2][\empty]{\ifthenelse{\equal{#1}{\empty}}{\left\| #2 \right\|}{\left\| #2 \right\|_{#1}}}
% math. Abkürzungen
\newcommand{\C}{\mathds{C}}   	\newcommand{\K}{\mathds{K}}
\newcommand{\R}{\mathds{R}}   	\newcommand{\N}{\mathds{N}}
\newcommand{\Q}{\mathds{Q}}   	\newcommand{\Z}{\mathds{Z}}
\newcommand{\sB}{\mathcal{B}} 	\newcommand{\sC}{\mathcal{C}}
\newcommand{\sK}{\mathcal{K}} 	\newcommand{\sR}{\mathcal{R}}
\newcommand{\sN}{\mathcal{N}} 	\newcommand{\sA}{\mathcal{A}}
\newcommand{\sM}{\mathcal{M}}
\newcommand{\sP}{\mathcal{P}} 	\newcommand{\sI}{\mathcal{I}}
\newcommand{\sL}{\mathcal{L}} 	\newcommand{\sO}{\mathcal{O}}
\newcommand{\sS}{\mathcal{S}} 	\newcommand{\sSs}{\mathcal{s}}
\newcommand{\pt}{\partial}		\renewcommand{\epsilon}{\varepsilon}
\newcommand{\dx}{\text{dx}} 	 	\newcommand{\ds}{\text{ds}}
\newcommand{\dt}{\text{dt}}	 	\newcommand{\dy}{\text{dy}}
\newcommand{\ifft}{\text{if and only if }}
%\renewcommand{\qed}{\hfill \square}
\newcommand{\half}[1]{\frac{#1}{2}}	%\newcommand{\over}[1]{\frac{1}{#1}}
\newcommand{\One}{\mathds{1}}
\newcommand{\weak}{\rightharpoonup}
\newcommand{\weaks}{\stackrel{*}{\rightharpoonup}}

%% Expectation Value and Variance
\newcommand\E[1]{\text{E}\left[ #1 \right]}
\newcommand\V[1]{\text{V}\left[ #1 \right]}
\newcommand\ld{\text{ld}}

%% Matrix notation
\newcommand\diag{\text{diag}}
\newcommand\D{\text{D}}

\newcommand{\argmax}{\operatorname*{arg\,max}}
\newcommand{\argmin}{\operatorname*{arg\,min}}
\newcommand{\myarray}[1]{\begin{pmatrix} #1 \end{pmatrix}}


