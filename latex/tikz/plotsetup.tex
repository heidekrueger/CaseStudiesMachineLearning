%% plots
\newcommand{\plotsetup}[4] {
	\pgfmathparse{#4} \pgfmathresult \let\maxY\pgfmathresult% evaluate maxY 
	\pgfmathparse{#3} \pgfmathresult \let\minY\pgfmathresult% evaluate minY	
 	\draw[very thin,color=gray] (#1-0.1,\minY) grid (#2+0.1,\maxY);;% GRID use minY & maxY
	\draw[->] (#1-0.1,\minY) -- (#2+0.1,\minY) node[right] {$x$};
	\draw[->] (#1,#3-0.1) -- (#1,#4+0.3) node[above] {$f(x)$};% y axis use minY & maxY too
	\foreach \y/\ytext in {#3,...,#4}
		\draw (1pt,\y cm) -- (-1pt,\y cm);
	\draw (-3pt,1.12 cm) node[anchor=east] {$1$};
	\draw (-3pt,0 cm) node[anchor=east] {$0$};
	\draw (1.1 cm,-2pt) node[anchor=north] {$1$};
	\foreach \x/\xtext in {#1,...,#2}
		\draw (\x cm,1pt) -- (\x cm,-1pt);% node[anchor=east] {$\x$};	
}

\newcommand{\plotsetupyf}[5] {
	\pgfmathparse{#4} \pgfmathresult \let\maxY\pgfmathresult% evaluate maxY 
	\pgfmathparse{#3} \pgfmathresult \let\minY\pgfmathresult% evaluate minY	
 	\draw[very thin,color=gray] (#1-0.1,\minY) grid (#2+0.1,\maxY);;% GRID use minY & maxY
	\draw[->] (#1-0.1,\minY) -- (#2+0.1,\minY) node[right] {$x$};
	\draw[->] (#1,#3-0.1) -- (#1,#4+0.3) node[above] {#5};% y axis use minY & maxY too
	\foreach \y/\ytext in {#3,...,#4}
		\draw (1pt,\y cm) -- (-1pt,\y cm);
	\draw (-3pt,1.12 cm) node[anchor=east] {$1$};
	\draw (-3pt,0 cm) node[anchor=east] {$0$};
	\draw (1.1 cm,-2pt) node[anchor=north] {$1$};
	\foreach \x/\xtext in {#1,...,#2}
		\draw (\x cm,1pt) -- (\x cm,-1pt);% node[anchor=east] {$\x$};	
}
%	\newcommand{\coordxmin}{#1} 		\newcommand{\coordxmax}{#2} 
%	\newcommand{\coordymin}{#3}			\newcommand{\coordymax}{#4}
%	\pgfmathparse{#4} \pgfmathresult \let\maxY\pgfmathresult% evaluate maxY 
%	\pgfmathparse{#3} \pgfmathresult \let\minY\pgfmathresult% evaluate minY
%		 \pgfmathparse{\maxY < 1} \pgfmathresult \let\BmaxY\pgfmathresult% What if maxY < 1? Then set Boolean to 1
%		   \ifthenelse{\equal{\BmaxY}{1.0}}{%
%		   \pgfmathparse{1.2} \pgfmathresult \let\maxY\pgfmathresult% Correct maxY to have correct graph
%		   }{};
%		 \pgfmathparse{\minY > 0} \pgfmathresult \let\BminY\pgfmathresult% What if minY > 0? Then set Boolean to 1
%		   \ifthenelse{\equal{\BminY}{1.0}}{%
%		   \pgfmathparse{0} \pgfmathresult \let\minY\pgfmathresult% Correct minY to have correct graph
%		   }{};
%        DRAW the graph of the function from here on
 
\hide{	% draw grid
	\draw[very thin,color=gray] (#1-0.1, #3 - 0.1) grid ( #2 + 0.1, #4 + 0.1);
    \draw[semithick,->] (#1-0.2,0) -- (#2+0.2,0) node[right] {$x$};
    \draw[->] (0,#3-0.2) -- (0,#4+0.2) node[above] {$f(x)$};
}
	% units for cartesian reference frame
%	\foreach \x in {0,1} 
 %       \draw (\x cm,1pt) -- (\x cm,-3pt)
  %          node[anchor=north,xshift=-0.15cm] {$\x$};
%	\draw (#2,1pt) -- (#2,-3pt)
%            node[anchor=north,xshift=-0.15cm] {$#2$};
%\ifthenelse{{\minY} <= 0 <= {\maxY}}{%
%		   \pgfmathparse{0} \pgfmathresult \let\minY\pgfmathresult% Correct minY to have correct graph
%		   }{};
   % \foreach \y/\ytext in {1}
    %    \draw (1pt,\y cm) -- (-3pt,\y cm) node[anchor=east] {$\ytext$};


\newcommand{\plotfunction}[7] {
	\begin{tikzpicture}[domain=#2:#3, #6]
		% setup
		\plotsetup{#2}{#3}{#4}{#5}
		%plots
		%	\draw plot file {Bsc.sin.table};
%		\draw[thin] plot[id=sin] function{#1} 
%		    node[right] {#7};
	\end{tikzpicture}

}

